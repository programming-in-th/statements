\documentclass[11pt,a4paper]{article}

\usepackage{../../templates/style}

\begin{document}

\begin{problem}{กระโดดสูง (HighJ)}{standard input}{standard output}{1 second}{32 megabytes}

คุณเล่นเกมกระโดดในตารางขนาด $n \times n$ ตารางหนึ่ง

            ตอนแรกคุณอยู่ที่ช่อง $(n,n)$ และคุณต้องการเดินทางไปยังช่อง $(1,1)$ โดยการกระโดดหลายๆครั้ง
            
            คุณสามารถกระโดดจากช่อง $(r,c)$ ใดๆ ไปยังช่อง $(r’,c’)$ ได้ ก็ต่อเมื่อ $r+c > r’+c’$เท่านั้น โดยคุณจะเสียพลังงานกระโดดไปทั้งสิ้น  $W[ r+c ][ r’+c’ ]$

            



\bigskip
\underline{\textbf{โจทย์}}  จงเขียนโปรแกรมเพื่อหาพลังงานที่น้อยที่สุดที่จะต้องใช้ในการกระโดดจาก $(n,n)$ ไปยัง $(1,1)$


\InputFile

\textbf{บรรทัดแรก} ประกอบด้วยจำนวนนับ $n$ แทนจำนวนแถวและคอลัมน์ของตาราง $( 2 \leq n \leq 300 )$

\textbf{บรรทัดที่ $2$ ถึง $2n+1$} จะแสดงถึงตาราง $W$ โดยบรรทัดที่ $i+1$ จะมีจำนวนนับ $2n$ จำนวน ซึ่งจำนวนนับที่ $j$ ของบรรทัดที่ $i+1$ จะแสดงค่าของ  $W[ i ][ j ]$  $( 1 \leq W[ i ][ j ] \leq 10\,000 )$


\OutputFile

\textbf{มีบรรทัดเดียว} แสดงค่าพลังงานที่น้อยที่สุดที่จะต้องใช้ในการกระโดดจาก $(n,n)$ ไปยัง $(1,1)$

\bigskip
\textbf{Note} สังเกตได้ว่าค่าของ $W[ i ][ j ]$ ที่ $i \leq j$ หรือ $i = 1$ หรือ $j = 1$ จะไม่สามารถนำมาคำนวณพลังงานการกระโดดได้ในกรณีใดๆทั้งสิ้น

\Examples

\begin{example}
\exmp{5
0 0 0 0 0 0 0 0 0 0 
0 0 0 0 0 0 0 0 0 0 
0 1 0 0 0 0 0 0 0 0 
0 1 3 0 0 0 0 0 0 0 
0 2 5 1 0 0 0 0 0 0 
0 8 4 2 2 0 0 0 0 0 
0 8 3 1 3 2 0 0 0 0 
0 9 4 1 6 6 1 0 0 0 
0 20 4 9 8 7 6 1 0 0 
0 20 14 18 15 1 1 3 2 0 }{3}%
\exmp{5
0 0 0 0 0 0 0 0 0 0 
0 0 0 0 0 0 0 0 0 0 
0 14 0 0 0 0 0 0 0 0 
0 30 15 0 0 0 0 0 0 0 
0 41 21 13 0 0 0 0 0 0 
0 51 42 22 11 0 0 0 0 0 
0 75 58 34 28 12 0 0 0 0 
0 67 71 44 37 23 14 0 0 0 
0 95 77 51 41 44 28 15 0 0 
0 96 94 66 72 41 37 30 11 0}{87}%
\end{example}

\Scoring 

\textbf{$50$\% ของชุดทดสอบทั้งหมด:} $n \leq 10$

           \textbf{ $100$\% ของชุดทดสอบทั้งหมด:} $n \leq 300$

  
\Source

สรวิทย์  สุริยกาญจน์ ( PS.int ) และแนวคิดจากค่ายสสวท. ค่ายที่ 2 ระยะ 2 ประจำปี 2554

ศูนย์ สอวน. โรงเรียนมหิดลวิทยานุสรณ์


\end{problem}

\end{document}