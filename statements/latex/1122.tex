\documentclass[11pt,a4paper]{article}

\usepackage{../../templates/style}

\begin{document}

\begin{problem}{ตั้งวงดนตรี (band)}{standard input}{standard output}{1 second}{32 megabytes}

คุณได้รับเชิญมาเป็นอาจารย์สอนวิชาคอมพิวเตอร์ในค่ายอบรมเข้มแห่งหนึ่ง ซึ่งมีนักเรียนอยู่ทั้งหมด $N$ คน คุณได้พบว่านักเรียนในค่าย นอกจากจะมีความสามารถในการเขียนโปรแกรมคอมพิวเตอร์แล้ว ยังมีความสามารถในการเล่นดนตรีอีกด้วย คุณจึงมีความคิดที่จะจัดตั้งวงดนตรี \textit{light music} ประจำค่ายขึ้นมา

วงดนตรี \textit{light music} นั้น ประกอบด้วยสมาชิก $4$ คน ซึ่งแต่ละคนก็จะมีหน้าที่ในการเล่นกีตาร์ เบส กลอง และคีย์บอร์ด คุณต้องการเลือกนักเรียนในค่าย $4$ คน มาเข้าร่วมเป็นสมาชิกวงดนตรี สำหรับนำไปแสดงในการแข่งขันดนตรีนานาชาติ \textit{International Olympiad in Instrument (IOI) ครั้งที่ 22} ที่ประเทศแคนาดา เพื่อสร้างชื่อเสียงให้กับประเทศชาติและค่ายอบรมเข้มแห่งนี้

อย่างไรก็ตาม ด้วยความที่ในค่ายมีนักเรียนอยู่เป็นจำนวนมาก ทำให้นักเรียนในค่ายไม่สามารถรู้จักกันได้อย่างทั่วถึง นักเรียนบางคนก็เป็นเพื่อนกัน ในขณะที่บางคนก็ไม่รู้จักกันเลย ความเป็นเพื่อนจะมีลักษณะสมมาตรเสมอ (นั่นคือถ้า $A$ เป็นเพื่อนกับ $B$ แล้ว $B$ ก็จะเป็นเพื่อนกับ $A$ ด้วย) และการที่จะตั้งวงดนตรีให้ได้อย่างมีประสิทธิภาพนั้น สมาชิกในวงจะต้องมีความสนิทสนมกลมเกลียวกันเป็นอย่างดี กล่าวคือ เมื่อเราพิจารณาคู่ของสมาชิก $2$ คนใดๆ ในวง ซึ่งมีอยู่ทั้งหมด $6$ คู่ จะต้องมีอย่างน้อย $5$ คู่ ที่เป็นเพื่อนกัน นั่นคือมีสมาชิกที่ไม่เป็นเพื่อนกันได้อย่างมากเพียงคู่เดียว

คุณต้องการทราบว่า มีวิธีในการเลือกนักเรียนในค่าย $4$ คน มาเป็นสมาชิกวงดนตรีอยู่ทั้งหมดกี่วิธี

\bigskip
\underline{\textbf{โจทย์}}  จงเขียนโปรแกรมเพื่อรับจำนวนนักเรียนในค่าย และความเป็นเพื่อนของนักเรียนแต่ละคู่ แล้วคำนวณว่ามีวิธีในการเลือกสมาชิกวงดนตรีอยู่ทั้งหมดกี่วิธี


\InputFile

\textbf{บรรทัดแรก} ระบุจำนวนเต็ม $N$ และ $M$ $(4 \leq N \leq 1\,000; 1 \leq M \leq 5\,000)$ แทนจำนวนนักเรียนในค่าย และจำนวนคู่ของนักเรียนที่เป็นเพื่อนกัน นักเรียนแต่ละคนจะมีหมายเลขประจำตัวตั้งแต่ $1, 2, 3$ เรียงไปเรื่อยๆ จนถึง $N$

\textbf{บรรทัดที่ $2$ ถึง $M+1$} ในแต่ละบรรทัดจะระบุจำนวนเต็ม $X$ และ $Y$ ที่แตกต่างกัน $(1 \leq X,Y \leq N) $ ซึ่งหมายความว่า นักเรียนหมายเลข $X$ เป็นเพื่อนกับนักเรียนหมายเลข $Y$



\OutputFile

\textbf{มีบรรทัดเดียว} แสดงจำนวนวิธีทั้งหมดในการเลือกสมาชิกวงดนตรี

\Examples

\begin{example}
\exmp{4 5
1 2
1 3
1 4
2 3
2 4}{1}%
\exmp{6 12
1 2
2 3
3 1
4 1
4 2
4 3
5 1
5 2
5 3
6 1
6 2
6 3}{12}%
\end{example}

\Scoring

\textbf{$30$\% ของข้อมูลทดสอบ:} $N \leq 100$

\Source

สุธี เรืองวิเศษ

การแข่งขัน IOI Thailand League เดือนมิถุนายน 2553


\end{problem}

\end{document}