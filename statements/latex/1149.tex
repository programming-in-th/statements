\documentclass[11pt,a4paper]{article}

\usepackage{../../templates/style}

\begin{document}

\begin{problem}{บีคุงบันคุง (2Bk)}{standard input}{standard output}{1 second}{32 megabytes}

 ณ โปรแกรมมิ่งดอทไอเอ็นดอททีเอช มีเด็กชายสองคนชื่อว่า \textit{"บีคุง"} และ \textit{"บันคุง"} ทั้งคู่รักกันมากและต้องการที่จะทดสอบความรักที่มีให้กัน พวกเขาจึงตัดสินใจที่จะเล่นเกมแห่งความรัก เกมแห่งความรักนี้บีคุงและบันคุงจะต้องถูกส่งเข้าไปในกระดานหมากรุกขนาด $N \times N$

                เริ่มต้น บีคุงกับบันคุงจะยืนอยู่คนละตำแหน่งกัน ทั้งคู่จะต้องพยายามเดินไปบนช่องของกระดานหมากรุก โดยจะเดินได้แค่ช่องบน ล่าง ซ้าย ขวา ของช่องที่อยู่ปัจจุบันเท่านั้น หากทั้งคู่สามารถเดินไปหากันได้ เกมจะจบลง

                แต่เกมทุกเกมก็ย่อมมีอุปสรรค ในแต่ละช่องของกระดานหมากรุก ยกเว้นช่องที่บีคุงกับบันคุงอยู่ในตอนเริ่มต้น จะมี \textit{"เด็กสาว”} อยู่ช่องละหนึ่งคน เด็กสาวแต่ละคนจะมี \textit{"ค่าเสน่ห์”} แตกต่างกัน เนื่องจากบีคุงและบันคุงก็ยังเป็น \textit{"ผู้ชาย"} พวกเขาจึงอาจเผลอหลงรักเด็กสาวกลางทางได้ การเดินผ่านเด็กสาวที่มีเสน่ห์มากๆ จึงไม่ดีต่อการพิสูจน์ความรักของเขาทั้งสอง

                บีคุงและบันคุงต้องการทราบว่า ค่าเสน่ห์ของเด็กสาวที่มากที่สุดที่เขาทั้งสองจะต้องทนระหว่างการเดินทางมา พบกันมีค่าได้น้อยที่สุดเป็นเท่าใด

\bigskip
\underline{\textbf{โจทย์}}  จงเขียนโปรแกรมรับข้อมูลตารางหมากรุก $N \times N$ และค่าเสน่ห์ของเด็กสาวในแต่ละช่อง แล้วแสดงค่าเสน่ห์ของเด็กสาวที่มากที่สุดบนเส้นทาง \textbf{ที่น้อยที่สุดที่เป็นไปได้} ที่พวกเขาต้องทนเผชิญในเส้นทางพิสูจน์รักของพวกเขา


\InputFile

\textbf{บรรทัดแรก} ระบุจำนวนเต็ม $N$ $(1 \leq  N \leq 1\,000)$ แทนขนาดของตารางหมากรุก

\textbf{บรรทัดที่ $2$ ถึง $N+1$} ในบรรทัดที่ $i + 1$ สำหรับ $1 \leq i \leq N$ ระบุจำนวนเต็ม $N$ จำนวนคั่นด้วยช่องว่างหนึ่งช่อง ในจำนวนที่ $j$ สำหรับ $1 \leq j \leq N$ จะแทนค่าเสน่ห์ของผู้หญิงที่อยู่ในแถวที่ $i$ คอลัมน์ที่ $j$ โดยที่ค่าเสนห์นี้มีค่าเป็นจำนวนเต็มบวกที่ไม่เกิน $1\,000\,000$ (เฉพาะช่องที่บีคุงและบันคุงอยู่เท่านั้นที่จะมีค่าเสน่ห์เป็น $0$)


\OutputFile

\textbf{มีบรรทัดเดียว} แสดงผลจำนวนเต็มเพียงจำนวนเดียวออกทางหน้าจอ เป็นค่าเสน่ห์ของเด็กสาวที่มากที่สุดที่เขาทั้งสองจะต้องทนระหว่างการเดิน ทางมาพบกันที่มีค่าน้อยที่สุด

\Examples

\begin{example}
\exmp{3
10 100 10
0 10 100
12 5 0}{10}%
\exmp{5
1 1 1 0 1
3 1 1 1 1
0 3 4 3 2
1 1 1 4 1
1 4 2 2 2}{2}%
\end{example}


\Source

จิรายุ ชิว ชิว ( Fur )

ศูนย์ สอวน. โรงเรียนมหิดลวิทยานุสรณ์

นักแสดง: \url{https://www.facebook.com/BkungUsbGod} และ\\ \url{https://www.facebook.com/profile.php?id=100001255970791}


\end{problem}

\end{document}