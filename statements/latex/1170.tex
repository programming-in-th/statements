\documentclass[11pt,a4paper]{article}

\usepackage{../../templates/style}

\begin{document}

\begin{problem}{ดันเจี้ยน (dungeon)}{standard input}{standard output}{0.5 second}{5 megabytes}

\textit{Dungeon} ลับแห่งหนึ่งมีลักษณะเป็นรูปกล่องสี่เหลี่ยมมุมฉากซึ่งถูกแบ่งออกเป็นห้องย่อยๆขนาด $W \times H \times L$ โดยมีทางเข้าที่ $(1, 1, 1)$ และมีทางออกที่ $(W, H, L)$

การเดินทางใน \textit{dungeon} นั้นจะสามารถเดินไปยังห้องที่ติดกันอยู่ได้ คือ ด้านบน, ด้านล่าง, ด้านขวา, ด้านซ้าย, ด้านหน้า, ด้านหลัง (สำหรับห้องที่อยู่ขอบเช่น $(1, 1, 1)$ จะเดินไปได้เพียงข้างขวา ข้างล่าง ข้างหลัง) เมื่อออกห้องใด ๆ แล้ว ห้องนั้นจะถูกปิดจากทุกทิศทุกทางทำให้เข้าไม่ได้อีกเลยไม่ว่าจะจากทิศใดก็ตาม ในแต่ละห้องจะมีแต้มต่าง ๆ อยู่ คุณอยากจะเก็บแต้มให้มากที่สุดเท่าที่จะเป็นไปได้ อย่างไรก็ตาม หากว่าเก็บแต้มได้มากที่สุดแต่ออกจาก \textit{dungeon} ไม่ได้ก็จะไม่มีประโยชน์แต่อย่างใด



\bigskip
\underline{\textbf{โจทย์}}  จงเขียนโปรแกรมหาแต้มรวมให้มากที่สุดเท่าที่จะเป็นไปได้ที่ทำให้เมื่อเดินทางแล้วสามารถออกจาก dungeon ได้ด้วย


\InputFile

\textbf{บรรทัดแรก} มีจำนวนเต็มบวก $W$ $H$ $L$ $(1 \leq W, H, L \leq 100)$

\textbf{ต่อมาจะมีข้อมูลอีก $L$ กลุ่ม แต่ละกลุ่มมีตาราง $H$ แถว แต่ละแถวมีข้อมูล $W$ ตัว}\\
โดยในกลุ่มที่ $i$ ในแถวที่ $j$ ของกลุ่มนั้น และข้อมูลลำดับที่ $k$ ของแถวนั้นจะเป็นแต้ม $V_{i,j,k}$ ในห้องที่มีพิกัด $(i,j,k)$ โดย $1 \leq V_{i,j,k} \leq 1\,000\,000$ 

ขอให้ดูตัวอย่างข้อมูลนำเข้าเพื่อความเข้าใจเพิ่มเติม

\OutputFile

\textbf{มีบรรทัดเดียว} มีจำนวนเต็มบอกถึงแต้มรวมมากสุดที่สามารถเก็บได้และสามารถออก \textit{dungeon} ได้ด้วย

\Examples

\begin{example}
\exmp{2 2 3
\ \ \ 
\\5 10
\\11 12
\ \ \ 
\\11 4
\\25 10
\ \ \ 
\\9 50
\\31 100}{268}%
\end{example}


\Source

สรวีย์ พรเจริญวาสน์

การแข่งขัน TUMSO ครั้งที่ 9

\end{problem}

\end{document}