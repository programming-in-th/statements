\documentclass[11pt,a4paper]{article}

\usepackage{../../templates/style}

\begin{document}

\begin{problem}{หม้อวิเศษ (pot)}{standard input}{standard output}{1 second}{32 megabytes}

งานเลี้ยงฉลองการแข่งขัน...ได้ถูกจัดขึ้นในคฤหาสน์หรูหรากลางป่าใหญ่ คุณได้รับมอบหมายหน้าที่ให้เป็นหัวหน้างานจัดเตรียมความเรียบร้อยในงานนี้ ตั้งแต่ทำความสะอาด ถูพื้น จัดโต๊ะ เตรียมอาหาร ยกของว่าง ต้อนรับแขก และอีกมากมาย

แม่ครัวเจ้าปัญหา ได้เสนออุปกรณ์ทำอาหารแสนวิเศษมาช่วยแบ่งเบาภาระของคุณ นั่นคือ หม้อหุงข้าววิเศษ เพียงแค่ใส่ข้าวสารลงไป มันก็สามารถหุงสารพัดเมนูข้าวออกมาให้คุณได้อย่างง่ายดาย (จำนวนเมนูที่หุงได้มีไม่จำกัด) ตั้งแต่ ข้าวสวย ข้าวกล้อง ข้าวเหนียว ข้าวหลาม ข้าวต้ม ข้าวผัด  หรือแม้กระทั่งข้าวหน้าไก่ย่าง 

      ในวันวันหนึ่งแม่ครัวสามารถหุงข้าวได้หลายครั้ง และในแต่ละครั้งก็จะได้ข้าวชนิดแตกต่างกันออกไป  แต่หม้อหุงข้าวเจ้าปัญหาจะมีลำดับการหุงที่แน่นอนอีกด้วย (เช่น หุงครั้งแรกของวันจะได้ข้าวสวย ครั้งที่สองได้ข้าวต้ม ครั้งที่สามได้ข้าวผัด เสมอ) ซึ่งในการหุงข้าวแต่ละครั้งอาจหุงข้าวได้หนึ่งจานหรือมากกว่าหนึ่งจานก็ได้

      แม่ครัวของคุณต้องหุงข้าว $N$ จานให้แขก $N$ คนที่จะมาเยี่ยมในวันนี้ หม้อหุงข้าววิเศษสามารถหุงข้าวได้จำนวนไม่เกิน $K$ จานต่อครั้ง เนื่องจากคุณเป็นคนที่ต้องวางแผนอะไรอย่างรอบคอบเสมอ คุณจึงอยากรู้ว่า คุณจะสามารถหุงข้าวให้แขกทั้งหมดออกมาได้กี่วิธี (ถ้าจำนวนวิธีมีมากกว่า $2\,008$ วิธี ให้ตอบเศษที่ได้จากการหารจำนวนวิธีด้วย $2\,009$)

      สมมติว่าคุณต้องหุงข้าวให้แขก $3$ คน โดยหม้อหุงข้าวของคุณหุงข้าวได้ไม่เกินครั้งละ $2$ จาน คุณจะหุงข้าวได้ $3$ วิธีดังนี้

\begin{itemize}
\item ข้าวชนิดแรก $2$ จาน ข้าวชนิดที่สอง $1$ จาน 
\item ข้าวชนิดแรก $1$ จาน ข้าวชนิดที่สอง $2$ จาน    
\item ข้าวชนิดแรก $1$ จาน ข้าวชนิดที่สอง $1$ จาน ข้าวชนิดที่สาม $1$ จาน
\end{itemize}

\bigskip
\underline{\textbf{โจทย์}}   เขียนโปรแกรมหาจำนวนวิธีการหุงข้าวทั้งหมดที่สามารถทำได้ โดยใช้หม้อหุงข้าววิเศษนี้

\InputFile

\textbf{มีบรรทัดเดียว} ประกอบไปด้วยจำนวนเต็ม $N$ $(1 \leq N \leq 100\,000)$ และ $K$ $(1 \leq K \leq 100\,000)$ แทนจำนวนแขกที่มา และจำนวนจานที่สามารถหุงได้มากที่สุดต่อการหุงข้าวหนึ่งครั้ง ตามลำดับ


\OutputFile

\textbf{มีบรรทัดเดียว} จำนวนวิธีที่เป็นไปได้ทั้งหมดในการหุงข้าวให้กับแขก (ถ้าเป็นไปได้มากกว่า $2\,008$ วิธี ให้ตอบเศษที่ได้จากการหารจำนวนวิธีด้วย $2\,009$)

\Examples

\begin{example}
\exmp{3 2}{3}%
\exmp{5 2}{8}%
\end{example}


\Source

ทักษพร กิตติอัครเสถียร

\underline{\href{http://www.thailandoi.org/toi.c/02-2009}{TOI.CPP:02-2009}}

\end{problem}

\end{document}