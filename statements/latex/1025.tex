\documentclass[11pt,a4paper]{article}

\usepackage{../../templates/style}

\begin{document}

\begin{problem}{Least Common Multiplier}{standard input}{standard output}{1 second}{64 megabytes}

\textbf{ตัวคูณร่วมน้อย – ค.ร.น. \textit{(least common multiplier – lcm)}} ของจำนวนเต็มสองจำนวนคือจำนวนเต็มที่มีค่าน้อยที่สุดที่สามารถหารได้ด้วยสองจำนวนนั้นๆ

วิธีการสามารถทำได้โดยการหาตัวประกอบจำนวนเฉพาะของจำนวนเต็มแต่ละตัว ถ้าหากมีค่าที่ซ้ำกันให้ใส่ค่าในบริเวณที่ซ้อนกันของแผนภาพเวนน์ จากนั้นนำตัวประกอบทั้งหมดมาคูณกัน ตัวอย่างเช่นค.ร.น.ของ $32$ และ $60$ เราแยกตัวประกอบของ $32$ และ $60$ ได้เป็น $2\times 2\times 2\times 2\times 2$ และ $2\times 2\times 3\times 5$ จะเห็นว่ามี $2$ ซ้ำกันสองตัวดังนั้นค.ร.น.จึงเท่ากับ $2\times 2\times 2\times 2\times 2\times 3\times 5 = 480$ อย่างไรก็ตามแนวคิดนี้สามารถนำมาขยายต่อเนื่องเพื่อหาค.ร.น.ของจำนวนเต็ม $n$ ค่า

\bigskip
\underline{\textbf{โจทย์}}  จงเขียนโปรแกรมหาตัวคูณร่วมน้อยของจำนวนเต็มบวกในเซ็ต $S$ ซึ่งมีสมาชิก $n$ ตัว

\InputFile

\textbf{บรรทัดแรก} รับค่าจำนวนเต็ม $n$ ค่าขนาดของเซ็ต $S$  $(2 \leq n \leq 50\,000)$

\textbf{บรรทัดที่ $2$ ถึง $n+1$} บรรทัดที่ $i+1$ รับสมาชิกลำดับที่ $i$ ของเซ็ต $S$ ในรูปจำนวนเต็ม $a_i$ โดยที่ค่าจะอยู่ในช่วง $1 \leq a_i \leq 100\,000$

\OutputFile

\textbf{มีบรรทัดเดียว} แสดงค่าตัวคูณร่วมน้อยของจำนวนเต็มในเซ็ต $S$ รับประกันว่าผลลัพธ์ของชุดทดสอบจะไม่เกิน $4,000,000,000$

\Examples

\begin{example}
\exmp{5
3
9
12
24
18}{72}%
\end{example}


\Source

การแข่งขันคณิตศาสตร์ วิทยาศาสตร์ โอลิมปิกแห่งประเทศไทย สาขาวิชาคอมพิวเตอร์ ประจำปี 2548


\end{problem}

\end{document}