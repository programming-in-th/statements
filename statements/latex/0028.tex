\documentclass[11pt,a4paper]{article}

\usepackage{../../templates/style}

\begin{document}

\begin{problem}{ฟุตบอลโลก (worldcup)}{standard input}{standard output}{1 second}{32 megabytes}

ในการแข่งขันฟุตบอลโลก 2010 ที่ประเทศแอฟริกาใต้ มีทีมที่เข้าร่วมแข่งขันทั้งหมด 32 ทีม การแข่งขันในรอบแรกจะแบ่งออกเป็น 8  กลุ่ม กลุ่มละ 4 ทีม ทีมในแต่ละกลุ่มจะทำการแข่งขันแบบพบกันหมด โดยในการแข่งขันแต่ละนัด ทีมที่ชนะจะได้ $3$ คะแนน ทีมที่แพ้จะไม่ได้คะแนน แต่ถ้าเสมอกันก็จะได้คะแนนทีมละ $1$ คะแนน หลังจากแข่งขันจบ ทีมที่ได้อันดับ 1 และอันดับ 2 ของแต่ละกลุ่ม จะได้ผ่านเข้าสู่รอบต่อไป ซึ่งการจัดอันดับทีมในแต่ละกลุ่มจะพิจารณาจากเกณฑ์ดังนี้:

\begin{enumerate}

\item ดูจากคะแนนที่ได้ ทีมใดได้คะแนนมากกว่าจะอยู่ในอันดับที่ดีกว่า
\item หากเกณฑ์ในข้อ 1 ยังเท่ากัน ให้ดูจากจำนวนประตูที่ยิงได้ลบด้วยจำนวนประตูที่ถูกยิง ทีมใดได้มากกว่าก็จะอยู่ในอันดับที่ดีกว่า
\item หากเกณฑ์ในข้อ 2 ยังเท่ากัน ดูจากจำนวนประตูที่ยิงได้ ทีมใดได้มากกว่าก็จะอยู่ในอันดับที่ดีกว่า
\end{enumerate}

หลังจากที่คุณได้ดูการแข่งขันทุกคู่จนจบ คุณก็เกิดความสงสัยว่าแต่ละทีมได้คะแนนเท่าไร และได้อันดับเท่าไรในกลุ่ม

\underline{\textbf{โจทย์}} จงเขียนโปรแกรมเพื่อรับผลการแข่งขันของทีมในกลุ่มหนึ่งแล้วคำนวณหาคะแนนและอันดับในกลุ่มของแต่ละทีม

\InputFile

\textbf{สี่บรรทัดแรก} ในบรรทัดที่ $i$ $(1 \leq i \leq  4)$ จะระบุชื่อทีมที่ $i$ ซึ่งเป็นสตริงที่ประกอบด้วยตัวอักษรภาษาอังกฤษตัวพิมพ์ใหญ่หรือพิมพ์เล็กเท่านั้น (ไม่มีเว้นวรรค) และจะมีความยาวไม่เกิน $20$ ตัวอักษร

\textbf{บรรทัดที่ห้าถึงแปด} จะระบุจำนวนเต็มบรรทัดละ $4$ ตัว โดยจำนวนเต็มตัวที่ $j$ $(1 \leq j \leq 4)$ ในบรรทัดที่ $i+4$ $(1 \leq i \leq 4)$ จะแสดงถึงจำนวนประตูที่ทีมที่ $i$ ยิงได้ ในนัดที่แข่งขันกับทีมที่ $j$ ยกเว้นในกรณีที่ $i = j$ จำนวนเต็มดังกล่าวจะมีค่าเป็นศูนย์เสมอ

การแข่งขันแต่ละนัดจะมีการยิงประตูเกิดขึ้นไม่เกิน $10$ ประตู และรับประกันว่าจะไม่มีสองทีมใดที่เท่ากันหมดในเกณฑ์การจัดอันดับทั้ง 3 ข้อ


\OutputFile

\textbf{มีสี่บรรทัด} โดยในบรรทัดที่ $i$ $(1 \leq i \leq 4)$ แสดงชื่อทีมที่ได้อันดับที่ $i$ ของกลุ่ม ตามด้วยคะแนนของทีมดังกล่าว

\Examples

\begin{example}
\exmp{Denmark
Netherlands
Cameroon
Japan
0 0 2 1
2 0 2 1
1 1 0 0
3 0 1 0}{Netherlands 9
Japan 6
Denmark 3
Cameroon 0}%
\exmp{Germany
Serbia
Australia
Ghana
0 0 4 1
1 0 1 0
0 2 0 1
0 1 1 0}{Germany 6
Ghana 4
Australia 4
Serbia 3}%
\end{example}

\Scoring

\textbf{40\% ของข้อมูลทดสอบ:} ไม่มีสองทีมใดที่ได้คะแนนเท่ากันเลย


\Source

โจทย์โดย: สุธี เรืองวิเศษ

การแข่งขัน IOI Thailand League เดือนกรกฏาคม 2553


\end{problem}

\end{document}