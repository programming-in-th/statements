\documentclass[11pt,a4paper]{article}

\usepackage{../../templates/style}

\begin{document}

\begin{problem}{3-11 (three-eleven)}{standard input}{standard output}{1 second}{32 megabytes}

ในระหว่างที่คุณกำลังเดินอยู่บนถนนสายหนึ่ง คุณเหลือบไปเห็นร้านสะดวกซื้อชื่อดัง 3-11 ตั้งอยู่ ด้วยความอยากรู้อยากเห็น คุณจึงเข้าไปในร้านแห่งนี้ และเลือกซื้อสินค้า หลังจากที่คุณเลือกซื้อสินค้าเสร็จสิ้นแล้ว คุณเหลือบไปเห็นป้ายโฆษณาที่กล่าวว่า คุณจะได้ส่วนลด 10\% ของราคาสินค้าทั้งหมด ถ้าคุณสามารถตอบพนักงานได้ว่า เศษจากการหารราคาสินค้าของคุณด้วย $3$ และ $11$ เป็นเท่าไร แน่นอนว่าคุณอยากได้ส่วนลด 10\% นี้ เพราะคุณต้องการจ่ายเงินให้น้อยที่สุด

\underline{\textbf{โจทย์}} จงเขียนโปรแกรมรับราคาสินค้าทั้งหมด แล้วระบุว่าเศษจากการหารราคาสินค้านั้นด้วย $3$ และ $11$ เป็นเท่าไร

\InputFile

\textbf{มีบรรทัดเดียว}  ระบุจำนวนเต็ม $N$ $(0 \leq N  <  10^{1\,000\,000})$ แทนราคาสินค้าทั้งหมด ราคาสินค้าเป็นจำนวนเต็มเสมอ

\textbf{ข้อควรระวัง:} ตัวเลขในข้อมูลนำเข้าอาจมีค่ามากเกินกว่าที่จะเก็บใน int หรือ long long ได้

\OutputFile

\textbf{มีบรรทัดเดียว} ระบุเศษจากการหาร $N$ ด้วย $3$ และเศษจากการหาร $N$ ด้วย $11$ ตามลำดับ คั่นด้วยช่องว่างหนึ่งช่อง

\Examples

\begin{example}
\exmp{11}{2 0}%
\exmp{40}{1 7}%
\end{example}


\Scoring 

\textbf{30\% ของชุดทดสอบ:} $N < 1\,000\,000\,000$

\Source

โจทย์โดย: ธงชัย วิโรจน์ศักดิ์เสรี

การแข่งขัน IOI Thailand League เดือนสิงหาคม 2553

\end{problem}

\end{document}