\documentclass[11pt,a4paper]{article}

\usepackage{../../templates/style}

\begin{document}

\begin{problem}{Herman}{standard input}{standard output}{1 second}{64 megabytes}

เรขาคณิตรถแท็กซี่ (Taxicab geometry) คือวิธีการวัดที่ได้จากแนวคิดของรถแท็กซี่ ที่จะต้องเดินทางจากจุดหนึ่ง ไปยังอีกจุดหนึ่ง ด้วยการเลี้ยวตามมุมตึกที่เป็นสี่แยก  ซึ่งทำให้ระยะระหว่างจุดสองจุด บนระนาบ $XY$ จากตำแหน่ง $T_1(x_1, y_1)$ มายังตำแหน่ง $T_2(x_2, y_2)$ ไม่เท่ากัน ทั้งนี้ถ้าวัดแบบรถแท็กซี่ระยะทางคิดได้จากจริง (reality show)

$$D_T(T_1, T_2) = |x_1 - x_2| + |y_1 - y_2|$$

ในขณะที่ระยะทางตามเรขาคณิตทั่วไป (Euclidean geometry) คือ

$$D_E(T_1, T_2) = \sqrt{(x_1 - x_2)^2 + (y_1 - y_2)^2}$$

ทั้งนี้เพื่อให้แท็กซี่แต่ละคัน แบ่งโซนในการให้บริการให้แน่ชัด จึงมีการกำหนด “พื้นที่ให้บริการ” ซึ่งนิยามโดย ระบุจุดกึ่งกลางหนึ่งตำแหน่ง และกำหนด “รัศมี” คือระยะที่ห่างออกไปจากจุดกึ่งกลางนั้น ถ้ากำหนดรัศมีมาให้ ต้องทราบค่าพื้นให้บริการ ตามเรขาคณิตทั่วไป และเรขาคณิตรถแท็กซี่ มีค่าเท่าใด

\underline{\textbf{โจทย์}} จงหาค่าพื้นที่ให้บริการของรถแท็กซี่แบบเรขาคณิตทั่วไป (Euclidean geometry) และแบบเรขาคณิตรถแท็กซี่

\InputFile

ค่าของรัศมี ($R$) เป็นข้อมูลนำเข้าในบรรทัดแรกที่เป็นค่าจำนวนเต็ม $R (0 \leq R \leq 10\,000)$

\OutputFile

\textbf{มีสองบรรทัด} บรรทัดแรกเป็นค่าจำนวนจริงของพื้นที่เรขาคณิตทั่วไป บรรทัดที่สองเป็นค่าจำนวนจริงของพื้นที่เรขาคณิตรถแท็กซี่

\Examples

\begin{example}
\exmp{1
}{3.141593
2.000000
}%
\exmp{21
}{1385.442360 
882.000000
}%
\exmp{42
}{5541.769441
3528.000000
}%
\end{example}

\Source

Croatian Open Competition in Informatics

Contest 1 - October 28, 2006

\end{problem}

\end{document}