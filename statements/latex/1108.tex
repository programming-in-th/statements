\documentclass[11pt,a4paper]{article}

\usepackage{../../templates/style}

\begin{document}

\begin{problem}{สายแร่แห่งพลัง 2 (ore2)}{standard input}{standard output}{0.5 second}{64 megabytes}

    เมืองเฟอรอนเป็นเมืองลอยฟ้า มันต้องใช้พลังงานสูงในการขับเคลื่อนตัวเองให้ลอยอยู่เหนือพื้นดิน แต่พลังงานที่ได้จากโรงไฟฟ้านิวเคลียร์กว่าร้อยแห่งบนเมืองนี้มันยังมีไม่ มากพอ เนื่องจากท่านคิงส์เฟอร์เคยเรียนอยู่ในโรงเรียนนักวิจัยที่มีชื่อเสียงแห่ง หนึ่ง ทำให้เขามีความสามารถในการวิเคราะห์และสกัดพลังงานออกมาจากแร่ชนิดต่างๆได้ เขาค้นพบแร่ที่ทรงพลังชนิดหนึ่ง มันมีชื่อว่า \textit{“เฟอเร่”} แร่ชนิดนี้จะอยู่เรียงชิดติดกันเป็นเส้นตรงความยาว $N$ ชิ้น โดยแต่ละชิ้นสามารถสกัดเป็นพลังงานออกมาได้ $E_i$ เฟอรูล (คล้ายกับหน่วยจูลในระบบเอสไอ) 
    
   ในการสกัดพลังงานจากแร่ชนิดนี้ เราไม่สามารถสกัดมันทีละชิ้นๆได้ เราจำเป็นต้องสกัดทุกๆชิ้นที่อยู่ติดกันพร้อมกันในคราวเดียว (ได้พลังงานเท่ากับผลรวมของพลังงานที่จะได้จากการสกัดแร่แต่ละชิ้น) แต่ทว่าหากเราสกัดแร่ที่มีพลังงานรวมกันมากกว่า $K$ เฟอรูลออกมาในครั้งเดียว ด้วยความทรงพลังของแร่เฟอเร่ มันอาจจะก่อให้เกิดการระเบิดที่น่าสะพรึงกลัวกว่าระเบิดนิวเคลียร์ก็เป็นได้ คิงส์เฟอร์จึงต้องทำลายแร่บางชิ้นทิ้งไปเพื่อให้สายแร่ความยาว $N$ นี้แยกออกเป็นสายแร่สั้นๆหลายๆสาย โดยแต่ละสายจะต้องมีผลรวมของพลังงานที่จะสกัดได้ไม่เกิน $K$ เฟอรูล (แร่ที่เลือกทำลายทิ้งเพื่อแบ่งสายแร่ออกเป็นสายย่อยๆ จะไม่สามารถนำมาสกัดเป็นพลังงานได้)

          เนื่องจากสายแร่เฟอเร่นั้นเป็นสายแร่หายาก เมื่อคิงส์เฟอร์หามันมาได้สายหนึ่ง เขาจึงต้องการที่จะใช้ประโยชน์จากมันให้ได้มากที่สุด 

          อย่างไรก็ตาม ในตอนแรกนั้นความยาวสายแร่มีได้อย่างมาก $2\,000$ แต่นั่นทำให้คิงพีเอสอิ๊นไม่พอใจเพราะโจทย์มันจะง่ายไป เขาจึงเพิ่มความยาวของสายแร่เป็น $1\,000\,000$

\bigskip
\underline{\textbf{โจทย์}}  จงเขียนโปรแกรมรับข้อมูลของสายแร่เฟอเร่ แล้วหาว่าพลังงานที่จะสามารถสกัดออกมาได้มากที่สุดเป็นเท่าไหร่


\InputFile

\textbf{บรรทัดแรก} ประกอบด้วยจำนวนเต็มสองจำนวน $N$ และ $K$ คั่นด้วยช่องว่าง $1$ ช่อง $(1 \leq N \leq 1\,000\, 000; 1 \leq K \leq 1\,000\,000\,000)$ แทนจำนวนแร่เฟอเร่ในสายแร่ที่คิงส์เฟอร์หามาได้ และพลังงานสูงสุดที่ไม่ทำให้เกิดการระเบิด 

\textbf{บรรทัดที่สอง} ประกอบด้วยจำนวนเต็ม $N$ ตัว แต่ละตัวถูกคั่นด้วยช่องว่าง $1$ ช่อง แสดงค่า $E_i$ สำหรับ $1 \leq i \leq N$ $(1 \leq E_i \leq 1\,000\,000)$ แทนพลังงานที่ได้จากการสกัดแร่ชิ้นที่ $i$ บนสายแร่ในหน่วยเฟอรูล



\OutputFile

\textbf{มีบรรทัดเดียว} โดยให้แสดงค่าพลังงานสูงสุดที่คิงส์เฟอร์จะได้จากการสกัดสายแร่เฟอเร่สายนี้ โดยไม่ก่อให้เกิดการระเบิด (ตอบในหน่วยเฟอรูล)


\Examples

\begin{example}
\exmp{5 1
1 2 3 4 5}{1}%
\exmp{5 3
1 2 3 1 2}{6}%
\exmp{6 1000
1 2 3 4 5 6}{21}%
\end{example}

\Scoring

\textbf{25\% ของชุดทดสอบทั้งหมด:} $n \leq 10$

\textbf{50\% ของชุดทดสอบทั้งหมด:} $n \leq 1\,000$

\textbf{100\% ของชุดทดสอบทั้งหมด:} $n \leq 1\,000\,000$

\Source

จิรายุ ชิว ชิว

FOI ( Fur Olympiad in Informatics ) และ ศูนย์ สอวน. โรงเรียนมหิดลวิทยานุสรณ์

\end{problem}

\end{document}