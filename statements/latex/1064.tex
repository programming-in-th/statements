\documentclass[11pt,a4paper]{article}

\usepackage{../../templates/style}

\begin{document}

\begin{problem}{Cromartie School}{standard input}{standard output}{1 second}{1 megabytes}

โรงเรียนคุโรมาตี้มีที่ดินเป็นรูปสี่เหลี่ยมผืนผ้า มีความกว้าง $W$ เมตร และยาว $L$ เมตร แต่ไม่เกินด้านละ $64$ เมตร สำหรับแต่ละตารางเมตรของที่ดินจะเป็นหนึ่งในรูปแบบต่อไปนี้

\begin{enumerate}
\item ที่ดินว่าง
\item ส่วนของแอ่งน้ำ 
\item ที่ดินที่มีต้นไม้ปลูกอยู่
\end{enumerate}

ตารางเมตรที่เป็นส่วนของแอ่งน้ำที่อยู่ติดกันในทิศเหนือ ใต้ ตะวันออก และตะวันตก จะถือว่าอยู่ในแอ่งน้ำเดียวกัน

ผู้อำนวยการโรงเรียนต้องการสร้างอาคารเรียน $1$ หลัง โดยมีเงื่อนไขว่า อาคารเรียนดังกล่าวจะต้องมีพื้นที่เป็นรูปสี่เหลี่ยมจัตุรัสที่มีพื้นที่มากที่สุด และจะต้องตั้งอยู่บนที่ดินว่าง แอ่งน้ำนั้นสามารถถมเป็นที่ดินว่างได้แต่ต้องถมทั้งแอ่ง แต่ต้นไม้มีประโยชน์ดังนั้นที่ดินที่มีต้นไม้ปลูกอยู่จึงจะต้องถูกคงไว้ดังเดิม

ในการเลือกบริเวณสร้างอาคาร อาจมีบริเวณที่มีพื้นที่มากที่สุดหลายบริเวณ  เพื่อความประหยัด ผู้อำนวยการต้องการบริเวณที่ต้องถมแอ่งน้ำเป็นจำนวนน้อยที่สุด   โดยผู้อำนวยการสนใจเฉพาะจำนวนแอ่งน้ำเท่านั้นแต่ไม่สนใจพื้นที่ของแอ่งน้ำที่ต้องถม

\bigskip
\underline{\textbf{โจทย์}}  จงเขียนโปรแกรมคำนวณพื้นที่ที่มากที่สุดของบริเวณสำหรับสร้างอาคารเรียน พร้อมทั้งระบุจำนวนแอ่งน้ำที่ต้องถม

\InputFile

\textbf{บรรทัดแรก} รับจำนวนเต็มบวกสองจำนวน คือ $W$ และ $L$ คั่นด้วยช่องว่าง โดย $W$ $(1 \leq W \leq 64)$ ระบุความกว้างของที่ดิน และ $L$ $(1 \leq L \leq 64)$ ระบุความยาวของที่ดิน

\textbf{บรรทัดที่ $2$ ถึง $L+1$} จะระบุข้อมูลของที่ดินในแต่ละตารางเมตร บรรทัดที่ $i+1$ จะระบุข้อมูลที่ดินแถวที่ $i$ ในแต่ละบรรทัดระบุตัวอักษรติดกัน $W$ ตัวแทนรูปแบบของพื้นที่แต่ละตารางเมตรของที่ดิน ตัวอักษรแต่ละตัวมีความหมายดังนี้ ตัวอักษร ‘S’ แทนที่ว่าง, ‘P’ แทนส่วนของแอ่งน้ำ และ ‘T’ แทนตารางเมตรที่มีต้นไม้ปลูกอยู่

\newpage
\OutputFile

\textbf{มีหนึ่งบรรทัด} ประกอบด้วยจำนวนเต็มสองจำนวน $a$ และ $b$ คั่นด้วยช่องว่าง โดย $a$ คือพื้นที่ที่มากที่สุดของบริเวณสำหรับสร้างอาคารเรียน และ $b$ คือจำนวนของแอ่งน้ำทั้งหมดที่ถม ในกรณีที่มีบริเวณที่มีพื้นที่มากที่สุดหลายบริเวณให้เลือกบริเวณที่ต้องถมแอ่งน้ำเป็นจำนวนน้อยที่สุด และในกรณีที่ไม่มีที่ว่างเหลือพอสร้างอาคารเรียนได้เลยทั้ง $a$ และ $b$ มีค่าเป็น $0$

\Examples

\begin{example}
\exmp{8 6
SSSSSSSS
SSSSSSSS
SPPSSSSS
SSSPSSSS
SSSSSTTS
PSSSSTSS}{25 2}%
\exmp{6 5
TSSSSS
TTSSSS
SSSPSS
SSPPPS
TSSPST}{16 1}%
\exmp{2 2
TT
TT}{0 0}%
\end{example}


\Source

การแข่งขันคอมพิวเตอร์โอลิมปิกสอวน.ครั้งที่ 4 ปี 2551 วันที่ 2

\end{problem}

\end{document}