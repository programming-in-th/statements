\documentclass[11pt,a4paper]{article}

\usepackage{../../templates/style}

\begin{document}

\begin{problem}{ระยะห่าง (distance)}{standard input}{standard output}{1 second}{32 megabytes}

\textbf{นิยาม:} \textit{ระยะห่างแมนฮัตตัน (Manhattan distance)} ระหว่างจุด $(x_1, y_1)$ และ $(x_2, y_2)$ บนระนาบมีค่าเท่ากับ $|x_1 - x_2| + |y_1 - y_2|$

มีจุด $N$ จุดอยู่บนระนาบ คุณต้องการทราบว่าผลรวมของระยะห่างแมนฮัตตันระหว่างจุดสองจุดใดๆ ทุกคู่ มีค่าเท่าไร


\bigskip
\underline{\textbf{โจทย์}}  จงเขียนโปรแกรมเพื่อรับพิกัดบนระนาบของจุดแต่ละจุด และคำนวณหาผลรวมของระยะห่างแมนฮัตตันระหว่างจุดสองจุดใด ๆทุกคู่



\InputFile

\textbf{บรรทัดแรก} ระบุจำนวนเต็ม $N$ $(2 \leq N \leq 500\,000)$ แทนจำนวนจุดทั้งหมด

\textbf{บรรทัดที่ $2$ ถึง $N+1$} ระบุพิกัดของจุดต่าง ๆ กล่าวคือ บรรทัดที่ $i+1$ $(1 \leq i \leq N)$ จะระบุจำนวนเต็ม $X_i$ และ $Y_i$ $(1 \leq X_i,Y_i \leq 1\,000\,000)$ แทนพิกัดตามแกน $X$ และแกน $Y$ ของจุดที่ $i$

\OutputFile

\textbf{มีบรรทัดเดียว} แสดงผลรวมของระยะห่างแมนฮัตตันระหว่างจุดสองจุดใดๆ ทุกคู่

\Examples

\begin{example}
\exmp{3
1 1
2 4
4 3}{12}%
\exmp{5
3 3
5 1
4 4
1 3
4 7}{44}%
\end{example}

\Scoring

\textbf{$30$\% ของข้อมูลทดสอบ:} $N \leq 1\,000$

\Source

สุธี เรืองวิเศษ

การแข่งขัน IOI Thailand League เดือนกรกฏาคม 2553

\end{problem}

\end{document}