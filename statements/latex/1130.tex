\documentclass[11pt,a4paper]{article}

\usepackage{../../templates/style}

\begin{document}

\begin{problem}{ความจุ (capacity)}{standard input}{standard output}{1 second}{32 megabytes}

ให้ $S$ เป็นเซตที่มีสมาชิก $N$ ตัว $(N \geq 2)$ เราจะนิยามความจุของเซต $S$ ดังต่อไปนี้
\begin{itemize}

\item ถ้า $N = 2$ แล้ว $S$ จะมีความจุเท่ากับค่าของสมาชิกตัวที่มากกว่าลบด้วยสมาชิกตัวที่น้อยกว่า
\item ถ้า $N > 2$ แล้ว $S$ จะมีความจุเท่ากับผลรวมของความจุของสับเซตของ $S$ ทุกสับเซตที่มีสมาชิก $N-1$ ตัว
\end{itemize}

\bigskip
\underline{\textbf{โจทย์}}  จงเขียนโปรแกรมเพื่อรับเซต $S$ และคำนวณหาความจุของเซตดังกล่าว


\InputFile

\textbf{บรรทัดแรก} ระบุจำนวนเต็ม $N$ $(2 \leq N \leq 10\,000)$ แทนจำนวนสมาชิกของเซต $S$

\textbf{บรรทัดที่สอง} ระบุจำนวนเต็ม $N$ จำนวน แต่ละจำนวนมีค่าอยู่ในช่วงตั้งแต่ $-1\,000\,000$ ถึง $1\,000\,000$ แทนสมาชิกแต่ละตัวของเซต $S$ โดยจำนวนเต็มเหล่านี้จะแตกต่างกันทั้งหมด และจะเรียงจากน้อยไปหามาก



\OutputFile

\textbf{มีบรรทัดเดียว} แสดงความจุของเซต $S$

หากคำตอบที่ได้มีค่ามากกว่าหรือเท่ากับ $49\,999$ ให้พิมพ์เศษจากการหารคำตอบที่ได้ด้วย $49\,999$

\Examples

\begin{example}
\exmp{3
-3 1 4}{14}%
\exmp{4
0 1 2 3}{20}%
\end{example}

\newpage
\Scoring

$30$\% ของข้อมูลทดสอบ: $N \leq 20$

$60$\% ของข้อมูลทดสอบ: $N \leq 1\,000$

\Source

สุธี เรืองวิเศษ

การแข่งขัน IOI Thailand League เดือนสิงหาคม 2553 

\end{problem}

\end{document}