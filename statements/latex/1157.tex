\documentclass[11pt,a4paper]{article}

\usepackage{../../templates/style}

\begin{document}

\begin{problem}{ส่งกระแสไฟฟ้า (electricity)}{standard input}{standard output}{1 second}{32 megabytes}

ในการส่งกระแสไฟฟ้าจากต้นทางไปถึงปลายทาง เมื่อไฟฟ้าเดินทางผ่านสายไฟ แรงดันไฟฟ้าจะลดลงไปเรื่อย ๆ ทำให้ต้องมีการตั้งสถานีเปลี่ยนแรงดันไฟฟ้าเพื่อเพิ่มแรงดันให้อยู่ในเกณฑ์ที่กำหนด แต่การเลือกตำแหน่งที่ตั้งสถานีเปลี่ยนแรงดันไฟฟ้าไม่ใช่เรื่องที่ง่ายนัก เพราะการไฟฟ้าต้องซื้อที่ดินสำหรับตั้งสถานีและราคาที่ดินแต่ละแปลงก็แตกต่างกันไป

กำหนดให้การไฟฟ้าจ่ายกระแสไฟฟ้าโดยเริ่มจากที่ดินแปลงหมายเลข $1$ และกระแสไฟถูกส่งผ่านต่อไปยังแปลงหมายเลข $2, 3, 4$ ไปเรื่อย ๆจนถึงปลายทางคือที่ดินแปลงหมายเลข $N$ โดยที่ดินเหล่านี้เรียงต่อกันเป็นเส้นตรงตามลำดับหมายเลขจากน้อยไปมาก ซึ่งในที่นี้หมายเลข $1$ คือที่ดินแปลงเริ่มต้น และหมายเลข $N$ คือที่ดินแปลงปลายทาง

นิยามระยะห่างระหว่างสถานีเปลี่ยนแรงดันไฟฟ้าสองแห่งที่อยู่บนที่ดินแปลงหมายเลข $a$ และ $b$ คือ $b-a$ โดยที่ $b > a$ กำหนดเพิ่มเติมว่าสถานีสองแห่งที่ส่งไฟฟ้าถึงกันโดยตรง (คือไม่มีสถานีอื่นมาคั่น) ต้องมีระยะห่างกันไม่เกิน $k$ แปลง นั่นคือ $b-a \leq k$ และหากการไฟฟ้าต้องการสร้างสถานีในที่ดินแปลงใดก็จะต้องซื้อที่ดินแปลงนั้น สำหรับราคาที่ดินของแปลงหมายเลข $1,2,3,..,n$ คือ $P_1,P_2,P_3,...,P_n$ ตามลำดับ

          

\bigskip
\underline{\textbf{โจทย์}}  จงเขียนโปรแกรมที่มีประสิทธิภาพในการหาค่าใช้จ่ายรวมที่น้อยที่สุดในการซื้อที่ดินเพื่อตั้งสถานีทั้งหมดสำหรับการส่งกระแสไฟฟ้าจากที่ดินแปลงหมายเลข $1$ ไปถึงแปลงหมายเลข $n$ เมื่อกำหนดให้การไฟฟ้าต้องตั้งสถานีในแปลงหมายเลข $1$ และหมายเลข $n$ เสมอ


\InputFile

\textbf{บรรทัดแรก} ระบุจำนวนแปลงที่ดิน $N$ ที่กระแสไฟจะถูกส่งผ่าน โดยที่ $2 \leq N \leq 500\,000$ 

\textbf{บรรทัดที่สอง} ระบุค่า $k$ แทนระยะห่างซึ่งเป็นจำนวนแปลงที่มากที่สุดระหว่างสถานีสองแห่งที่สามารถส่งไฟฟ้าถึงกันได้โดยตรง โดยที่ $1 \leq k \leq N$  และ $k \leq 20\,000$ 

\textbf{บรรทัดที่สาม} ประกอบด้วยเลขจำนวนเต็ม $N$ จำนวน คั่นด้วยช่องว่าง เลขเหล่านี้แทนราคาที่ดินของแต่ละแปลงคือ  ตามลำดับ $P_1,P_2,...,P_n$ โดยที่ $1 \leq P_i \leq 2\,000$


\OutputFile

\textbf{มีบรรทัดเดียว} แสดงค่าใช้จ่ายรวมที่น้อยที่สุดในการซื้อที่ดินเพื่อตั้งสถานีทั้งหมดสำหรับการส่งกระแสไฟฟ้าตามเงื่อนไข

\Examples

\begin{example}
\exmp{7
3
1 4 2 6 2 4 2}{7}%
\exmp{10
4
2 1 4 3 2 1 5 1 2 3}{7}%
\end{example}


\Source

การแข่งขันคอมพิวเตอร์โอลิมปิกระดับชาติครั้งที่ 8 (SUTOI8)

\end{problem}

\end{document}