\documentclass[11pt,a4paper]{article}

\usepackage{../../templates/style}

\begin{document}

\begin{problem}{บัญชาแห่งจำนวน (number)}{standard input}{standard output}{1 second}{8 megabytes}

    คุณและนักโบราณคดีหลบระเบิดมาได้อย่างเฉียดฉิว ทำให้เทพเจ้าแห่งตัวเลขตกใจเป็นอย่างมาก อย่างไรก็ดีก่อนที่เทพเจ้าจะยอมรับว่าคุณและนักโบราณคดีไม่ใช่ผู้ตั้งใจลบหลู่ เป็นเพียงแค่หมู่คนที่จริงๆ แล้วมีสัมมาคารวะ เพียงแต่ยังไม่รู้กาลเทศะดีเท่านั้น เทพเจ้าต้องการทดสอบคุณเป็นขั้นสุดท้าย

เทพเจ้ากำหนดให้มีลำดับจำนวนเต็ม $N$ จำนวน แทนด้วย $a_1,a_2,a_3,...,a_N$ โดยที่ $0 ≤ a_i ≤ 109$ และกำหนดให้มีปฏิบัติการกับลำดับของตัวเลขนี้ $4$ ประเภทดังนี้

\begin{itemize}

\item การเวียนวนตัวเลข (\textbf{a}) เมื่อกำหนดค่าจำนวนเต็มบวก $x$ และ $y$ มาให้ หน้าที่ของคุณคือการสลับค่าของ $a_x$ และ $a_y$ (ค่า $x$ และ $y$ อาจเท่ากันได้)
\item การจำแลงตัวเลข (\textbf{b}) เมื่อกำหนดจำนวนเต็มบวก $x$ และ $k$ หน้าที่ของคุณคือแทนค่า $a_x$ ใหม่ด้วยค่า $k$ ที่รับเข้ามา
\item การปัดเป่าตัวเลข (\textbf{c}) เมื่อกำหนดจำนวนเต็มบวก $x$ หน้าที่ของคุณคือแบ่งตัวเลขออกเป็นส่องกลุ่ม กลุ่มแรกคือตัวเลข $x$ ตัวแรก และกลุ่มที่สองคือตัวเลข $N - x$ ตัวที่เหลือ หลังจากนั้นให้เรียงตัวเลขทั้งสองกลุ่มจากหลังไปหน้าแล้วนำมาต่อกัน กล่าวคือ เปลี่ยนลำดับจาก  $a_1,a_2,a_3,...,a_N$  ให้เป็น $a_x,a_{x-1},a_{x-2},...,a_2,a_1,a_N,a_{N-1},a_{N-2},...,a_{x+2},a_{x+1}$
\item การออกดอกของตัวเลข (\textbf{q}) เมื่อกำหนดจำนวนเต็ม $x$ หน้าที่ของคุณบอกเทพเจ้าว่า $a_x$ มีค่าเท่าใด
\end{itemize}

\bigskip
\underline{\textbf{โจทย์}}  จงเขียนโปรแกรมรับลำดับตั้งต้นและรายการปฏิบัติการที่เทพเจ้าสั่งตามลำดับก่อนหลัง แล้วแสดงผลลัพธ์ตัวเลขของปฏิบัติการออกดอกของตัวเลขออกมาตามลำดับในข้อมูลเข้า

\InputFile

\textbf{บรรทัดแรก} มีจำนวนเต็ม $N$ และ $M$ $(1 \leq N, M \leq 500\,000)$ แสดงความยาวของลำดับตัวเลข และจำนวนปฏิบัติการตามลำดับ

\textbf{บรรทัดที่ $2$ ถึ ง $N+1$} มีข้อมูลของจำนวนเริ่มต้นในลำดับ โดยในบรรทัดที่ $i + 1$ ของข้อมูลนำเข้าจะมีจำนวนเต็มหนึ่งตัว แทนค่า $a_i$

\textbf{บรรทัดที่ $N+2$ ถึง $N+M+1$} มีข้อมูลของปฏิบัติการที่เทพเจ้าสั่งให้คุณทำ โดยแต่ละบรรทัดจะมีรูปแบบหนึ่งในสี่แบบดังต่อไปนี้

\begin{itemize}

\item “a $x$ $y$”  โดย $x$, $y$ คือจำนวนเต็มซึ่ง $1 \leq x, y \leq N$ หมายความว่าให้ทำการเวียนวนตัวเลขด้วยค่า $x$ และ $y$ ที่กำหนด
\item “b $x$ $k$”  โดย $x$ เป็นจำนวนเต็มซึ่ง $1 \leq x \leq N$ และ $k$ เป็นจำนวนเต็มซึ่ง $0 \leq k \leq 109$ หมายความว่าให้ทำการจำแลงตัวเลขด้วยค่า $x$ และ $k$ ที่กำหนด
\item “c $x$”  โดย $x$ เป็นจำนวนเต็มซึ่ง $1 \leq x \leq N$ หมายความว่าใหทำการปัดเป่าตัวเลขด้วยค่า $x$ ที่กำหนด
\item “q $x$” โดย $x$ เป็นจำนวนเต็มซึ่ง $1 \leq x \leq N$ หมายความว่าให้ทำการออกดอกตัวเลขโดยใช้ค่า $x$ ที่กำหนด
\end{itemize}

\OutputFile

\textbf{มี $D$ บรรทัด} เมื่อ $D$ คือ\textbf{จำนวนการออกดอกตัวเลข}ในข้อมูลเข้า โดยในบรรทัดที่ $i$ ให้พิมพ์คำตอบของการออกดอกตัวเลขครั้งที่ $i$ ตามลำดับก่อนหลังในข้อมูลเข้า

\Examples

\begin{example}
\exmp{5 6
1
3
4
5
2
q 3
b 3 6
a 2 4
q 2
c 1
q 4}{4
5
6}%
\end{example}

\Note 

\begin{tabular}{|l|l|l|}
\hline
\multicolumn{3}{|c|}{อธิบายตัวอย่างข้อมูลนำเข้า}\\
\hline
ข้อมูลนำเข้า&ข้อมูลส่งออก&คำอธิบาย\\
\hline
5 6&	 	&รับค่า $N = 5$,$M = 6$\\[1ex]
1	&& 	กำหนดค่า $a_1  = 1$\\[1ex]
3	&& 	กำหนดค่า $a_2  = 3$\\[1ex]
4	&& 	กำหนดค่า $a_3  = 4$\\[1ex]
5	&& 	กำหนดค่า $a_4  = 5$\\[1ex]
2	 &&	กำหนดค่า $a_5  = 2$\\[1ex]
q 3	&4	&แสดงค่า $a_3$\\[1ex]
b 3 6&	 &	ลำดับของจำนวนใหม่คือ $1$ $3$ $6$ $5$ $2$\\[1ex]
a 2 4&	 	&ลำดับของจำนวนใหม่คือ $1$ $5$ $6$ $3$ $2$\\[1ex]
q 2&	5&	แสดงค่า $a_2$\\[1ex]
c 1&&	 	ลำดับของจำนวนใหม่คือ $1$ $2$ $3$ $6$ $5$\\[1ex]
q 4&	6&	แสดงค่า $a_4$\\[1ex]
\hline
\end{tabular}

\Scoring

\textbf{50\% ของชุดข้อมูลทดสอบ:} $N \leq 5\,000 ;M \leq 5\,000$ 

\textbf{100\%ของชุดข้อมูลทดสอบ:} $N \leq 500\,000;M \leq 500\,000$


\Source

การแข่งขัน YTOPC Challenge เมษายน 2552


\end{problem}

\end{document}