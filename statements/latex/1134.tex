\documentclass[11pt,a4paper]{article}

\usepackage{../../templates/style}

\begin{document}

\begin{problem}{ลำดับ (sequence)}{standard input}{standard output}{0.3 second}{32 megabytes}

ให้ $A, B, C, D, E, F, G, H$ เป็นจำนวนเต็มใดๆ เราจะนิยามลำดับ $a_1, a_2, a_3, …$ ดังนี้

\begin{itemize}

\item $a_1 = A$, $a_2 = B$, $a_3 = C$, $a_4 = D$
\item $a_k = Ea_{k-1} + Fa_{k-2} + Ga_{k-3} + Ha_{k-4}$ สำหรับทุกจำนวนเต็ม $k \geq 5$
\end{itemize}

\bigskip
\underline{\textbf{โจทย์}}  จงเขียนโปรแกรมเพื่อตอบคำถามทั้งหมด $Q$ คำถามว่า สำหรับแต่ละค่า $N$ เศษจากการหาร $a_N$ ด้วย $2\,553$ มีค่าเท่าไร


\InputFile

\textbf{บรรทัดแรก} ระบุจำนวนเต็ม $A, B, C, D, E, F, G$ และ $H$ ซึ่งแต่ละจำนวนจะมีค่าอยู่ในช่วงตั้งแต่ $0$ ถึง $1\,000$

\textbf{บรรทัดที่สอง} ระบุจำนวนเต็ม $Q$ $(2 \leq Q \leq 200\,000)$ แทนจำนวนคำถามทั้งหมด

\textbf{บรรทัดที่ $3$ ถึง $Q+2$} ในบรรทัดที่ $i+2$ $(1 \leq i \leq Q)$ จะระบุจำนวนเต็ม $N$ $(1 \leq N \leq 10^{18})$ แสดงถึงคำถามที่ $i$


\OutputFile

\textbf{มี $Q$ บรรทัด} โดยในบรรทัดที่ $i$ $(1 \leq i \leq Q)$ แสดงคำตอบของคำถามที่ $i$

\Examples

\begin{example}
\exmp{1 2 3 4 2 3 0 0
7
1
2
3
4
5
6
7}{1
2
3
4
17
46
143}%
\exmp{5 2 1 8 4 9 3 6
8
10
20
15
1
5
4
12
15}{1127
2306
443
5
77
8
1598
443}%
\end{example}

\Scoring

\textbf{$20$\% ของข้อมูลทดสอบ:} $N \leq 1\,000\,000$ ในทุกคำถาม

\textbf{$50$\% ของข้อมูลทดสอบ:} $G = H = 0$

\textbf{$60$\% ของข้อมูลทดสอบ:} $Q \leq 2\,000$

\Source

สุธี เรืองวิเศษ

การแข่งขัน IOI Thailand League เดือนตุลาคม 2553

\end{problem}

\end{document}