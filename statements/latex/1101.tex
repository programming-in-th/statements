\documentclass[11pt,a4paper]{article}

\usepackage{../../templates/style}

\begin{document}

\begin{problem}{KRUSKA}{standard input}{standard output}{1 second}{64 megabytes}

\textit{Aladdin} รู้สึกเบื่อกับชีวิตในวัง เขามีการงานที่มั่นคง มีภรรยาชื่อ \textit{Jasmine} และคงกำลังจะมีลูกในเร็ว ๆ นี้ ชีวิตของเขากำลังจะน่าเบื่อหน่ายและเพื่อเตรียมตัวให้พร้อมกับสิ่งเหล่านี้ เขาจึงตัดสินใจที่จะเริ่มต้นการผจญภัยอีกครั้งก่อนที่จะลงหลักปักฐานที่นี่

เขาจึงตัดสินใจที่จะหาลูกแพร์ทองคำ วัตถุที่มีค่ามากมายมหาศาลในตำนานซึ่งยังไม่มีใครค้นพบเลยทะเลทรายที่ \textit{Aladdin} จะต้องไปค้นหา สามารถจำลองแบบเป็นกริดของเซลล์ขนาด $N \times N$ กริด แถวและคอลัมน์ต่าง ๆจะถูกนับจำนวนจาก $1$ ถึง $N$ ซึ่งนับจากบนลงล่างและจากซ้ายไปขวา และเซลล์บางส่วนในทะเลทรายจะมีพ่อมดที่ช่วยให้ \textit{Aladdin} เดาทางที่แตกต่างไปจากเดิมด้วย

Aladdin เริ่มต้นการเดาของเขาจากมุมบนซ้ายของทะเลทรายในวันจันทร์ โดยหันหน้าไปทางขวา การเคลื่อนที่ของเขาจะเป็นการทำขั้นตอนเหล่านี้ซ้ำ ๆ กัน ดังนี้
\begin{enumerate}

\item ถ้าเซลล์ปัจจุบัน มีพ่อมดที่ตื่นแล้ว ให้ \textit{Aladdin} หมุนตัวไปทางซ้ายหรือขวา $90$ องศาขึ้นกับสิ่งที่พ่อมดบอก
\item ถ้าการเคลื่อนที่ตรงไปข้างหน้าแล้วจะทำให้ \textit{Aladdin} ออกไปจากทะเลทราย ให้เขาหมุนตัวกลับ $180$ องศา
\item การเคลื่อนที่ไปข้างหน้า $1$ เซลล์ของ \textit{Aladdin} จะใช้เวลา $1$ วันเต็ม
\end{enumerate}

สำหรับพ่อมดแต่ละคน เราจะรู้ตำแหน่งที่อยู่และกำหนดการกิจกรรมของเขาในแต่ละวันของสัปดาห์ โดยกำหนดการจะเป็นสายอักขระของตัวอักษร ‘\textbf{L}’, ‘\textbf{R}’ หรือ ‘\textbf{S}’ ทั้งหมด $7$ ตัวอักษร แต่ละอักขระจะบอกเราในสิ่งที่พ่อมดกระทำในแต่ละวันของสัปดาห์นั้น ๆ โดยเริ่มต้นจากวันจันทร์ ตัวอักษร \textbf{L} หมายถึง \textit{Aladdin} จะถูกบอกให้หมุนตัวไปทางซ้าย ตัวอักษร \textbf{R} หมายถึง \textit{Aladdin} จะถูกบอกให้หมุนตัวไปทางขวา และตัวอักษร \textbf{S} หมายถึง พ่อมดนอนหลับตลอดทั้งวันนั้น

มีคำทำนายเก่าแก่ได้กล่าวไว้ว่า หลังจากการเปลี่ยนแปลงทิศทาง $K$ ครั้ง (ในขั้นตอนที่ 1 และ/หรือ 2) \textit{Aladdin} จะค้นพบลูกแพร์นั้น

\bigskip
\underline{\textbf{โจทย์}}  จงเขียนโปรแกรมเพื่อคำนวณหาจำนวนวันทั้งหมดที่การค้นหาจะสิ้นสุด ตามคำทำนายเก่าแก่



\InputFile

\textbf{บรรทัดแรก} ประกอบด้วยเลขจำนวนเต็มของขนาดของทะเลทราย ($N$) และจำนวนครั้งของการเปลี่ยนแปลงทิศทางในคำทำนายเก่าแก่ ($K$) ซึ่งจำนวนทั้งสองมีค่า ดังนี้ $2 \leq N \leq 200;1 \leq K \leq 1\,000\,000\,000$

\textbf{บรรทัดที่สอง} รับเลขจำนวนเต็มของจำนวนพ่อมด ($M$) ซึ่งมีค่า $0 \leq M \leq 10\,000$

\textbf{บรรทัดที่ $3$ ถึง $M+2$}แต่ละบรรทัดรับเลขจำนวนเต็มแสดงแถว ($R$) และคอลัมน์ ($C$) ของตำแหน่งที่พ่อมดอยู่ ซึ่งมีค่าดังนี้ $1 \leq R, C \leq N$ และตามด้วยสายอักขระแสดงกำหนดการกิจกรรมของพ่อมดซึ่งจะประกอบด้วยสายอักขระของตัวอักษร ‘L’, ‘R’ และ ‘S’ ทั้งหมด 7 ตัวอักษร

ไม่มีพ่อมดอยู่ในเซลล์เดียวกันถึง $2$ คนและไม่มีพ่อมดคนไหนอยู่ในเซลล์ ($1$, $1$)




\OutputFile

\textbf{มีบรรทัดเดียว} ให้แสดงผลระยะเวลาที่ใช้ในการค้นหา หน่วยเป็นวัน

\Examples

\begin{example}
\exmp{3 1
0}{2}%
\exmp{5 2
2
1 3 RRSRRRR
1 5 RRRRLRR}{4}%
\exmp{5 5
3
1 3 SSRSSSS
3 3 SSSLSSS
4 3 SSRSSLS}{10}%
\end{example}

\Note 

\textbf{ในตัวอย่างที่ 1} \textit{Aladdin} เคลื่อนที่ $2$ ครั้งจะพบขอบของทะเลทราย ดังนั้นเขาจึงหมุนตัวไป $180$ องศาและพบลูกแพร์

\textbf{ในตัวอย่างที่ 2} \textit{Aladdin} พบพ่อมดคนแรกในวันที่ $3$ แต่พ่อมดกำลังนอนหลับอยู่ ดังนั้น \textit{Aladdin} จึงยังคงเดินต่อไปในทิศทางเดิม หลังจากนั้นอีก $2$ วัน เขาพบพ่อมดอีกคนที่บอกให้เขาหมุนตัวไปทางซ้าย และ \textit{Aladdin} ก็ทำตามนั้นและพบขอบของทะเลทราย เขาจึงหันหลังกลับและพบลูกแพร์ในที่สุด

\Scoring

\textbf{ในกรณีทดสอบ จะได้รับคะแนนเป็นครึ่งหนึ่งของคะแนนทั้งหมด} ถ้า $K$ มีค่ามากที่สุดที่ $1\,000$
  
\Source

COCI 2008/2009, Contest \#5 – February 7, 2009

\end{problem}

\end{document}