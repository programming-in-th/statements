\documentclass[11pt,a4paper]{article}

\usepackage{../../templates/style}

\begin{document}

\begin{problem}{โดมิโน (domino)}{standard input}{standard output}{2 second}{64 megabytes}

มีโดมิโนจำนวน $N^2 $ ตัว ตั้งเรียงเป็นรูปสี่เหลี่ยมจัตุรัสขนาด $N\times N$ อยู่บนพื้น โดมิโนเหล่านี้เป็นโดมิโนพิเศษที่สามารถผลักให้ล้มได้ในทุกทิศทาง

คุณต้องการผลักโดมิโนทั้งหมด $M$ ครั้ง ในแต่ละครั้งจะผลักไปตามแนวของแถวหรือหลักของรูปสี่เหลี่ยมจัตุรัส โดยจะเริ่มผลักที่ตัวริมสุดของแถวหรือหลักนั้นเสมอ โดมิโนที่ถูกผลักจะล้มลง และหากมีโดมิโนตั้งอยู่ที่ตำแหน่งถัดไปในทิศทางที่ผลักก็จะทำให้โดมิโนตัวนั้นก็จะล้มลงด้วย และจะล้มลงต่อกันไปเรื่อย ๆจนกว่าจะสุดแถว หรือถ้ามีโดมิโนที่ล้มอยู่แล้วตั้งอยู่ระหว่างทางก็จะทำให้การล้มก็จะหยุดทันที

คุณต้องการทราบว่าในการผลักแต่ละครั้ง จะมีโดมิโนล้มลงทั้งหมดกี่ตัว

\bigskip
\underline{\textbf{โจทย์}}  จงเขียนโปรแกรมเพื่อรับขนาดของรูปสี่เหลี่ยมจัตุรัสและทิศทางในการผลักแต่ละครั้ง แล้วคำนวณหาจำนวนโดมิโนที่ล้มลงในการผลักแต่ละครั้ง


\InputFile

\textbf{บรรทัดแรก} ระบุจำนวนเต็ม $N$ และ $M$ $(1 \leq N \leq 1\,000\,000\,000; 1 \leq M \leq 100\,000)$ แทนขนาดของรูปสี่เหลี่ยมจัตุรัส และจำนวนครั้งในการผลัก

\textbf{บรรทัดที่ $2$ ถึง $N+1$} ในบรรทัดที่ $i+1$ $(1 \leq i \leq N)$ ระบุตัวอักษร N, S, W หรือ E แล้วตามด้วยจำนวนเต็ม $X_i$ แทนการผลักครั้งที่ $i$
\begin{itemize}

\item หากอักษรตัวแรกคือ N หมายความว่า ผลักโดมิโนตามแนวหลักที่ $X_i$ โดยเริ่มผลักที่ตัวในแถวบนสุด (ผลักในทิศลงมาด้านล่าง)
\item หากอักษรตัวแรกคือ S หมายความว่า ผลักโดมิโนตามแนวหลักที่ $X_i$ โดยเริ่มผลักที่ตัวในแถวล่างสุด (ผลักในทิศขึ้นไปด้านบน)
\item หากอักษรตัวแรกคือ W หมายความว่า ผลักโดมิโนตามแนวแถวที่ $X_i$ โดยเริ่มผลักตัวในแถวซ้ายสุด (ผลักในทิศไปทางขวา)
\item หากอักษรตัวแรกคือ E หมายความว่า ผลักโดมิโนตามแนวแถวที่ $X_i$ โดยเริ่มผลักที่ตัวในแถวขวาสุด (ผลักในทิศไปทางซ้าย)


\end{itemize}
\OutputFile

\textbf{มี $N$ บรรทัด} โดยในบรรทัดที่ $i$ $(1 \leq i \leq N)$ ระบุจำนวนโดมิโนที่ล้มลงในการผลักครั้งที่ $i$

\Examples

\begin{example}
\exmp{3 4
N 2
W 3
S 2
N 1}{3
1
0
2}%
\exmp{4 5
E 3
N 2
E 1
N 3
S 2}{4
2
2
0
1}%
\end{example}

\Scoring

$30$\% ของข้อมูลทดสอบ: $N \leq 1\,000$

$60$\% ของข้อมูลทดสอบ: $N \leq 100\,000$

\Source

สุธี เรืองวิเศษ

\end{problem}

\end{document}