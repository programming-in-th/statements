\documentclass[11pt,a4paper]{article}

\usepackage{../../templates/style}

\begin{document}

\begin{problem}{รถไฟใต้ดิน (subway)}{standard input}{standard output}{1 second}{32 megabytes}

พ.ศ. 2570 รัฐบาลได้ดำเนินโครงการก่อสร้างรถไฟใต้ดินซึ่งเป็นโครงการเมกะโปรเจกต์จนเสร็จสิ้น ทำให้กรุงเทพฯ กลายเป็นเมืองที่มีเครือข่ายรถไฟใต้ดินที่ใหญ่ที่สุดแห่งหนึ่งของโลก ประกอบด้วยเส้นทางรถไฟใต้ดินหลายร้อยสายและสถานีอีกนับล้านสถานี

คุณต้องการเดินทางโดยรถไฟใต้ดินจากสถานีหนึ่งไปยังอีกสถานีหนึ่ง โดยในระหว่างทาง สามารถทำการเปลี่ยนสายรถไฟได้โดยการไปลงที่บางสถานีแล้วขึ้นรถไฟใต้ดินสายอื่นที่ผ่านสถานีนั้นต่อ แต่การเปลี่ยนสายรถไฟแต่ละครั้งก็ทำให้เสียเวลาเป็นอย่างมาก คุณจึงต้องการเดินทางโดยเปลี่ยนสายรถไฟให้น้อยครั้งที่สุดเท่าที่จะทำได้

\bigskip
\underline{\textbf{โจทย์}}  จงเขียนโปรแกรมเพื่อตอบคำถามทั้งหมด $Q$ คำถามว่า การเดินทางจากสถานี $A_i$ ไปยังสถานี $B_i$ จะต้องทำการเปลี่ยนสายรถไฟอย่างน้อยกี่ครั้ง


\InputFile

\textbf{บรรทัดแรก} ระบุจำนวนเต็ม $N$ และ $M$ $(2 \leq N \leq 1\,000\,000; 1 \leq M \leq 500)$ แทนจำนวนสถานีทั้งหมดและจำนวนสายของรถไฟใต้ดิน

\textbf{บรรทัดที่ $2$ ถึง $M+1$ }ในบรรทัดที่ $i+1$ $(1 \leq i \leq M)$ ระบุจำนวนเต็มตัวแรกคือ $S_i$ $(2 \leq S_i \leq 2\,000)$ แทนจำนวนสถานีที่รถไฟใต้ดินสายที่ $i$ ผ่าน และจำนวนเต็มอีก $S_i$ จำนวนถัดมา ระบุหมายเลขของสถานีที่รถไฟใต้ดินสายดังกล่าวผ่าน เรียงตามลำดับจากปลายทางข้างหนึ่งไปจนถึงปลายทางอีกข้างหนึ่ง

\textbf{บรรทัดที่ $M+2$} ระบุจำนวนเต็ม $Q$ $(2 \leq Q \leq 1\,000\,000)$ แทนจำนวนคำถามทั้งหมด

\textbf{บรรทัดที่ $M+3$ ถึง $M+Q+2$ } ในบรรทัดที่ $M+i+2$ $(1 \leq i \leq Q)$ ระบุจำนวนเต็ม $A_i$ และ $B_i$ $(1 \leq A_i,B_i \leq N)$ แสดงถึงคำถามที่ $i$

สถานีแต่ละสถานีจะมีรถไฟใต้ดินผ่านไม่เกิน $20$ สาย โดยที่บางสถานีอาจไม่มีรถไฟใต้ดินผ่านเลยแม้แต่สายเดียวก็ได้ นอกจากนี้เส้นทางของรถไฟใต้ดินแต่ละสายอาจผ่านบางสถานีมากกว่าหนึ่งครั้งก็ได้


\OutputFile

\textbf{มี $Q$ บรรทัด} ในบรรทัดที่ $i$ $(1 \leq i \leq Q)$ ให้พิมพ์จำนวนครั้งของการเปลี่ยนสายรถไฟที่น้อยที่สุดที่ต้องใช้ในการเดินทางจากสถานี $A_i$ ไปยังสถานี $B_i$ แต่ถ้าไม่สามารถเดินทางโดยรถไฟใต้ดินจากสถานี $A_i$ ไปยังสถานี $B_i$ ได้ ให้พิมพ์คำว่า impossible

\Examples

\begin{example}
\exmp{6 2
3 1 2 3
3 2 4 5
3
1 3
1 4
2 6}{0
1
impossible}%
\exmp{15 5
6 1 2 3 4 2 5
2 6 7
4 1 6 8 9
4 10 11 12 13
3 14 11 15
6
9 2
10 13
10 5
3 7
6 14
15 12}{1
0
impossible
2
impossible
1}%
\end{example}

\Scoring

\textbf{$50$\% ของข้อมูลทดสอบ:} $N \leq 1\,000; M \leq 100; Q \leq 1\,000$ และ $S_i \leq 20$ สำหรับทุกจำนวนเต็ม $i$ $(1 \leq i \leq M)$
  
\Source

สุธี เรืองวิเศษ

การแข่งขัน IOI Thailand League เดือนกันยายน 2553


\end{problem}

\end{document}