\documentclass[11pt,a4paper]{article}

\usepackage{../../templates/style}

\begin{document}

\begin{problem}{Bond}{standard input}{standard output}{1 second}{32 megabytes}

ทุกคนคงจะรู้จักสายลับศูนย์ศูนย์เจ็ด เจมส์ บอนด์ ผู้โด่งดัง แต่ก็ยังไม่มีคนทราบว่าความจริงแล้ว เขาไม่ได้ปฏิบัติภารกิจส่วนใหญ่ด้วยตัวเขาเอง แต่เป็นลูกพี่ลูกน้องของเขา จิมมี่ บอนด์ ต่างหาก ส่วนเจมส์บอนด์จะเป็นคนกำหนดลำดับภารกิจสำหรับจิมมี่ทุกครั้งที่มีภารกิจใหม่เข้ามา ดังนั้นเขาจึงต้องการให้คุณช่วย

ทุกเดือนจะมีรายการของภารกิจเข้ามา ด้วยความอัจฉริยะและประสบการณ์ของเจมส์ เขาสามารถคาดคะเนความน่าจะเป็นที่จิมมี่จะปฏิบัติภารกิจนั้นๆ สำเร็จได้ เมื่อเขาลงมือปฏิบัติภารกิจนั้นเป็นลำดับที่ต่างๆกันไป (ภารกิจหนึ่งอาจจะมีความน่าจะเป็นไม่เท่ากัน เมื่อเลือกทำเป็นลำดับแรก หรือ ลำดับที่สอง หรือ ลำดับที่สาม …)

\bigskip
\underline{\textbf{โจทย์}}  จงเขียนโปรแกรมที่รับจำนวนภารกิจและความน่าจะเป็นของความสำเร็จของภารกิจต่างๆ และหาว่าความน่าจะเป็น\textbf{สูงสุด}ที่จิมมี่จะปฏิบัติภารกิจทุกภารกิจสำเร็จเป็นเท่าใด โดยที่ความน่าจะเป็นที่จะปฏิบัติภารกิจ\textbf{ทุกภารกิจ}สำเร็จคือผลคูณของความน่าจะเป็นของทุกภารกิจที่ปฏิบัติ

\InputFile

\textbf{บรรทัดแรก} รับค่าจำนวนเต็ม $N$ $(1 \leq N \leq 20)$ คือจำนวนภารกิตที่ได้รับมอบหมาย

\textbf{บรรทัดที่ $2$ ถึง $N+1$} แต่ละบรรทัดประกอบไปด้วยจำนวนเต็ม $N$, $a_1$ $a_2$ $a_3$ $...$ $a_N$ โดยแต่ละจำนวนคือความน่าจะเป็นที่จะปฏิบัติภารกิจ โดยในบรรทัดที่ $i+1$ ตัวเลข $a_j$ คือความน่าจะเป็นของภารกิจที่ $j$ เมื่อเลือกทำเป็นลำดับที่ $i$ โดยค่าเหล่านี้จะเป็นร้อยละ $(0 \leq a_j \leq 100)$

\OutputFile

\textbf{มีบรรทัดเดียว} แสดงความน่าจะเป็นที่สูงที่สุดของการปฏิบัติภารกิจของจิมมี่ ด้วยร้อยละเป็นจำนวนทศนิยม โดยผลลัพธ์จะต้องต่างกับคำตอบไม่เกิน $0.000001$ (หนึ่งในล้านส่วน) จึงจะถือว่าถูกต้อง

\Examples

\begin{example}
\exmp{2
100 100
50 50}{50.000000}%
\exmp{2
0 50
50 0}{25.00000}%
\exmp{3
25 60 100
13 0 50
12 70 90}{9.10000}%
\end{example}

\Note 

ถ้าจิมมี่เลือกทำภารกิจตามลำดับเป็น ภารกิจที่ $3$ ภารกิจที่ $1$ และภารกิจที่ $2$ ตามลำดับ

จะได้ความน่าจะเป็นเท่ากับ $1.0 * 0.13 * 0.7 = 0.091 = 9.1\%$

ส่วนรูปแบบลำดับของภารกิจอื่นนั้นจะได้ความน่าจะเป็นที่น้อยกว่าเสมอ

\Source

COCI 2006/2007, Contest \#1 – October 28, 2008

\end{problem}

\end{document}