\documentclass[11pt,a4paper]{article}

\usepackage{../../templates/style}

\begin{document}

\begin{problem}{เกมนับเลข (counting)}{standard input}{standard output}{1 second}{32 megabytes}

ณ โรงเรียนอนุบาลแห่งหนึ่ง มีนักเรียนอยู่ทั้งหมด $N$ คน แต่ละคนจะมีหมายเลขประจำตัวตั้งแต่ $1, 2, 3$ เรียงไปเรื่อยๆ จนถึง $N$ วันหนึ่ง คุณครูให้นักเรียนทั้งหมดมานั่งล้อมวงเพื่อเล่นเกมนับเลข นักเรียนจะนั่งล้อมวงโดยเรียงตามลำดับหมายเลขคือ $1, 2, 3$ เรียงไปเรื่อย ๆจนถึง $N$ โดยนักเรียนหมายเลข $N$ จะนั่งติดกับนักเรียนหมายเลข $1$ 

เกมนับเลขเริ่มจากนักเรียนหมายเลข $1$ จะนับ $1$ จากนั้นนักเรียนคนถัดไปคือหมายเลข $2$ ก็จะนับ $2$ จากนั้นนักเรียนคนถัดไปอีกคือหมายเลข $3$ ก็จะนับ $3$ เป็นเช่นนี้ไปเรื่อยๆ จนกระทั่งนับถึง $K$ นักเรียนคนที่นับ $K$ จะต้องเดินออกจากห้องเรียนไป จากนั้นนักเรียนคนถัดไปก็จะเริ่มนับจาก $1$ ใหม่ แล้วนักเรียนคนถัดไปอีกก็จะนับ $2, 3, 4$ เรียงไปเรื่อยๆ จนถึงคนที่นับ $K$ ก็จะต้องเดินออกจากห้องเรียน เกมจะดำเนินเช่นนี้ไปเรื่อยๆ จนในที่สุดจะเหลือนักเรียนอยู่ในห้องเรียนเพียงคนเดียว นักเรียนคนนั้นก็จะเดินออกจากห้องเรียนเป็นคนสุดท้าย

\bigskip
\underline{\textbf{โจทย์}}  จงเขียนโปรแกรมเพื่อรับจำนวนเต็ม $N$ และ $K$ แล้วคำนวณหมายเลขของนักเรียนแต่ละคนที่เดินออกจากห้องเรียน เรียงไปตามลำดับเวลา


\InputFile

\textbf{มีบรรทัดเดียว} ระบุจำนวนเต็ม $N$ และ $K$ $(3 \leq N \leq 5\,000; 1 \leq K \leq 1\,000\,000)$


\OutputFile

\textbf{มี $N$ บรรทัด} แต่ละบรรทัดแสดงหมายเลขประจำตัวของนักเรียนที่เดินออกจากห้องเรียน โดยเรียงไปตามลำดับเวลา

\Examples

\begin{example}
\exmp{5 3}{3
1
5
2
4}%
\exmp{8 15}{7
8
3
2
6
5
1
4}%
\end{example}

\Scoring

\textbf{$30$\% ของข้อมูลทดสอบ:} $K \leq 1\,000$
  
\Source

สุธี เรืองวิเศษ

การแข่งขัน IOI Thailand League เดือนสิงหาคม 2553

\end{problem}

\end{document}