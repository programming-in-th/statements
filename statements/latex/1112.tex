\documentclass[11pt,a4paper]{article}

\usepackage{../../templates/style}

\begin{document}

\begin{problem}{ลำดับสลับสับสน (Inversion)}{standard input}{standard output}{1 second}{32 megabytes}

    ในลำดับของตัวเลข $n$ ตัว ( มีค่าตั้งแต่ $1$ ถึง $n$ ไม่ซ้ำกัน ) รูปแบบหนึ่งๆ เราจะกำหนดค่าความสับสนของลำดับคือ จำนวนของคู่อันดับ      $( i , j )$ ที่ $i < j$ แต่ตำแหน่งของเลข $i$ นั้นอยู่ข้างหลัง $j$ กล่าวคือเป็นคู่ของตัวเลขที่เลขมากกว่าอยู่ข้างหน้าเลขที่น้อยกว่า

                        ตัวอย่างเช่น          ลำดับ $4$ $1$ $5$ $3$ $2$ มีค่าความสับสนเป็น $6$ คือ $(4,1)$ $(4,3)$ $(4,2)$ $(5,3)$ $(5,2)$ และ $(3,2)$
                                                ลำดับ $2$ $4$ $1$ $5$ $3$ มีค่าความสับสนเป็น $4$ คือ $(2,1)$ $(4,1)$ $(4,3)$ และ $(5,3)$

            

\bigskip
\underline{\textbf{โจทย์}}  กำหนดค่า $n$ และ $k$ จงหาจำนวนของรูปแบบการเรียงสับเปลี่ยนเลข $1$ ถึง $n$ เพื่อให้มีค่าความสับสนของลำดับเป็น $k$


\InputFile

\textbf{มีบรรทัดเดียว} ประกอบด้วยจำนวนนับ $n$ และ $k$ $( 1 \leq n , k \leq 10\,000 )$


\OutputFile

\textbf{มีบรรทัดเดียว} แสดงค่าจำนวนของรูปแบบการเรียงสับเปลี่ยนเลข $1$ ถึง $n$ เพื่อให้มีค่าความสับสนของลำดับเป็น $k$ โดยหากคำตอบมีค่ามากกว่า $2\,012$ ให้แสดงค่าเศษที่ได้จากการหารคำตอบด้วย $2\,012$ ( นั่นก็คือการ mod ด้วย $2\,012$)


\Examples

\begin{example}
\exmp{9 2}{35}%
\exmp{6 4}{49}%
\end{example}

\Scoring

 \textbf{$30$\% ของชุดทดสอบทั้งหมด:} $n, k \leq 10$
 
            \textbf{$70$\% ของชุดทดสอบทั้งหมด:} $n, k \leq 1\,000$
            
           \textbf{ $100$\% ของชุดทดสอบทั้งหมด:} $n, k \leq 10\,000$
            
\Source

สรวิทย์  สุริยกาญจน์ ( PS.int )

ศูนย์ สอวน. โรงเรียนมหิดลวิทยานุสรณ์

\end{problem}

\end{document}