\documentclass[11pt,a4paper]{article}

\usepackage{../../templates/style}

\begin{document}

\begin{problem}{Ptice}{standard input}{standard output}{1 second}{32 megabytes}

Adrian, Bruno และ Goran ต้องการที่จะเข้าร่วมคลับคนรักนก แต่ทว่า พวกเขาไม่รู้ว่าคนที่จะเข้าคลับนั้นต้องผ่านการทดสอบซะก่อน
การสอบมีทั้งสิ้น $N$ คำถาม แต่ละคำถามมี $3$ ตัวเลือกคือ A, B และ C
แต่โชคไม่ดีที่พวกเขาไม่สามารถทำกันได้สักเท่าไหร่ พวกเขาจึงพยายามเดาตัวเลือกที่ถูกต้องแทน

\textbf แต่ละคนนั้นจะมีวิธีในการเดาที่แตกต่างกันดังนี้
\begin{itemize}
  \item Adrian จะทำข้อสอบด้วยรูปแบบ A, B, C, A, B, C, A, B, C, A, B, C, ...
  \item Bruno อ้างว่าที่วิธีที่ดีกว่าน่าจะเป็น B, A, B, C, B, A, B, C, B, A, B, C, ...
  \item Goran ก็หัวเราะใส่เพื่อนและจะตอบโดยใช้ C, C, A, A, B, B, C, C, A, A, B, B, ...
\end{itemize}

\underline{\textbf{โจทย์}} จงเขียนโปรแกรมที่รับเฉลยของข้อสอบ และตอบว่าวิธีของใครเป็นวิธีที่ดีที่สุด (กล่าวอีกนัยหนึ่งคือ เป็นวิธีที่ทำให้ได้คะแนนสอบมากที่สุด)

\InputFile

\textbf{บรรทัดแรก} จำนวนเต็ม $N$ โดยที่ $1 \leq N \leq 100$ คือจำนวนคำถามในข้อสอบ

\textbf{บรรทัดที่สอง} เป็นข้อความสายอักขระความยาว $N$ ที่ประกอบด้วยตัวอักษร 'A', 'B' และ 'C' เป็นเฉลยของคำถามในแต่ละข้อเรียงตามลำดับ

\OutputFile

\textbf{บรรทัดแรก} แสดงจำนวนเต็ม $M$ แสดงคะแนนที่มากที่สุดจากการใช้วิธีเดาของทั้งสามคน

\textbf{บรรทัดที่สอง} คือชื่อของคนที่ได้คะแนนสูงสุด ถ้ามีหลายคนให้ตอบชื่อเรียงตามลำดับพจนานุกรม บรรทัดละหนึ่งชื่อ

\Source

COCI 2008/2009, Contest 1 – October 18, 2008

\underline{\textbf{หมายเหตุ:}} ที่ต้องใช้ชื่อภาษาอังกฤษในคำอธิบายเพราะคำตอบจะได้สอดคล้องกับโจทย์

\Examples

\begin{example}
\exmp{6
BBAACC}{3
Adrian}%
\exmp{9
AAAABBBBB}{4
Adrian
Bruno
Goran}%
\end{example}

\end{problem}

\end{document}
