\documentclass[11pt,a4paper]{article}

\usepackage{../../templates/style}

\begin{document}

\begin{problem}{ไดอารี่ (diary)}{standard input}{standard output}{0.2 second}{16 megabytes}

ในที่สุด คุณก็ได้มันมา! สิ่งที่คุณตามหามานาน... ไดอารี่ ของเพื่อนของคุณ

      แต่เพื่อนของคุณรู้แล้ว ว่าไดอารี่ของเขากำลังตกอยู่ในอันตราย เขากำลังวิ่งมาหาคุณด้วยความเร็วหนึ่งหน่วยต่อวินาที คุณยืนห่างจากเพื่อน $K$ หน่วย ดังนั้น คุณมีเวลาอ่านมันเพียง $K$ หน่วยเท่านั้น

      การพลิกคร่าวๆ ทำให้คุณรู้ว่า เพื่อนของคุณได้เขียนไดอารี่ทั้งหมด $N$ วัน จำนวนหน้าที่เพื่อนของคุณเขียนในแต่ละวันอาจไม่เท่ากัน สมองของคุณบอกคุณทันทีว่า สำหรับแต่ละวัน เพื่อนของคุณเขียนลงไปกี่หน้า โดยการเขียนของวันเดียวกันจะต่อเนื่องกัน และไม่มีหน้าไหนเว้นว่าง

      คุณใช้เวลาอ่านไดอารี่หนึ่งหน้าต่อวินาที คุณกำลังรีบ ดังนั้น คุณอยากอ่านไดอารี่ $K$ หน้านี้ ให้มาจากหลายวันมากที่สุดเท่าที่จะเป็นไปได้

      สมมติว่าคุณเริ่มอ่านที่หน้าที่ $M$ เนื่องจากมือของคุณพลิกกระดาษได้ทีละ $A$ หน้า คุณตัดสินใจว่าคุณจะอ่านหน้าที่ $M,M+A,M+2A,… ,M+(K-1)A$  เป็นจำนวน $K$ หน้าพอดี ถามว่า คุณควรเริ่มอ่านที่หน้าใด จึงจะทำให้ได้อ่านไดอารี่จากหลายวันมากที่สุด และการอ่านจากหน้านั้น ทำให้คุณได้อ่านบันทึกจากกี่วัน กำหนดว่าคุณต้องใช้เวลาให้ครบ K วินาทีในการอ่านไดอารี่

\bigskip
\underline{\textbf{โจทย์}}  เขียนโปรแกรมรับรายละเอียดการเขียนไดอารี่ของเพื่อนของคุณ จำนวนกระดาษที่คุณพลิกในแต่ละครั้ง และตอบว่าคุณจะเริ่มอ่านที่หน้าใด

\InputFile

\textbf{ บรรทัดแรก} มีจำนวนเต็มสามตัว คือ $N, K$ และ $A$ $(N \leq 100; K \leq 100; A \leq 100)$

\textbf{บรรทัดที่ $2$ ถึง $N+1$} มีจำนวนเต็มบรรทัดละหนึ่งตัว แต่ละตัวมีค่าไม่เกิน $100$ บอกว่า ไดอารี่ของแต่ละวัน บันทึกไว้กี่หน้า

รับประกันว่าในทุกชุดข้อมูลทดสอบ จะสามารถอ่านได้อย่างน้อย $K$ หน้าเสมอ เมื่อเริ่มอ่านจากหน้าแรก

\OutputFile

\textbf{มีบรรทัดเดียว} มีจำนวนเต็มสองจำนวนบอกว่า คุณเริ่มอ่านที่หน้าใด และคุณได้อ่านบันทึกจากกี่วัน    ถ้ามีหลายคำตอบที่อ่านได้จำนวนวันเท่ากัน ให้พิมพ์คำตอบที่เริ่มอ่านหน้าที่เลขน้อยที่สุดที่ได้จำนวนวันมากที่สุด

\Examples

\begin{example}
\exmp{4 4 2
12
3
1
5}{12 4}%
\end{example}

\Note 

\textbf{อธิบายข้อมูลน้ำเข้าและส่งออก}

      หน้าที่ $1-12$ เป็นของวันแรก หน้าที่ $13-15$ เป็นของวันที่สอง หน้าที่ $16$ เป็นของวันที่สาม และหน้าที่ $17-21$ เป็นของวันที่สี่ การเริ่มอ่านจากหน้าที่ $12$ จะทำให้คุณได้อ่านหน้า $12, 14, 16$ และ $18$ ซึ่งทำให้คุณได้อ่านไดอารี่จากครบทั้งสี่วัน

\Source

ทักษพร กิตติอัครเสถียร

\underline{\href{http://www.thailandoi.org/toi.c/04-2009}{TOI.C:04-2009}}

\end{problem}

\end{document}