\documentclass[11pt,a4paper]{article}

\usepackage{../../templates/style}

\begin{document}

\begin{problem}{ลำดับไอโอโอไอ (IOOI\_sequence)}{standard input}{standard output}{1 second}{32 megabytes}

\textbf{นิยาม:} \textit{ลำดับไอโอโอไอ (IOOI sequence)} คือลำดับของจำนวนเต็มบวกที่ทุกจำนวนหารด้วย $1\,001$ ลงตัว ซึ่งลำดับจะมีความยาวเท่าไรก็ได้

\textbf{ตัวอย่างเช่น} $(1\,001$, $5\,005$, $8\,008)$ และ $(123\,123$, $20\,020$, $97\,097$, $20\,020)$ เป็น\textit{ลำดับไอโอโอไอ} แต่ $(3\,003$, $987$, $1\,001)$ และ $(5\,000$, $333\,333$, $1\,234$, $75)$ ไม่เป็น\textit{ลำดับไอโอโอไอ}

\bigskip
\underline{\textbf{โจทย์}}  จงเขียนแกรมเพื่อรับค่า $N$ และ $K$ แล้วคำนวณว่ามี\textit{ลำดับไอโอโอไอ}อยู่ทั้งหมดกี่ลำดับ ที่แต่ละจำนวนในลำดับเป็นตัวประกอบของ $1\,001^K$ และผลคูณของทุกจำนวนในลำดับมีค่าเท่ากับ $1\,001^N$


\InputFile

\textbf{มีบรรทัดเดียว} ซึ่งระบุจำนวนเต็ม $N$ และ $K$ $(1 \leq K \leq N \leq 5\,000)$


\OutputFile

\textbf{มีบรรทัดเดียว} แสดงจำนวนของลำดับไอโอโอไอทั้งหมดที่สอดคล้องกับเงื่อนไข

หากคำตอบที่ได้มีค่ามากกว่าหรือเท่ากับ $2\,553$ ให้พิมพ์เศษจากการหารคำตอบที่ได้ด้วย $2\,553$

\Examples

\begin{example}
\exmp{4 2}{29}%
\exmp{5 4}{345}%
\end{example}

\Scoring 

\textbf{$15$\% ของข้อมูลทดสอบ:} $N \leq 100$

\textbf{$50$\% ของข้อมูลทดสอบ:} $N \leq 2\,500$

\Source

สุธี เรืองวิเศษ
 
การแข่งขัน IOI Thailand League เดือนกรกฏาคม 2553

\end{problem}

\end{document}