\documentclass[11pt,a4paper]{article}

\usepackage{../../templates/style}

\begin{document}

\begin{problem}{ลุ้นตอบปัญหา (riddle)}{standard input}{standard output}{0.3 second}{32 megabytes}

 ด้วยการลงทุนอย่างชาญฉลาด คุณก็สามารถหางบประมาณได้พอที่จะจ่ายค่าเครื่องบิน ค่าอาหาร ค่าที่พัก และค่าของฝาก ในออกเดินทางสำรวจโบราณสถานได้ หลังจากคุณใช้เงินไปเกือบทั้งหมดในการเลือกซื้อของฝากอยู่นาน ในที่สุด คุณก็ยืนอยู่หน้าโบราณสถานแห่งหนึ่ง... คุณสังเกตเห็นอักขระโบราณที่สลักอยู่บนประตูทางเข้า

\begin{center}
        \textit{“เทพธิดาตัวเลขทรงโปรดจำนวนเต็มมาก ท่านไม่อยากให้คนที่ไม่รักจำนวนเต็มเข้ามาในโบราณสถานแห่งนี้ ข้างล่างหินก้อนนี้จะมีตัวเลขอยู่สามจำนวน คือ $x$, $y$ และ $k$ สำหรับทุก $x \leq a \leq y ให้หาผลบวกของจำนวนเลข $ $0$ ที่ต่อท้าย $a!$ แล้วตอบเศษจากการเอาผลบวกนั้นมาหารด้วย $k$ อย่าลืมลุ้นตอนตอบคำถามท่านเทพธิดานะจ๊ะ \^{} - \^{}”}
\end{center}

        คุณรู้สึกว่าคุณได้เห็นข้อความทำนองนี้เป็นครั้งที่สามแล้ว แต่นั่นไม่สำคัญ ที่สำคัญคือ คุณต้องเข้าไปในโบราณสถานให้ได้ต่างหาก คุณรู้ว่า $n!$ มีค่าเท่ากับ $n \times (n-1) \times (n-2) \times … \times 1$ ดังนั้นคุณคิดว่า คุณต้องแก้ไขปริศนานี้ได้อย่างแน่นอน คุณยกหินก้อนนั้นขึ้นมาและพบตัวเลขสามตัว...


\bigskip
\underline{\textbf{โจทย์}}   จงเขียนโปรแกรมรับค่า $x, y$ และ $k$ จากนั้น แสดงผลบวกจำนวน $0$ ที่ต่อท้าย $a!$ สำหรับทุกจำนวนเต็ม $a$ ที่ $x \leq a \leq y$ 

\InputFile

\textbf{มีบรรทัดเดียว} ประกอบไปด้วยจำนวนเต็มบวกสามตัว $x, y$ และ $k$ โดย $(y-x) \leq 5\,000\,000; y \leq 2\,000\,000\,000;$ $x \geq 1; 1 \leq k \leq 1\,000\,000\,000$

\OutputFile

\textbf{มีบรรทัดเดียว} แสดงเศษจากผลลัพท์การหารผลบวกของจำนวน $0$ ที่ต่อท้าย $a!$ สำหรับทุกจำนวนเต็ม $a$ ที่ $x \leq a \leq y$ ด้วย $k$

\Examples

\begin{example}
\exmp{1 5 2}{1}%
\end{example}


\Source

ภัทร สุขประเสริฐ

\underline{\href{http://www.thailandoi.org/toi.c/03-2009}{TOI.CPP:03-2009}}

\end{problem}

\end{document}