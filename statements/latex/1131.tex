\documentclass[11pt,a4paper]{article}

\usepackage{../../templates/style}

\begin{document}

\begin{problem}{เป่ายิ้งฉุบ (PRS)}{standard input}{standard output}{1 second}{32 megabytes}

คุณกำลังเล่นเกมเป่ายิ้งฉุบออนไลน์กับเพื่อนของคุณทางอินเตอร์เน็ต โดยคุณได้ป้อนข้อมูลการเล่นในอีก $N$ ตาข้างหน้าใส่ไว้ในโปรแกรม แล้วให้โปรแกรมทำการเล่นตามข้อมูลที่ป้อนไว้ให้โดยอัตโนมัติ คำสั่งในการป้อนข้อมูลจะอยู่ในรูปสตริงความยาว $N$ ที่ประกอบด้วยตัวอักษร \textbf{P}, \textbf{R} และ \textbf{S} ซึ่งแทน\textit{กระดาษ (paper)}, \textit{ก้อนหิน (rock)}, และ\textit{กรรไกร (scissors)} ตามลำดับ

อย่างไรก็ตาม หากคุณออก\textit{กระดาษ ก้อนหิน} หรือ\textit{กรรไกร} อย่างใดอย่างหนึ่งติดต่อกันมากจนเกินไป จะทำให้เพื่อนของคุณสามารถจับทางในการเล่นของคุณได้ คุณจึงกำหนดเงื่อนไขว่า ในสตริงจะต้องไม่มีอักษรเหมือนกัน $3$ ตัวเรียงติดต่อกันโดยเด็ดขาด ซึ่งเราจะเรียกสตริงที่สอดคล้องกับเงื่อนไขดังกล่าวว่า \textit{"สตริงสมดุล"}

คุณต้องการทราบว่า เมื่อนำสตริงสมดุลที่มีความยาว $N$ ทั้งหมด มาเรียงลำดับตามพจนานุกรม (\textbf{P} มาก่อน \textbf{R} และ \textbf{R} มาก่อน \textbf{S}) สตริงที่คุณป้อนไว้ในโปรแกรม จะอยู่ในลำดับที่เท่าไร


\bigskip
\underline{\textbf{โจทย์}}  จงเขียนโปรแกรมเพื่อรับสตริงสมดุลสตริงหนึ่ง และคำนวณหาลำดับตามพจนานุกรมของสตริงดังกล่าว


\InputFile

\textbf{บรรทัดแรก} ระบุจำนวนเต็ม $N$ $(1 \leq N \leq 1\,000\,000)$ แทนความยาวของสตริง

\textbf{บรรทัดที่สอง} มีสตริงสมดุลความยาว $N$ ที่ประกอบด้วยตัวอักษร P, R หรือ S ตัวพิมพ์ใหญ่เท่านั้น



\OutputFile

\textbf{มีบรรทัดเดียว} แสดงลำดับตามพจนานุกรมของสตริงในข้อมูลนำเข้า

หากคำตอบที่ได้มีค่ามากกว่าหรือเท่ากับ $2\,553$ ให้พิมพ์เศษจากการหารคำตอบที่ได้ด้วย $2\,553$

\Examples

\begin{example}
\exmp{3
RPR}{10}%
\exmp{5
PPSSR}{16}%
\end{example}

\Scoring

\textbf{$20$\% ของข้อมูลทดสอบ:} $N \leq 10$

\Source

สุธี เรืองวิเศษ

การแข่งขัน IOI Thailand League เดือนสิงหาคม 2553

\end{problem}

\end{document}