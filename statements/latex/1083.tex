\documentclass[11pt,a4paper]{article}

\usepackage{../../templates/style}

\begin{document}

\begin{problem}{ลงทุนซื้อหุ้น (stock)}{standard input}{standard output}{1 second}{32 megabytes}

    เรื่องเริ่มต้นเมื่อวันหนึ่งคุณได้ข่าวมา ว่า มีอัญมณีอันล้ำค่ายิ่งกว่าที่เคยมีใครพบเจอถูกซ่อนอยู่... ได้เวลาปลุกความเป็นนักล่าสมบัติของคุณแล้ว! แต่กลับมีอุปสรรคสำคัญมาขวางคุณซะได้... เพราะการสำรวจทางโบราณคดีใดๆนั้นต้องใช้งบประมาณอยู่มาก คุณจึงตัดสินใจที่จะเอาเงินที่คุณมีไปลงทุนในตลาดหุ้น

        เนื่องจากราคาหุ้นของบริษัทต่างๆ ขึ้นอยู่กับข่าวสารที่จะมาชี้ชะตาอนาคตของบริษัท และบางครั้งคุณก็ไม่ได้รู้ข่าวพวกนี้พร้อมกับคนอื่น ทำให้คุณอาจเสียเปรียบคนอื่นได้ แต่ด้วยสมองอันปราดเปรื่องของคุณ คุณได้สร้างเครื่องทำนายราคาหุ้นล่วงหน้ามา

สมมติว่า เครื่องให้ข้อมูลราคาหุ้นในวันต่างๆ เป็นดังนี้

\begin{itemize}

		 \item วันที่ $1$ ราคา $10$
         \item วันที่ $2$ ราคา $20$
         \item วันที่ $3$ ราคา $15$
         \item วันที่ $4$ ราคา $12$
         \item วันที่ $5$ ราคา $21$
         \item วันที่ $6$ ราคา $30$

\end{itemize}


        ในช่วงวันที่ $1$ - $6$ คุณจะสามารถทำกำไรได้ $28$ บาท โดยซื้อวันที่ $1$ ขายวันที่ $2$ และ ซื้อวันที่ $4$ ขายวันที่ $6$ (เริ่มต้นคุณมีเงินไม่จำกัด)

         คุณมีข้อมูลราคาหุ้นอยู่ $n$ วัน และคุณต้องการตอบคำถาม $q$ คำถาม โดยแต่ละคำถามจะถาม
ว่า ในช่วงการลงทุนตั้งแต่วันที่ $a_j$ ถึงวันที่ $b_j$ ที่กำหนดให้ คุณจะสามารถทำการซื้อและขายหุ้นในช่วง
เฉพาะในช่วงวันดังกล่าว (หรือก็คือช่วง $[a_j, b_j]$) ให้ได้กำไรสูงสุดเท่าไร

        เนื่องจากคุณไม่ต้องการให้เครื่องทำนายราคาหุ้นของคุณเป็นที่จับตามองของนักลงทุนคนอื่น ๆ
คุณจึงถือหุ้นในมือ ณ ขณะใด ๆ ไม่เกินหนึ่งหน่วยเท่านั้น (การถือหุ้นต้องถือเป็นจำนวนเต็มหน่วย
เท่านั้น)

\bigskip
\underline{\textbf{โจทย์}}  จงเขียนโปรแกรมรับราคาหุ้นที่เครื่องของคุณทำนายออกมา และช่วงการลงทุนที่คุณต้องการทราบกำไร จากนั้นให้แสดงผลกำไรสูงสุดที่คุณสามารถทำได้สำหรับแต่ละช่วงที่กำหนด

\InputFile

\textbf{บรรทัดที่หนึ่ง}  รับจำนวนเต็มบวก $n$ $(1 \leq n \leq 1\,000\,000)$ แทนจำนวนวันที่คุณมีข้อมูลราคาหุ้น
  
\textbf{บรรทัดที่สอง} รับจำนวนเต็มบวกอยู่ $n$ ตัว ตัวที่ $i$ แทนราคาของหุ้นในวันที่ $i$ โดยราคาหุ้นในแต่ละวันจะมีค่าไม่เกิน $7\,000$
        
\textbf{บรรทัดที่สาม} รับจำนวนเต็มบวก $q$ $(1 \leq q \leq 1\,000\,000)$ แทนจำนวนช่วงการลงทุนที่คุณต้องการทราบกำไร

\textbf{บรรทัดที่ $4$ ถึง $q+3$} แต่ละบรรทัดรับข้อมูลของคำถามที่ $i$ ในบรรทัดที่ $i+3$ โดยรับจำนวนเต็มบวก $2$ ตัว $a$ $b$ $(1 \leq a, b \leq n)$ แทนวันเริ่มต้นและวันสิ้นสุดของแต่ละช่วงการลงทุน


\OutputFile

\textbf{มี $q$ บรรทัด} บรรทัดที่ $i$ มีจำนวนเต็มบวกหนึ่งตัว แทนกำไรที่มากที่สุดที่คุณสามารถทำได้ในช่วงที่ $i$

\Examples

\begin{example}
\exmp{6
10 20 15 12 21 30
3
1 6
2 4
3 5}{28
0
9}%
\end{example}


\Source

วรภัทร บุญญฤทธิพงษ์

\underline{\href{http://www.thailandoi.org/toi.c/03-2009}{TOI.CPP:03-2009}}

\end{problem}

\end{document}