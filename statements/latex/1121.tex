\documentclass[11pt,a4paper]{article}

\usepackage{../../templates/style}

\begin{document}

\begin{problem}{สามวงเล็บ (threeparen)}{standard input}{standard output}{1 second}{32 megabytes}

พิจารณาสตริงที่ประกอบด้วยเครื่องหมายวงเล็บ $3$ ชนิด คือวงเล็บกลม "( )" วงเล็บเหลี่ยม "[ ]" และวงเล็บปีกกา "\{\}" เราจะเรียกสตริงหนึ่งว่าเป็น "สตริงวงเล็บสมดุล" เมื่อวงเล็บในสตริงนั้นสามารถจับคู่กันได้อย่างถูกต้อง ซึ่งเราสามารถนิยามสตริงวงเล็บสมดุลอย่างเป็นทางการได้ดังนี้

\begin{enumerate}
\item "()", "[]" และ "\{\}" เป็นสตริงวงเล็บสมดุล
\item ถ้า "$A$" เป็นสตริงวงเล็บสมดุล แล้ว "($A$)", "[$A$]" และ "\{$A$\}" ก็เป็นสตริงวงเล็บสมดุลเช่นกัน
\item ถ้า "$A$" และ "$B$" เป็นสตริงวงเล็บสมดุล แล้ว "$AB$" ก็เป็นสตริงวงเล็บสมดุลเช่นกัน
\end{enumerate}

สังเกตว่า เราจะสามารถสร้างสตริงวงเล็บสมดุลความยาวต่างๆ ได้โดยใช้กฎสามข้อข้างบนนี้ เช่น เราสามารถสร้าง "[()\{\}]" โดยเริ่มจากใช้กฎข้อที่ 1 สร้าง "()" และ "\{\}" แล้วใช้กฎข้อที่ 3 สร้าง "()\{\}" แล้วจึงใช้กฎข้อที่ 2 สร้าง "[()\{\}]"

\bigskip
\underline{\textbf{โจทย์}}  จงเขียนโปรแกรมเพื่อตอบคำถามทั้งหมด $Q$ คำถามว่า สตริงที่ให้มาเป็นสตริงวงเล็บสมดุลหรือไม่


\InputFile

\textbf{บรรทัดแรก} ระบุจำนวนเต็ม $Q$ $(2 \leq Q \leq 10)$ แทนจำนวนคำถามทั้งหมด

\textbf{บรรทัดที่ $2$ ถึง $Q+1$} ในบรรทัดที่ $i+1$ $(1 \leq i \leq Q)$ จะมีสตริงในคำถามที่ $i$ ซึ่งแต่ละสตริงจะประกอบไปด้วยเครื่องหมายวงเล็บกลม วงเล็บเหลี่ยม หรือวงเล็บปีกกาเท่านั้น และแต่ละสตริงจะมีความยาวไม่เกิน $100\,000$


\OutputFile

\textbf{มี $Q$ บรรทัด} โดยในบรรทัดที่ $i$ $(1 \leq i \leq Q)$ ให้พิมพ์ yes ถ้าสตริงในคำถามที่ $i$ เป็นสตริงวงเล็บสมดุล และพิมพ์ no ถ้าสตริงในคำถามที่ $i$ ไม่เป็นสตริงวงเล็บสมดุล

\Examples

\begin{example}
\exmp{3
(())
(()))(()
(()())()}{yes
no
yes}%
\exmp{3
(\{])[]
[(\{\}])
()[\{\}()]}{no
no
yes}%
\end{example}

\Scoring 

\textbf{$30$\% ของข้อมูลทดสอบ:} สตริงในคำถามทุกสตริงจะประกอบด้วยวงเล็บเพียงชนิดเดียวเท่านั้น คือวงเล็บกลม

\textbf{$50$\% ของข้อมูลทดสอบ:} สตริงในคำถามทุกสตริงจะมีความยาวไม่เกิน $100$

\textbf{$15$\% ของข้อมูลทดสอบ:} จะสอดคล้องกับเงื่อนไขด้านบนทั้งสองข้อ 

\Source

สุธี เรืองวิเศษ

การแข่งขัน IOI Thailand League เดือนมิถุนายน 2553


\end{problem}

\end{document}