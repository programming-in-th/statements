\documentclass[11pt,a4paper]{article}

\usepackage{../../templates/style}

\begin{document}

\begin{problem}{TRESNJA}{standard input}{standard output}{1 second}{64 megabytes}

Lana อาศัยอยู่ในหมู่บ้านเล็ก ๆซึ่งเป็นหมู่บ้านที่ครึกครื้นที่หมู่บ้านแห่งนี้มีต้นเชอรี่เรียงกันเป็นแถวตลอดทางของถนนสายหลัก Lana นับหมายเลขของต้นไม้เหล่านี้ด้วยตัวเลขที่ต่อเนื่องกันตามลําดับโดยเริ่มต้นจาก $1$ และหลังจากที่ได้ศึกษามามาก Lana พบว่าหมายเลขของต้นไม้สามารถนํามาใช้ในการพิจารณาหาจํานวนของลูกเชอรี่ทั้งหมดที่ต้นไม้แต่ละต้นจะให้ผลได้

สําหรับต้นไม้ $1$ ต้นให้พิจารณาที่กลุ่มที่ต่อเนื่องกันของตัวเลข ($0$ ถึง $9$) ในหมายเลขของต้นไม้ และในแต่ละกลุ่มของตัวเลขให้คูณตัวเลขนั้นด้วยค่าความยาวของกลุ่มยกกําลังสอง แล้วบวกตัวเลขเหล่านั้นในทุกกลุ่มก็จะได้ผลรวมของลูกเชอรี่ทั้งหมดที่ต้นไม้ต้นนั้นให้ผลได้

ยกตัวอย่างเช่นถ้าหมายเลขของต้นไม้คือ $77744007$ สามารถแบ่งออกเป็น $4$ กลุ่มคือ $777$, $44$, $00$ และ $7$ จะได้จํานวนของลูกเชอรี่ทั้งหมดคือ $(7 \times 3)  + (4 \times 2)  + (0 \times 2)  + (7 \times 1)$  หรือ $86$ ลูก

ตอนนี้ถึงเวลาที่จะเก็บผลจากต้นเชอรี่แล้วและคนในหมู่บ้านมีความเห็นตรงกันว่าจะเก็บผลจากต้นไม้ทุกต้นซึ่งเริ่มต้นจากต้นหมายเลขที่ $A$ ถึงต้นหมายเลขที่ $B$ (รวมต้น $A$ และ $B$ ด้วย)

\bigskip
\underline{\textbf{โจทย์}}  จงเขียนโปรแกรมเพื่อคํานวณหาจํานวนผลทั้งหมดของลูกเชอรี่ที่จะสามารถเก็บได้

\InputFile

\textbf{มีบรรทัดเดียว} รับค่าเลขจํานวนเต็ม $2$ ค่าคือต้นเชอรี่ต้นแรกที่ถูกเก็บผล ($A$) และต้นเชอรี่ต้นสุดท้ายที่ถูกเก็บผล ($B$) ซึ่งมีค่าดังนี้ $1 \leq A \leq B ≤ 10^{15}$

\OutputFile

\textbf{มีบรรทัดเดียว} ให้แสดงผลเลขจํานวนเต็มเพียงค่าเดียวคือจํานวนผลทั้งหมดของลูกเชอรี่ที่จะสามารถเก็บได้

\Examples

\begin{example}
\exmp{1 9}{45}%
\exmp{100 1}{68}%
\exmp{7774407 77744}{86}%
\end{example}


\Source

COCI 2008/2009, Contest \#5 – February 7, 2009


\end{problem}

\end{document}