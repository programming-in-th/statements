\documentclass[11pt,a4paper]{article}

\usepackage{../../templates/style}

\begin{document}

\begin{problem}{วุ้น (jelly)}{standard input}{standard output}{1 second}{32 megabytes}

คมกฤษณ์เปิดตู้เย็น พบวุ้นกะทิค้างคืนส่งกลิ่นคละคลุ้งไปทั่ว ขณะที่เขากำลังจะหยิบวุ้นก้อนนั้นไปทิ้ง เขาก็ระลึกได้ว่าวุ้นก้อนนั้นเหมาะที่จะใช้ทดสอบความคมของมีดที่เพิ่งซื้อมาจากรายการแนะนำสินค้าทางโทรทัศน์ยามค่ำคืน

คมกฤษณ์ต้องการตัดวุ้นซึ่งเป็นทรงสี่เหลี่ยมมุมฉากขนาด $A\times B\times C$ นิ้ว โดยแต่ละครั้งที่เขาตัดวุ้น เขาจะเลือกตัด\textbf{โดยพยายามแบ่งด้านที่ยาวที่สุดของวุ้นออกเป็นสองส่วนเท่าๆ กัน} แต่เนื่องจากเขาต้องการให้ขนาดของวุ้นเป็นจำนวนเต็มนิ้วเสมอ หากด้านที่ยาวที่สุดนั้นมีความยาวเป็นจำนวนคี่นิ้ว เขาจะฝานวุ้นทิ้ง $1$ นิ้วก่อนจะแบ่งด้านนั้นออกเป็น $2$ ส่วนเท่าๆ กัน เมื่อตัดแบ่งแล้ว คมกฤษณ์จะโยนวุ้นส่วนใดส่วนหนึ่งทิ้งไป ก่อนจะพยายามตัดวุ้นส่วนที่เหลืออีกครั้งด้วยวิธีการเดิม

สังเกตว่าคมกฤษณ์จะตัดวุ้นไปได้เรื่อยๆ จนกว่าจะเหลือวุ้นขนาด $1\times 1\times 1$ นิ้ว (หากคมกฤษณ์ยังรั้นตัดวุ้นต่อไปอีก วุ้นจะเละ ส่งกลิ่นคละคลุ้งรุนแรงกว่าเดิม)

\underline{\textbf{โจทย์}} รับข้อมูลขนาดของวุ้น แล้วคำนวณว่าคมกฤษณ์จะตัดวุ้นตามขั้นตอนได้กี่ครั้ง (ไม่นับการฝานวุ้น)

\InputFile

\textbf{มีบรรทัดเดียว}  ระบุจำนวนเต็ม $A, B$ และ $C$ $(1 \leq A,B,C \leq 1\,000\,000)$ คั่นด้วยช่องว่างหนึ่งช่อง

\OutputFile

\textbf{มีบรรทัดเดียว} ระบุว่าคมกฤษณ์จะตัดวุ้นตามขั้นตอนได้กี่ครั้ง

\Examples

\begin{example}
\exmp{2 5 4}{5}%
\exmp{17 13 11}{10}%
\end{example}
\newpage
\Note 

\underline{\textbf{อธิบายตัวอย่างข้อ}}\textbf{มู}\underline{\textbf{ลที่ 1}}

\textbf{ครั้งที่ 1: }วุ้นขนาด $2\times 5\times 4$ นิ้ว จะถูกตัดเป็นวุ้นขนาด $2\times 2\times 4$ นิ้ว เพราะด้านที่ยาวที่สุดคือด้านที่ยาว $5$ นิ้ว เนื่องจากเป็นจำนวนคี่จึงถูกฝานออก $1$ นิ้ว เหลือ $4$ นิ้ว ก่อนจะถูกตัดแบ่งเป็นส่วนละ $2$ นิ้ว

\textbf{ครั้งที่ 2:} วุ้นขนาด $2\times 2\times 4$ นิ้ว จะถูกตัดเป็นวุ้นขนาด $2\times 2\times 2$ นิ้ว เพราะด้านที่ยาวที่สุดคือด้านที่ยาว $4$ นิ้ว เนื่องจากเป็นจำนวนคู่จึงตัดแบ่งเป็นส่วนละ $2$ นิ้ว

\textbf{ครั้งที่ 3: }วุ้นขนาด $2\times 2\times 2$ นิ้ว จะถูกตัดเป็นวุ้นขนาด $1\times 2\times 2$ นิ้ว

\textbf{ครั้งที่ 4:} วุ้นขนาด $1\times 2\times 2$ นิ้ว จะถูกตัดเป็นวุ้นขนาด $1\times 1\times 2$ นิ้ว

\textbf{ครั้งที่ 5:} วุ้นขนาด $1\times 1\times 2$ นิ้ว จะถูกตัดเป็นวุ้นขนาด $1\times 1\times 1$ นิ้ว

\Source

โจทย์โดย: ธนะ วัฒนวารุณ

การแข่งขัน IOI Thailand League เดือนสิงหาคม 2553

\end{problem}

\end{document}