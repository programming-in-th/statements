\documentclass[11pt,a4paper]{article}

\usepackage{../../templates/style}

\begin{document}

\begin{problem}{นักสู้ตัวเลข (fighter)}{standard input}{standard output}{1 second}{16 megabytes}

นักสู้ฝ่ายเลขคู่กับฝ่ายเลขคี่ทำการประลองฝีมือกันแบบตัวต่อตัว โดยในตอนเริ่มประลอง นักสู้ทั้งสองมี \textit{‘พลังงาน’} เริ่มต้นอยู่ฝ่ายละ $P$ หน่วย การโจมตีของแต่ละฝ่ายถูกกำหนดโดยเลขที่เป็นข้อมูลเข้า หากข้อมูลเข้าเป็นเลขคู่แสดงว่านักสู้ฝ่ายเลขคู่ทำการโจมตี หากข้อมูลเข้าเป็นคี่แสดงว่านักสู้ฝ่ายเลขคี่ทำการโจมตี 

การโจมตีแต่ละครั้งจะทำให้อีกฝ่ายเสียพลังงาน $1$ หน่วย ทั้งนี้หากฝ่ายใดโจมตีติดต่อกันครั้งที่สามหรือมากกว่าจะถือเป็นท่าชุดโจมตี ซึ่งจะทำให้อีกฝ่ายเสียพลังงานครั้งละ $3$ หน่วยต่อการโจมตี

\textbf{ตัวอย่างเช่น} หากข้อมูลเข้าเป็น $0$ $2$ $4$ $6$ $8$ $1$ $10$ $3$ $7$ $9$ $12$ แสดงว่าฝ่ายเลขคู่โจมตีติดต่อกันถึง 5 ครั้งก่อนที่ฝ่ายเลขคี่จะทำการโจมตีแทรกขึ้นมา จากข้อมูลเข้านี้ฝ่ายเลขคู่ได้ลดพลังงานของฝ่ายเลขคี่เป็นจำนวนทั้งหมด $1 + 1 + 3 + 3 + 3 + 1 + 1$ ซึ่งมาจากเลข $0$ $2$ $4$ $6$ $8$ $10$ $12$ ตามลำดับ โดยเลข $4$ $6$ และ $8$ เป็นการโจมตีติดต่อกันครั้งที่ $3$ $4$ และ $5$ ตาม ลำดับ ทำให้พลังงานฝ่ายเลขคี่ลดลงครั้งละ $3$ หน่วย ส่วนเลข $10$ และ $12$ จะลดพลังงานฝ่ายเลขคี่ได้แค่ครั้งละ $1$ หน่วยเท่านั้นเพราะไม่ใช่การโจมตีที่ติดต่อกันถึงสามครั้ง จากข้อมูลเข้าเดียวกัน ฝ่ายเลขคี่ได้ลดพลังงานฝ่ายเลขคู่เป็นปริมาณเท่ากับ $1 + 1 + 1 + 3$ หน่วยจากตัวเลข $1$ $3$ $7$ และ $9$ ตามลำดับ โดยเลข $9$ ลดพลังงานฝ่ายเลขคู่ $3$ หน่วยเพราะเป็นการโจมตีติดต่อกันเป็นครั้งที่ $3$



\bigskip
\underline{\textbf{โจทย์}} จงเขียนโปรแกรมที่คำนวณหาผู้ชนะจากการประลองครั้งนี้ โดยการประลองจะจบลงทันที เมื่อพลังงานของฝ่ายใดฝ่ายหนึ่งลดลงจนเหลือศูนย์หรือติดลบ ส่วนอีกฝ่ายที่ยังเหลือพลังงานคือผู้ชนะในการประลอง


\InputFile

 \textbf{บรรทัดแรก} รับจำนวนเต็ม $P$ ระบุพลังงานเริ่มต้นที่นักสู้ทั้งสองมี โดยที่  $1 \leq P \leq 500\,000$

\textbf{บรรทัดที่สอง} รับจำนวนเต็มทั้งหมด $2P$ จำนวน ตัวเลขแต่ละตัวคั่นด้วยช่องว่างหนึ่งช่อง


\OutputFile

\textbf{บรรทัดแรก} ระบุผู้ชนะ โดยให้พิมพ์เลข $0$ เมื่อฝ่ายเลขคู่เป็นผู้ชนะ แต่ให้พิมพ์เลข $1$ หากฝ่ายเลขคี่เป็นผู้ชนะ

\textbf{บรรทัดที่สอง} ระบุตัวเลขแรกที่ทำให้ฝ่ายที่แพ้มีพลังชีวิตเหลือศูนย์หรือติดลบ

\textbf{หมายเหตุ:} โปรแกรมไม่จำเป็นต้องรับข้อมูลเข้าทุกตัว หากตัดสินผู้ชนะได้แล้ว (เพราะมีฝ่ายที่พลังงานลดลงเหลือศูนย์หรือน้อยกว่า) โปรแกรมสามารถพิมพ์ผลลัพธ์และจบการทำงานได้เลย

 

\Examples

\begin{example}
\exmp{6
7 5 2 4 8 1 3 9 11 12 13 14}{1
9}%
\exmp{8
1 2 3 4 5 6 7 8 9 10 2 4 6 8 10 12}{0
4}%
\exmp{10
1 2 3 4 5 6 7 8 9 10 11 13 15 16 17 18 19 20 21 22}{1
15}%
\end{example}


\Source

การแข่งขันคอมพิวเตอร์โอลิมปิกระดับชาติครั้งที่ 8 (SUTOI8)

\end{problem}

\end{document}