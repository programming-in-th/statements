\documentclass[11pt,a4paper]{article}

\usepackage{../../templates/style}

\begin{document}

\begin{problem}{Factory}{standard input}{standard output}{1 second}{64 megabytes}

นายใหญ่และนายหญิงมีธุรกิจขนาดยักษ์ โดยธุรกิจเหล่านี้มีสายการผลิตเป็นลำดับที่แน่นอน การผลิตเริ่มจากโรงงานที่หนึ่งไปยังโรงงานที่ $n$ โดยโรงงานแต่ละแห่งนั้นมีกำไรไม่เท่ากัน เนื่องจากนายใหญ่และนายหญิงต้องการแบ่งหน้าที่การควบคุมดูแลโรงงานออกให้ลูกสามคนโดยการแบ่งงานนั้นต้องการให้ลูกแต่ละคนได้กำไรเท่าเทียมกันที่สุด โดยความเท่าเทียมในที่นี้วัดจากผลต่างของกำไรของลูกคนที่ได้กำไรสูงสุดกับของลูกคนที่ได้กำไรต่ำสุด การแบ่งจะถือว่าเท่าเทียมมากที่สุดถ้าค่าผลต่างดังกล่าวมีค่าน้อยที่สุด และลูกแต่ละคนจะต้องได้รับงานที่ต่อเนื่องกันเท่านั้น และจะไม่มีโรงงานไหนที่ไม่ได้รับการดูแล

ยกตัวอย่างในกรณีที่ $n=6$ การแบ่ง $\{1,2\}, \{3,4,5\}$ และ $\{6\}$ ถือเป็นการแบ่งที่ถูกต้อง แต่การแบ่ง $\{1,3\}, \{2\}, \{4,5,6\}$ ถือเป็นการแบ่งที่ผิด เพราะว่าสายการผลิตของโรงงานของลูกคนแรกไม่ต่อเนื่องกัน

\bigskip
\underline{\textbf{โจทย์}}  จงเขียนโปรแกรมเพื่อหาวิธีแบ่งโรงงานของนายใหญ่และนายหญิงให้แก่ลูกทั้งสาม

\InputFile

\textbf{บรรทัดแรก} รับจำนวนเต็ม $n$ ซึ่งเป็นจำนวนของโรงงานทั้งหมด $(1 \leq n \leq 1\,000)$ 

\textbf{บรรทัดที่สอง}  รับจำนวนเต็ม $n$ จำนวน แต่ละจำนวนคั่นด้วยช่องว่าง $1$ ช่อง, $a_1$ $a_2$ $a_3$ ... $a_n$ โดย $a_i$ แทนกำไรของโรงงานที่ $i$ ซึ่งกำไรของแต่ละโรงงานจะอยู่ในช่วง $-1\,000 \leq a_i \leq 1\,000$

\OutputFile

\textbf{มีบรรทัดเดียว} ข้อมูลส่งออกประกอบด้วยจำนวนเต็มสองตัวเว้นด้วยช่องว่างหนึ่งช่อง โดยตัวแรกคือหมายเลขโรงงานของลูกคนที่สอง และตัวที่สองคือหมายเลขโรงงานของลูกคนที่สาม

\Examples

\begin{example}
\exmp{5
3 2 5 -1 6}{3 4}%
\end{example}


\Source

การแข่งขันคณิตศาสตร์ วิทยาศาสตร์ โอลิมปิกแห่งประเทศไทย สาขาวิชาคอมพิวเตอร์ ประจำปี 2548

\end{problem}

\end{document}