\documentclass[11pt,a4paper]{article}

\usepackage{../../templates/style}

\begin{document}

\begin{problem}{หุ่นยนต์ (robot)}{standard input}{standard output}{3 second}{32 megabytes}

มีเสา $N$ ต้น ตั้งอยู่ในสนาม เสาแต่ละต้นจะมีหมายเลขกำกับตั้งแต่ $1, 2, 3$ เรียงไปเรื่อยๆ จนถึง $N$

คุณได้ประดิษฐ์หุ่นยนต์ตัวหนึ่งที่มีคุณสมบัติว่า เมื่อนำหุ่นยนต์ดังกล่าวไปปล่อยไว้ในสนาม หุ่นยนต์จะเดินเป็นเส้นตรงไปหาเสาที่อยู่ใกล้ที่สุด (วัดจากตำแหน่งปัจจุบันของหุ่นยนต์) ที่มันยังไม่เคยไปถึง (หากมีเสามากกว่า $1$ ต้นที่อยู่ใกล้ที่สุดเท่ากันพอดี หุ่นยนต์จะเลือกเดินไปหาเสาที่มีหมายเลขน้อยที่สุด) และเมื่อเดินไปถึงเสาต้นนั้นแล้ว ก็จะเดินต่อไปหาเสาที่อยู่ใกล้ที่สุดที่มันยังไม่เคยไปถึงต่อไปเรื่อยๆ จนกระทั่งไปถึงเสาครบทุกต้นก็จะหยุดเดิน

\bigskip
\underline{\textbf{โจทย์}}  จงเขียนโปรแกรมเพื่อรับตำแหน่งของเสาแต่ละต้น แล้วตอบคำถามทั้งหมด $Q$ คำถามว่า หากเริ่มปล่อยหุ่นยนต์ที่พิกัด $(X, Y)$ เสาหมายเลข $K$ จะเป็นเสาลำดับที่เท่าไรที่หุ่นยนต์เดินไปถึง



\InputFile

\textbf{บรรทัดแรก} ระบุจำนวนเต็ม $N$ และ $Q $ $(1 \leq  N \leq 1\,000; 1 \leq Q \leq 100\,000)$ แทนจำนวนเสาในสนาม และจำนวนคำถาม

\textbf{บรรทัดที่ $2$ ถึง $N+1$}  ในบรรทัดที่ $i+1$ $(1 \leq i \leq N)$ ระบุจำนวนเต็ม $X_i$ และ $Y_i$ $(1 \leq X_i,Y_i \leq 10\,000)$ แทนพิกัดตามแกน $X$ และแกน $Y$ ของเสาหมายเลข $i$

\textbf{บรรทัดที่ $N+2$ ถึง $N+Q+1$} ในบรรทัดที่ $N+i+1$ $(1 \leq i \leq Q)$ ระบุจำนวนเต็ม $X, Y$ และ $K$ $(1 \leq X,Y \leq 10\,000; 1 \leq K \leq N)$ แสดงถึงคำถามที่ $i$



\OutputFile

\textbf{มี $Q$ บรรทัด} โดยในบรรทัดที่ $i$ $(1 \leq i \leq N)$ แสดงคำตอบของคำถามที่ $i$

\Examples

\begin{example}
\exmp{3 4
2 2
4 1
5 3
2 1 1
2 1 2
4 4 3
3 4 3}{1
2
1
3}%
\exmp{5 5
5 6
2 2
3 1
5 4
3 3
1 1 5
2 1 3
3 1 1
4 3 4
3 3 3}{3
2
5
4
3}%
\end{example}

\Scoring

\textbf{$20$\% ของข้อมูลทดสอบ:} $N \leq 100;Q \leq 1\,000$

\textbf{$50$\% ของข้อมูลทดสอบ:} $N \leq 100$

\Source

สุธี เรืองวิเศษ


\end{problem}

\end{document}