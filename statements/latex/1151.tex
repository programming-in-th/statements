\documentclass[11pt,a4paper]{article}

\usepackage{../../templates/style}

\begin{document}

\begin{problem}{ต้นไม้ (Tree)}{standard input}{standard output}{1 second}{32 megabytes}

 คุณต้องการที่จะมองต้นไม้จากจุดหนึ่ง โดยต้นไม้จะเรียงอยู่บนเส้นจำนวน โดยจะมีต้นไม้ $N$ ต้น ต้นที่ $i$ จะตั้งอยู่ที่พิกัด $i$ บนเส้นจำนวน และมีความสูง $H_i$

                คุณยืนอยู่ที่จุดพิกัด $0$ และต้องการที่จะมองไปยังต้นไม่เหล่านี้ โดยคุณมีข้อจำกัดที่ว่า คุณจะสามารถมองเห็นต้นไม้ต้นที่ $j$ ได้หาก สำหรับทุก $i < j$ แล้วจะมีค่า $H_i < H_j$ กล่าวคือต้นไม้ต้นนั้นไม่ถูกบังด้วยต้นก่อนหน้า

                คุณต้องการจะเห็นจำนวนต้นไม้มากที่สุด โชคดีที่คุณมีขวานวิเศษที่สามารถตัดต้นไม้ออกไปกี่ต้นก็ได้

                

\bigskip
\underline{\textbf{โจทย์}}  กำหนดความสูงของต้นไม้แต่ละต้น จงหาว่าหากคุณตัดต้นไม้อย่างดีที่สุดแล้ว เมื่อคุณยืนอยู่ที่จุดพิกัด $0$ คุณจะมองเห็นต้นไม้กี่ต้น


\InputFile

\textbf{บรรรทัดแรก} รับจำนวนนับ $N$ แทนจำนวนต้นไม้ $( 1 \leq N \leq 200\,000 )$

\textbf{บรรทัดที่สอง} รับจำนวนนับ $N$ จำนวน แทนความสูงของต้นไม้แต่ละต้น เรียงตามลำดับจากซ้ายไปขวาบนเส้นจำนวน $( 1 \leq H_i \leq 1\,000\,000 )$


\OutputFile

\textbf{มีบรรทัดเดียว} แสดงจำนวนต้นไม้กี่คุณจะเห็นมากที่สุดหากคุณตัดต้นไม้อย่างดี

\Examples

\begin{example}
\exmp{6
5 6 3 4 4 5}{3}%
\exmp{5
4 1 4 1 1}{2}%
\end{example}


\Source

Programming.in.th (PS.int)

\end{problem}

\end{document}