\documentclass[11pt,a4paper]{article}

\usepackage{../../templates/style}

\begin{document}

\begin{problem}{รถ (Cars)}{standard input}{standard output}{0.4 second}{32 megabytes}

     คุณเป็นเจ้าของสนามแข่งรถแห่งหนึ่งซึ่งกำลังจัดแข่งขันรถแข่งชิงแชมป์จังหวัด โดยในการแข่งขันครั้งนี้มีรถเข้าร่วมการแข่งขันทั้งสิ้น $n$ คัน และจะขับรถแข่งไปเรื่อยๆจนกว่าคุณจะพอใจ (เอาเป็นว่านานมากๆๆๆๆๆๆ)

            ข้อสังเกตที่สำคัญคือ \textbf{รถแต่ละคันจะมีค่าของความเร็วเฉพาะตัวอยู่ซึ่งจะมีค่าไม่ซ้ำกันเสมอ}

            ปัญหาของการจัดแข่งรถที่สำคัญมากที่สุดคือ \textit{รถชน} เนื่องจากรถแข่งที่นำมาแข่งขันกันนั้นจะขับด้วยความเร็วเฉพาะตัวของมันโดยตลอด (ไม่มีการเบรกหรือการชะลอ) แต่คุณก็ได้คิดวิธีการอันชาญฉลาดขึ้น นั่นคือการให้รถแต่ละคันนั้นออกตัวที่เวลาที่แตกต่างกัน กล่าวคือคุณจะระบุหมายเลขตั้งแต่ $1$ ถึง $n$ ไว้ที่รถแต่ละคัน (โดยไม่ซ้ำกัน) และอนุญาติให้รถออกจากจุดเริ่มต้นตามหมายเลข หมายเลข $1$ ออกเป็นคันแรก และหมายเลข $n$ ออกเป็นคันสุดท้าย\textbf{ ซึ่งคุณพบว่าจะมีรูปแบบการจัดการออกตัวของรถอยู่ $1$ รูปแบบเสมอ ที่จะไม่มีทางเกิด \textit{รถชน} ได้อย่างแน่นอน}

\bigskip
\underline{\textbf{โจทย์}}  กำหนดค่าความเร็วของรถแต่ละคัน จงแสดงค่าของความเร็วของรถที่ถูกระบุเป็นหมายเลข $k$ (นั่นคือออกจากจุดเริ่มต้นเป็นลำดับที่ $k$)


\InputFile

\textbf{บรรทัดแรก} ประกอบด้วยจำนวนนับ $n$ และ $k$ แทนจำนวนของรถที่เข้าร่วมแข่ง และหมายเลขของรถที่คุณต้องการทราบความเร็ว $(2 \leq n \leq  1\,000\,000 ; 1 \leq k \leq n )$

\textbf{บรรทัดที่ $2$ ถึง $n+1$} แต่ละบรรทัดแสดงค่าความเร็วของรถแต่ละคัน โดยจะเป็นจำนวนนับค่าระหว่าง $1$ ถึง $2\,000\,000$ (รวม $1$ และ $2\,000\,000$) และจะไม่ซ้ำกัน


\OutputFile

\textbf{มีบรรทัดเดียว} แสดงค่าของความเร็วของรถที่ถูกระบุเป็นหมายเลข $k$ (นั่นคือออกจากจุดเริ่มต้นเป็นลำดับที่ $k$)

\textbf{หมายเหตุ} กรุณาศึกษา \textit{Time Limit} และ \textit{Memory Limit} ของโจทย์ข้อนี้อย่างละเอียด

\Examples

\begin{example}
\exmp{5 4
7
1
8
3
2}{2}%
\exmp{5 1
7
1
8
3
2}{8}%
\end{example}

\Scoring

\textbf{$10$\% ของชุดทดสอบทั้งหมด:} $n \leq 10$

\textbf{$30$\% ของชุดทดสอบทั้งหมด: }$n \leq 5\,000$

\textbf{$80$\% ของชุดทดสอบทั้งหมด:} $n \leq 300\,000$

\textbf{$100$\% ของชุดทดสอบทั้งหมด:} $n \leq 1\,000\,000$

\Source

สรวิทย์  สุริยกาญจน์ ( PS.int )

ศูนย์ สอวน. โรงเรียนมหิดลวิทยานุสรณ์

\end{problem}

\end{document}