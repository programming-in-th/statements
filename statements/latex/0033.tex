\documentclass[11pt,a4paper]{article}

\usepackage{../../templates/style}

\begin{document}

\begin{problem}{ลูบไล้ (looblike)}{standard input}{standard output}{1 second}{32 megabytes}

เว็บไซต์สังคมออนไลน์แห่งหนึ่งเป็นแหล่งรวมผู้นิยมการขัดผิวด้วยสมุนไพรต่างๆ เมื่อผู้ใช้คนใดมีข่าวสารในวงการเครื่องประทินผิวก็จะนำมาเผยแพร่ทางเว็บไซต์แห่งนี้ หากผู้ใช้เว็บไซต์คนอื่นๆ เห็นว่าข่าวดังกล่าวเป็นประโยชน์ก็สามารถกดปุ่ม \textit{“ลูบไล้”} เพื่อแสดงความชื่นชม และทุกครั้งที่ \textit{“ลูบไล้”} ก็สามารถใส่ความคิดเห็นเพิ่มเติมลงไปได้ หากผู้ใช้เว็บไซต์คนอื่นๆ เห็นว่าการ \textit{“ลูบไล้”} นั้นเป็นประโยชน์ก็สามารถกดปุ่ม \textit{“ลูบไล้”} การ \textit{“ลูบไล้”} นั้นได้ ทำให้เกิดเป็นการ  \textit{“ลูบไล้”} แตกแขนงไม่รู้จบสิ้น

นิตยาเพิ่งเข้าสู่วงการสมุนไพรประทินผิวได้ไม่นานและเห็นเว็บไซต์แห่งนี้เป็นครั้งแรก เนื่องจากเธอไม่อยากไล่อ่านข้อความทั้งหมดในเว็บไซต์ เธอจึงอยากให้คุณช่วงหาความคิดเห็นที่ถูก \textit{“ลูบไล้”} มากที่สุด เพื่อที่จะนำความคิดเห็นนั้นไปใช้เสริมความงามของเธอเอง

\underline{\textbf{โจทย์}} จงเขียนโปรแกรมรับข้อมูลการ \textit{“ลูบไล้”} ทั้งหมด แล้วหาความคิดเห็นที่ถูก\textit{ “ลูบไล้”} มากที่สุด

\InputFile

\textbf{บรรทัดแรก} ระบุจำนวนเต็ม $N$ $(2 \leq N \leq 1\,000)$ แทนจำนวนการ \textit{“ลูบไล้”} ทั้งหมด

\textbf{บรรทัดที่สอง} ระบุจำนวนเต็ม $N$ จำนวน แทนหมายเลขความคิดเห็นที่ถูก \textit{“ลูบไล้”} หมายเลขความคิดเห็นเป็นจำนวนเต็มตั้งแต่ $1$ ถึง $10\,000$

\OutputFile

\textbf{มีบรรทัดเดียว} ระบุหมายเลขความคิดเห็นที่ถูก \textit{“ลูบไล้”} มากที่สุด หากมีหลายความคิดเห็นที่ถูก \textit{“ลูบไล้”} มากที่สุดเท่ากันให้พิมพ์หมายเลขความคิดเห็นเหล่านั้นจากน้อยไปมาก คั่นด้วยช่องว่างหนึ่งช่อง

\Examples

\begin{example}
\exmp{7
4 3 9 8 3 3 8}{3}%
\exmp{7
12 2 1 12 1 1 12}{1 12}%
\end{example}

\newpage
\Source

โจทย์โดย: ธนะ วัฒนวารุณ

การแข่งขัน IOI Thailand League เดือนสิงหาคม 2553

\end{problem}

\end{document}