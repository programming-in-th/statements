\documentclass[11pt,a4paper]{article}

\usepackage{../../templates/style}

\begin{document}

\begin{problem}{มัธยฐาน (Median)}{standard input}{standard output}{1 second}{128 megabytes}

 กำหนดลำดับหลักของตัวเลข $n$ ตัว ( ประกอบด้วย $1$ ถึง $n$ และไม่ซ้ำกัน ) จงหาว่ามีลำดับย่อยที่มีค่ามัธยฐานเท่ากับ $k$ ทั้งสิ้นกี่ลำดับ

กำหนดให้ลำดับย่อยมีคุณสมบัติดังนี้
\begin{enumerate}

\item ประกอบด้วยตัวเลข $m$ ตัว โดยเริ่มตั้งแต่ตัวเลขที่ $i$ ถึงตัวเลขที่ $i+m-1$ ของลำดับหลัก เมื่อ $i$ เป็นจำนวนนับใดๆ ที่ $1 \leq i \leq n$ และ $i+m-1 \leq n$
\item $m$ เป็นเลขคี่
\item ค่ามัธยฐานของลำดับย่อย คือค่าของตัวเลขที่มีค่าเป็นลำดับ $(m+1)/2$ เมื่อนำตัวเลขในลำดับย่อยมาเรียงจากน้อยไปมาก
\end{enumerate}

\bigskip
\underline{\textbf{โจทย์}}  จงเขียนโปรแกรมเพื่อหาว่ามีลำดับย่อยทั้งหมดเท่าไหร่ที่มีมัธยฐานเท่ากับค่า $k$


\InputFile

\textbf{บรรทัดแรก} ประกอบด้วยจำนวนนับ $n$ และ $k$ แทนจำนวนตัวเลขในลำดับหลัก และค่ามัธยฐานที่ต้องการทราบจำนวน $( 1 \leq k \leq n \leq 1\,000\,000)$

\textbf{บรรทัดที่ $2$ ถึง $n+1$} แต่ละบรรทัดจะประกอบด้วยเลขโดด $1$ จำนวน โดยในบรรทัดที่ $i+1$ จะแสดงค่าของเลขลำดับที่ $i$ ในลำดับหลัก


\OutputFile

\textbf{มีบรรทัดเดียว} แสดงจำนวนลำดับย่อยที่มีค่ามัธยฐานเท่ากับ $k$
\bigskip

\textbf{หมายเหตุ:} คำตอบอาจมีค่ามากเกินว่าที่ int จะรองรับได้ คุณควรใช้ long long ในการเก็บค่าคำตอบ

\Examples

\begin{example}
\exmp{10 5
6 3 9 4 7 5 10 8 2 1}{6}%
\exmp{10 5
10 2 6 4 7 1 5 8 9 3}{10}%
\end{example}

\Scoring

\textbf{$30\%$ ของชุดทดสอบทั้งหมด:} $n \leq 10$

\textbf{$50\%$ ของชุดทดสอบทั้งหมด:} $n \leq 1\,000$

\textbf{$70\%$ ของชุดทดสอบทั้งหมด:} $n \leq 100\,000$
            
\textbf{$100\%$ ของชุดทดสอบทั้งหมด:} $n \leq 1\,000\,000$
            
            
\Source

สรวิทย์  สุริยกาญจน์ ( PS.int )

ศูนย์ สอวน. โรงเรียนมหิดลวิทยานุสรณ์

\end{problem}

\end{document}