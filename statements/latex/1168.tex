\documentclass[11pt,a4paper]{article}

\usepackage{../../templates/style}

\begin{document}

\begin{problem}{ลำดับเลขไม่ซ้ำหลักไม่ลด}{standard input}{standard output}{0.1 second}{4 megabytes}

วิชาคอมพิวเตอร์ กับวิชาคณิตศาสตร์นั้นอยู่ห่างกันเพียงเอื้อมมือ นักคณิตศาสตร์เป็นผู้มีจินตนาการและความคิดสร้างสรรค์ล้ำเลิศ พวกเขาจึงสรรค์สร้างสิ่งใหม่ๆออกมาได้ตลอดเวลา ในวันนี้ก็เช่นกัน พวกเขาได้สร้างลำดับแบบใหม่ขึ้นมาบนโลกนี้ เรียกว่า “ลำดับเลขไม่ซ้ำหลักไม่ลด” หรือมีชื่อภาษาอังกฤษว่า \textit{“Non – Repeat Decrease Sequence” (NRDS)} ลำดับนี้เป็นลำดับของจำนวนเต็มบวกที่เลขโดดในแต่ละหลักไม่มีหลักใดซ้ำกันเลย และเลขโดดจะมีค่าไม่ลดลงเมื่อพิจารณาจากหลักที่มีค่าประจำหลักมากสุดมายัง หลักหน่วย ตัวอย่างของเลขในลำดับนี้ เช่น $1$, $23$, $127$ และตัวอย่างของเลขที่ไม่อยู่ในลำดับนี้ เช่น $131$, $101$, $609$

                แต่ถึงกระนั้น นักคณิตศาสตร์เหล่านั้นก็ไม่ยอมเปิดเผยลำดับของเลขนี้ออกมาทั้งหมด พวกเขาเปิดเผยมันออกมาแค่ $10$ จำนวนแรก ดังนี้ $1$, $2$, $3$, $4$, $5$, $6$, $7$, $8$, $9$, $12$ พร้อมกับท้าทายคนทั้งประเทศว่าหากใครสามารถหาจำนวนที่ $N$ ในลำดับนี้ได้ พวกเขาจะให้รางวัลตอบแทนอย่างงาม คุณซึ่งเป็นนักคอมพิวเตอร์ซึ่งคลั่งไคล้ในตัวเลข เห็นว่าลำดับนี้นั้นมีความน่าสนใจอย่างมาก คุณจึงพยายามที่จะเขียนโปรแกรมเพื่อหาจำนวนที่ $N$ ของลำดับนี้ให้จงได้

\bigskip
\underline{\textbf{โจทย์}}  จงเขียนโปรแกรมเพื่อหาจำนวนในลำดับ  \textit{“Non – Repeat Decrease Sequence” (NRDS)} อันดับที่ $N$


\InputFile

\textbf{บรรทัดแรก} ระบุจำนวนเต็ม Q (1 < Q < 100,000) แทนจำนวนชุดทดสอบย่อย 

\textbf{บรรทัดที่ $2$ ถึง $Q+1$} ในบรรทัดที่ $i+1$ สำหรับ $1 \leq i \leq Q$ ระบุค่า $N$ $(1 \leq N \leq 2^{32})$ ของชุดทดสอบย่อยที่ $i$


\OutputFile

\textbf{มี $Q$ บรรทัด} ในบรรทัดที่ $i$ สำหรับ $1 \leq i \leq Q$ แสดงจำนวนที่ $N$ ใน \textit{NRDS} ของชุดทดสอบย่อยที่ $i$ ถ้าหากว่าไม่มีจำนวนที่ $N$ ใน \textit{NRDS} ให้แสดง $-1$ ออกมาแทน

\Examples

\begin{example}
\exmp{11
1
2
3
4
5
6
7
8
9
10
2000000000}{1
2
3
4
5
6
7
8
9
12
-1}%
\end{example}


\Source

IOI Thailand League 2010 เดือนพฤษภาคม

\end{problem}

\end{document}