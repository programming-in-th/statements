\documentclass[11pt,a4paper]{article}

\usepackage{../../templates/style}

\begin{document}

\begin{problem}{อาคารเรียน (campus)}{standard input}{standard output}{0.5 second}{32 megabytes}

ณ สถาบันเทคโนโลยีแห่งหนึ่งทางภาคตะวันออกเฉียงเหนือ มีหอพักอยู่ทั้งหมด $N$ หอ ตั้งอยู่บนพิกัดที่เป็นจำนวนเต็มบนระนาบสองมิติ แต่ละหอก็มีนักเรียนจำนวนหนึ่งพักอยู่

วันหนึ่ง ทางสถาบันได้วางแผนที่จะสร้างอาคารเรียนแห่งใหม่ขึ้นมา $1$ อาคาร ซึ่งต้องตั้งอยู่บนพิกัดที่เป็นจำนวนเต็มเช่นกัน โดยต้องการให้ระยะทางรวมที่นักเรียนทุกคนใช้ในการเดินไปเรียนมีค่าน้อยที่สุดเท่าที่จะทำได้

เนื่องจากทางเดินในสถาบันแห่งนี้มีลักษณะเป็นตารางสี่เหลี่ยมจัตุรัส หากหอพักของนักเรียนตั้งอยู่ที่พิกัด $(x_1, y_1)$ และอาคารเรียนตั้งอยู่ที่พิกัด $(x_2, y_2)$ นักเรียนจะต้องเดินมาเรียนเป็นระยะทาง $|x_1 - x_2| + |y_1 - y_2|$ นอกจากนี้ หอพักแต่ละหอยังมีขนาดเล็กมาก จึงอาจมีหอพักมากกว่า $1$ หอ ตั้งอยู่ที่พิกัดเดียวกันได้ และตำแหน่งที่จะสร้างอาคารเรียน อาจตรงกับพิกัดของหอพักบางหอก็ได้

\bigskip
\underline{\textbf{โจทย์}}  จงเขียนโปรแกรมเพื่อรับตำแหน่งที่ตั้งและจำนวนนักเรียนในหอพักแต่ละหอ แล้วคำนวณหาระยะทางรวมที่น้อยที่สุดที่นักเรียนทุกคนใช้ในการเดินไปเรียน


\InputFile

\textbf{บรรทัดแรก} ระบุจำนวนเต็ม $N$ $(1 \leq N \leq 500\,000)$ แทนจำนวนหอพักทั้งหมด

\textbf{บรรทัดที่ $2$ ถึง $N+1$} ในบรรทัดที่ $i+1$ $(1 \leq i \leq N)$ ระบุจำนวนเต็ม $X_i, Y_i$ และ $S_i$ $(1 \leq X_i,Y_i \leq 1\,000\,000\,000; 1 \leq S_i \leq 1\,000)$ แทนพิกัดบนแกน $X$ พิกัดบนแกน $Y$ และจำนวนนักเรียนในหอพักที่ $i$


\OutputFile

\textbf{มีบรรทัดเดียว} แสดงระยะทางรวมที่น้อยที่สุดที่นักเรียนทุกคนใช้ในการเดินไปเรียน

\Examples

\begin{example}
\exmp{4
1 2 1
2 1 1
2 3 1
3 2 1}{4}%
\exmp{3
1 1 1
6 3 3
4 8 2}{21}%
\end{example}

\Scoring

\textbf{$30$\% ของข้อมูลทดสอบ:} $S_i = 1$ สำหรับทุกจำนวนเต็ม $i$  $(1 \leq i \leq N )$

\textbf{$60$\% ของข้อมูลทดสอบ:} $N \leq 100\,000$

\textbf{$20$\% ของข้อมูลทดสอบ:} สอดคล้องกับเงื่อนไขด้านบนทั้งสองข้อ

\Source

สุธี เรืองวิเศษ

\end{problem}

\end{document}