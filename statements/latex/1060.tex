\documentclass[11pt,a4paper]{article}

\usepackage{../../templates/style}


\begin{document}

\begin{problem}{Cromartie Mountain}{standard input}{standard output}{1 second}{64 megabytes}

นักสำรวจได้สำรวจเทือกเขาคุโรมาตี้ ซึ่งทอดยาวเป็นเส้นตรง และประกอบด้วยภูเขารูปสามเหลี่ยมหน้าจั่วจำนวน $N$ ลูก แล้วจดบันทึกเฉพาะ ตำแหน่งเริ่มต้น $S$ และความสูงของภูเขา $H$ แต่ละลูกเอาไว้ ภูเขาแต่ละลูกประกอบด้วยสัญลักษณ์ เนินเขา (\textbf{‘/’} หรือ \textbf{‘\textbackslash’}) และพื้นที่ป่าไม้ (\textbf{‘X’}) โดยที่รูปของภูเขาสัมพันธ์กับความสูง ดังตัวอย่าง

\begin{center}
\begin{tabular}{|c|c|c|c|c|}
\hline
\multicolumn{5}{|c|}{ตารางที่ 1 แสดงตัวอย่างความสัมพันธ์ระหว่างรูป}\\
\multicolumn{5}{|c|}{กับความสูงของภูเขาแต่ละระดับ $H = 1 … 5$}\\
\hline \hline
& & & &/\textbackslash \\
& & & /\textbackslash & /XX\textbackslash\\
& & /\textbackslash & /XX\textbackslash & /XXXX\textbackslash\\
& /\textbackslash & /XX\textbackslash & /XXXX\textbackslash &/XXXXXX\textbackslash\\
/\textbackslash & /XX\textbackslash & /XXXX\textbackslash & /XXXXXX\textbackslash & /XXXXXXXX\textbackslash \\
\hline
$H = 1$ & $H = 2$ & $H = 3$ & $H = 4$ & $H = 5$ \\
\hline
\end{tabular}
\end{center}

สังเกตว่า เมื่อ $H=1$ จะมีเฉพาะยอดเขาเท่านั้น ไม่มีพื้นที่ป่าไม้

\bigskip
\underline{\textbf{โจทย์}}  จากบันทึกของนักสำรวจ จงเขียนโปรแกรมวาดรูปเทือกเขาคุโรมาตี้ โดยมีเงื่อนไขต่อไปนี้
\begin{itemize}
\item ไม่มีภูเขาลูกใด ถูกภูเขาอื่นบังมิดทั้งลูก (จำนวนของยอดเขาที่ปรากฏต้องเท่ากับจำนวนของภูเขาที่บันทึกไว้)
\item ตำแหน่งที่ เนินเขา (‘\textbackslash’ และ ‘/’) ของภูเขาสองลูกเหลื่อมกันให้แทนด้วยตัวอักษรวีพิมพ์เล็ก (‘v’)
\item ตำแหน่งที่ เนินเขา ของภูเขาลูกหนึ่ง เหลื่อมกับพื้นที่ป่าไม้ของภูเขาอีกลูกหนึ่งให้ถือว่าเป็นพื้นที่ป่าไม้ ให้แทนด้วยตัวอักษรเอ็กซ์พิมพ์ใหญ่ (‘X’)
\item ในการแสดงผลลัพธ์ สำหรับพื้นที่ว่างให้แสดงด้วยเครื่องหมายจุด (‘.’) เท่านั้น
\end{itemize}

\newpage
\textbf{ตัวอย่างเช่น} มีภูเขา $3$ ลูก ซึ่งมีค่า $(S, H)$ เป็นดังนี้ $(4, 6)$ $(1, 4)$ $(15, 3)$ จะสามารถแสดงเป็นเทือกเขาได้ดังนี้

\begin{tabular}{c}\ttfamily
......../\textbackslash..........\\\ttfamily

......./XX\textbackslash.........\\\ttfamily

.../\textbackslash./XXXX\textbackslash........\\\ttfamily

../XXvXXXXXX\textbackslash.../\textbackslash..\\\ttfamily

./XXXXXXXXXXX\textbackslash./XX\textbackslash.\\\ttfamily

/XXXXXXXXXXXXXvXXXX\textbackslash\\
\end{tabular}

\InputFile

\textbf{บรรทัดแรก} รับค่า $N$ แทนจำนวนของภูเขาที่นักสำรวจบันทึกไว้ $(1 \leq N \leq 21)$ 

\textbf{บรรทัดที่ $2$ ถึง $N+1$} ในบรรทัดที่ $i+1$ ให้รับข้อมูลของภูเขาลูกที่ $i$ โดยแสดงตำแหน่งเริ่มต้น $S$ และความสูงของภูเขา $H$ ของภูเขาลูกที่ $i$ คั่นด้วยช่องว่าง $(1 \leq S \leq 60;1 \leq H \leq 10)$ 

\OutputFile

\textbf{มีหลายบรรทัด} แสดงรูปของเทือกเขาคุโรมาตี้ โดยความสูงของรูปเท่ากับความสูงของยอดเขาที่สูงที่สุด ตัวอักษรซ้ายสุดของรูปตรงกับตำแหน่ง $S = 1$ และขอบด้านขวาสุดต้องเป็นส่วนหนึ่งของภูเขาอย่างน้อย $1$ ลูก

\Examples

\begin{tabular}{|l|l|}
        \hline
        \multicolumn{1}{|c|}{\bf\InputFileName}&
        \multicolumn{1}{c|}{\bf\OutputFileName}\\
        \hline
        
        \begin{tabular}{l}
        \ttfamily 3\\
		\ttfamily5 6\\
		\ttfamily2 4\\
		\ttfamily16 3\\
        \end{tabular} &
        
        \begin{tabular}{l}
        \\
        \ttfamily........./\textbackslash..........\\
		\ttfamily......../XX\textbackslash.........\\
		\ttfamily..../\textbackslash./XXXX\textbackslash........\\
		\ttfamily.../XXvXXXXXX\textbackslash.../\textbackslash..\\
		\ttfamily../XXXXXXXXXXX\textbackslash./XX\textbackslash.\\
		\ttfamily./XXXXXXXXXXXXXvXXXX\textbackslash
        \end{tabular} \\[9ex]
        \hline
        
\end{tabular}    

\begin{tabular}{|l|l|}
        \hline
        \multicolumn{1}{|c|}{\bf\InputFileName}&
        \multicolumn{1}{c|}{\bf\OutputFileName}\\
        \hline        
         
         \begin{tabular}{l}
	    \ttfamily 5\\
		\ttfamily1 4\\
		\ttfamily6 7\\
		\ttfamily12 6\\
		\ttfamily21 5\\
		\ttfamily41 3\\
        \end{tabular} &
        \begin{tabular}{l}
       \\
        \ttfamily.........../\textbackslash.................................\\
	\ttfamily	........../XX\textbackslash../\textbackslash............................\\
	\ttfamily	........./XXXX\textbackslash/XX\textbackslash...../\textbackslash....................\\
	\ttfamily	.../\textbackslash.../XXXXXXXXXX\textbackslash.../XX\textbackslash...................\\
	\ttfamily	../XX\textbackslash./XXXXXXXXXXXX\textbackslash./XXXX\textbackslash............../\textbackslash..\\
	\ttfamily	./XXXXvXXXXXXXXXXXXXXvXXXXXX\textbackslash............/XX\textbackslash.\\
	\ttfamily	/XXXXXXXXXXXXXXXXXXXXXXXXXXXX\textbackslash........../XXXX\textbackslash\\
    \\
        \end{tabular} \\[9ex]
        
        \hline
\end{tabular}


\end{problem}

\end{document}