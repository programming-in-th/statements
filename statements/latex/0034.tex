\documentclass[11pt,a4paper]{article}

\usepackage{../../templates/style}

\begin{document}

\begin{problem}{สมการกำลังสอง (quadeq)}{standard input}{standard output}{1 second}{32 megabytes}

ว่ากันว่าชีวิตช่วงมัธยมต้นเป็นชีวิตที่สดใส มีแต่ความสุข แต่นั่นไม่เป็นจริงสำหรับคุณ ไม่ใช่เพราะว่าคุณกำลังมีเรื่องกลุ้มใจ แต่เป็นเพราะการบ้านวิชาคณิตศาสตร์ เรื่อง สมการกำลังสอง (Quadratic Equation) ที่กองเป็นภูเขาของคุณ คุณอยากจะไปเล่นกับเพื่อนๆมาก แต่คุณก็ต้องมานั่งปั่นการบ้านที่จะต้องส่งในวันพรุ่งนี้ คุณอยากจะทำมันให้เสร็จโดยเร็วที่สุด แต่ไม่ใช่ด้วยการลอกเพื่อน…

คุณตัดสินใจที่จะเขียนโปรแกรมแก้สมการกำลังสองออกมา สมการนี้สามารถเขียนในรูปทั่วไปได้เป็น $Ax^2 + Bx + C = 0$ โดยวิธีที่จะแก้สมการนี้ได้นั้น คุณจะต้องแยกตัวประกอบของมันออกมาเป็น $(ax + b)(cx + d)$ เมื่อ $A = ac, B = ad + bc, C = bd$ และ $a, b, c, d$ เป็นจำนวนเต็ม $(a,c > 0)$ ในการบ้านของคุณมีค่า $A, B, C$ มาให้ คุณต้องเขียนโปรแกรมเพื่อหาค่า $a, b, c$ และ $d$ ที่เป็นไปตามเงื่อนไขดังกล่าว

\underline{\textbf{โจทย์}} จงเขียนโปรแกรมค่า $A,B,C$ แล้วแสดงผลเป็นจำนวนเต็ม $a,b,c$ ตามที่โจทย์ต้องการ

\InputFile

\textbf{มีบรรทัดเดียว} จำนวนเต็ม $A, B$ และ $C$ คั่นด้วยช่องว่าง $1$ ช่อง $(1  \leq  A  \leq  100;  -10\,000 \leq B \leq 10\,000;  -100 \leq C \leq 100)$

\OutputFile

\textbf{มีบรรทัดเดียว} แสดงจำนวนเต็ม $a, b, c$ และ $d$ ที่เป็นไปตามเงื่อนไขคั่นด้วยช่องว่าง $1$ ช่อง

หากมีคำตอบที่เป็นไปได้หลายชุด ให้ตอบคำตอบที่มีค่า $a$ น้อยที่สุด หากมีคำตอบที่มีค่า $a$ น้อยที่สุดเท่ากันหลายชุด ให้ตอบคำตอบที่มีค่า $b$ น้อยที่สุดในบรรดาคำตอบเหล่านั้น และหากไม่มีคำตอบที่เป็นไปได้เลย ให้พิมพ์คำว่า "No Solution"

\Examples

\begin{example}
\exmp{4 5 1}{1 1 4 1}%
\exmp{1 1 1}{No Solution}%
\end{example}

\Source

โจทย์โดย: จิรายุ ลือเวศย์วณิช

การแข่งขัน IOI Thailand League เดือนสิงหาคม 2553

\end{problem}

\end{document}