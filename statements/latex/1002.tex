\documentclass[11pt,a4paper]{article}

\usepackage{../../templates/style}

\begin{document}

\begin{problem}{Roman}{standard input}{standard output}{1 second}{64 megabytes}

\textbf{เลขโรมัน}มักจะถูกใช้เป็นเลขหน้าในบทนำของหนังสือก่อนที่จะเริ่มเข้าสู่เนื้อหา โดยสัญลักษณ์เลขโรมันที่ใช้แทนจำนวนเต็มฐานสิบที่มีค่าไม่เกิน $100$ ประกอบด้วย ‘I’ ‘V’ ‘X’ ‘L’ และ ‘C’ ซึ่งใช้แทนจำนวนเต็มค่า $1$ $5$ $10$ $50$ และ $100$ ตามลำดับ

ในการเขียนแทนจำนวนเต็มฐานสิบแต่ละจำนวนในกลุ่มดังกล่าวด้วยเลขโรมัน เราสามารถทำได้โดยการเรียงลำดับสัญลักษณ์เลขโรมันแต่ละตัวต่อเนื่องกันจากซ้ายไปขวา โดยมีเงื่อนไข คือ
\begin{enumerate}

\item สามารถวางสัญลักษณ์ที่ใช้แทนค่า $1$ และ $10$ ต่อเนื่องกันได้ไม่เกิน $3$ ตัว
\item ไม่สามารถวางสัญลักษณ์ที่ใช้แทนค่า $5$ และ $50$ ต่อเนื่องกันได้
\item สัญลักษณ์ที่มีค่ามากกว่าจะอยู่ด้านซ้ายของสัญลักษณ์ที่มีค่าน้อยกว่าเสมอ เช่น $8$ จะเขียนแทนด้วย VIII (มีความหมายเท่ากับ $5+1+1+1$), $17$ จะเขียนแทนด้วย XVII (มีความหมายเท่ากับ $10+5+1+1$) และ $73$ จะเขียนแทนด้วย LXXIII (มีความหมายเท่ากับ $50+10+10+1+1+1$)
\end{enumerate}
\textbf{ข้อยกเว้นประการหนึ่ง} ของการแทนเลขโรมันที่มีค่าน้อยกว่า $400$ คือ ในการแทนค่า $4$ และ $9$ ในหลักหน่วย และการแทนค่า $40$ และ $90$ ในหลักสิบ จะวางสัญลักษณ์ที่มีค่าน้อยกว่าไว้ด้านซ้ายของสัญลักษณ์ที่มีค่ามากกว่า เช่น $4$ จะเขียนแทนด้วย IV (มีความหมายเท่ากับ $(-1)+5$), $9$ จะเขียนแทนด้วย IX (มีความหมายเท่ากับ $(-1)+10$), $40$ จะเขียนแทนด้วย XL (มีความหมายเท่ากับ $(-10)+50$) และ $90$ จะเขียนแทนด้วย XC (มีความหมายเท่ากับ $(-10) +100$) เป็นต้น ทำนองเดียวกันตามกฎนี้จะทำให้ $24$ $39$ $44$ $49$ $94$ เขียนแทนด้วยเลขโรมันได้เป็น XXIV XXXIX XLIV XLIX และ XCIV ตามลำดับ

\textbf{กำหนดให้} หนังสือเล่มหนึ่งมีจำนวนหน้าในบทนำทั้งหมด $d$ หน้า โดยที่ $1 \leq d < 400$ จงเขียนโปรแกรมเพื่อนับจำนวนสัญลักษณ์ ‘I’ ‘V’ ‘X’ ‘L’ และ ‘C’ ที่ใช้แทนหมายเลขหน้าในบทนำของหนังสือเล่มดังกล่าว ตัวอย่างเช่น ถ้าหนังสือมีจำนวนหน้าในบทนำ $5$ หน้า นั่นคือประกอบด้วยหน้าหมายเลข ‘I’ ‘II’ ‘III’ ‘IV’ และ ‘V’ ดังนั้น หนังสือเล่มนี้จะมีสัญลักษณ์ ‘I’ จำนวน $7$ ตัว, ‘V’ จำนวน $2$  ตัว, ‘X’ จำนวน $0$ ตัว, ‘L’ จำนวน $0$ ตัว และ ‘C’ จำนวน $0$ ตัว เป็นต้น


\underline{\textbf{โจทย์}} จงเขียนโปรแกรมเพื่อรับเลขจำนวนหน้าของบทนำ แล้วแสดงผลเป็นจำนวนของอักษรแต่ละตัวที่ใช้ทั้งหมดสำหรับการเขียนเลขโรมันแทนเลขหน้าในบทนำ


\InputFile
\textbf{มีบรรทัดเดียว} มีจำนวนเต็มบวก $d$ แทนจำนวนหน้าในบทนำของหนังสือ

\OutputFile

\textbf{มีบรรทัดเดียว} ประกอบด้วยจำนวนเต็มห้าตัวคั่นด้วยช่องว่างหนึ่งช่อง โดยตัวแรกถึงตัวที่ห้า จะแทนจำนวนของสัญลักษณ์โรมัน ‘I’ ‘V’ ‘X’ ‘L’ และ ‘C’ ตามลำดับ

\Examples

\begin{example}
\exmp{5}{7 2 0 0 0}%
\end{example}

\Source

การแข่งขันคอมพิวเตอร์โอลิมปิก สอวน. ครั้งที่ 1 มหาวิทยาลัยเกษตรศาสตร์

\end{problem}

\end{document}