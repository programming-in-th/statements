\documentclass[11pt,a4paper]{article}

\usepackage{../../templates/style}

\begin{document}

\begin{problem}{กิจกรรมยามว่าง (activity)}{standard input}{standard output}{1 second}{32 megabytes}

ณ ค่ายอบรมเข้มเพื่อการแข่งขันคอมพิวเตอร์โอลิมปิกระดับต๊อกต๋อย มีผู้สนใจเข้าร่วมมากมาย รวมถึงคนที่มีชื่อเสียงโด่งดังต่างต่างนานา เช่น คุณชายเล ผู้มีความสามารถด้านการเขียนโปรแกรมเป็นอย่างสูง เป็นต้น

อย่างไรก็ตาม การอบรมอย่างหนักย่อมทำให้เกิดความเครียดแก่ผู้คนในค่ายเป็นธรรมดา พวกเขาเหล่านั้นจึงหาเกมมาคลายเครียด และหนึ่งในเกมที่ได้รับความนิยมในกลุ่มพวกเขาเหล่านั้นคือ เกม Defense of The Ancient หรือรู้จักกันดีในนาม DOTA เกมนี้จะมีผู้เล่นเข้าร่วมแข่งขันไม่เกิน $25$ คนและแบ่งออกเป็น $2$ ฝ่าย คือ Sentinel และ Scourge

นายชายเลหลังจากเล่นเกมจนหายเครียดแล้ว ก็เกิดข้อสงสัยขึ้นว่า ถ้ามีผู้เล่น $N$ คน เข้าร่วมเกม DOTA จะมีวิธีแบ่งฝ่ายให้ผู้เล่นทั้งหมดกี่วิธี โดยมีเงื่อนไขว่า จำนวนผู้เล่นทั้งสองฝ่ายมีค่าต่างกันไม่เกิน $1$ (ผู้เล่นแต่ละคนมีความแตกต่างกัน และ ฝ่ายทั้งสองก็แตกต่างกันด้วย)

\underline{\textbf{โจทย์}} จงเขียนโปรแกรมที่รับจำนวนผู้เล่นเกม DOTA และคำนวณวิธีในการแบ่งฝ่ายทั้งหมดที่เป็นไปได้ โดยมีเงื่อนไขว่า จำนวนผู้เล่นทั้งสองฝ่ายมีค่าต่างกันไม่เกิน $1$

\InputFile

\textbf{มีบรรทัดเดียว} มีตัวเลขจำนวนเต็ม $N$ $(1 \leq N \leq 25)$ แทนจำนวนผู้เข้าเล่นเกมในรอบหนึ่งๆ

\OutputFile

\textbf{มีบรรทัดเดียว} จำนวนเต็มหนึ่งจำนวนแทนจำนวนวิธีการแบ่งฝ่ายทั้งหมดที่เป็นไปได้ โดยมีเงื่อนไขว่า จำนวนผู้เล่นทั้งสองฝ่ายมีค่าต่างกันไม่เกิน $1$


\Examples

\begin{example}
\exmp{1}{2}%
\exmp{2}{2}%
\exmp{4}{6}%
\end{example}

\Source

โจทย์โดย: วรภัทร จรางกุล

การแข่งขัน IOI Thailand League เดือนสิงหาคม 2553

\end{problem}

\end{document}