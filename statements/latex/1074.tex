\documentclass[11pt,a4paper]{article}

\usepackage{../../templates/style}

\begin{document}

\begin{problem}{Mravojed}{standard input}{standard output}{1 second}{32 megabytes}

นักโบราณคดีได้ค้นพบสิ่งที่หลงเหลืออยู่ของสถาปัตยกรรม เกรโก-โรมัน สถานที่ทั่วไปสามารถจำลองได้เป็นตารางช่องจัตุรัสขนาด $R \times C$ ในแต่ละช่องนั้น นักโบราณคดีได้สำรวจว่าเคยมีสิ่งก่อสร้างในอดีตสร้างบนพื้นที่ช่องนี้แล้ว หรือว่าช่องนี้ยังไม่เคยถูกสร้างทับเลย

หลังจากการพินิจพิจารณา พวกเขาก็ได้ข้อสรุปว่าบนพื้นที่แห่งนี้เคยเป็นที่ตั้งของ\textbf{สองอาคาร} ซึ่งอยู่ในช่วงเวลาที่แตกต่างกัน และแต่ละอาคารนั้นจะมีโครงสร้างฐานเป็นรูป\textbf{สี่เหลี่ยมจัตุรัส}

เนื่องจากทั้งสองอาคารนั้นอยู่ในช่วงเวลาที่แตกต่างกัน จึงเป็นไปได้ว่าทั้งสองอาคารอาจมีฐานที่เหลื่อมล้ำกัน

\bigskip
\underline{\textbf{โจทย์}}  จงเขียนโปรแกรมเพื่อรับข้อมูลการสำรวจ และ บอกที่ตั้งและขนาด(ความยาวของช่องที่รองรับฐานของอาคาร)ของอาคารแต่ละตัว

\InputFile

\textbf{บรรทัดแรก} เป็นจำนวนเต็ม $R$ และ $C$ แทนขนาดของพื้นที่สำรวจ โดยที่ $1 \leq R \leq 100$ และ $1 \leq C \leq 100$ 

\textbf{บรรทัดที่ $2$ ถึง $R+1$} บรรทัดที่ $i+1$ แทนข้อมูลแถวที่ $i$ ของพื้นที่ โดยมีสายอักขระความยาว $C$ ที่ประกอบด้วย ‘.’ (จุด) หรือ ‘x’ (ตัวเล็ก) โดยที่ ‘.’ แทนว่าช่องดังกล่าวไม่เคยมีอาคารถูกสร้าง ในขณะที่ ‘x’ หมายถึงช่องดังกล่าวเคยมีอาคารถูกสร้าง


\OutputFile

\textbf{มีสองบรรทัด} แต่ละบรรทัด แสดงแถวและหลักของมุมบนซ้ายของอาคารและขนาดของอาคาร

\textbf{หมายเหตุ:} รับประกันว่าจะมีคำตอบเสมอ และคำตอบอาจเป็นไปได้หลายรูปแบบ ให้เลือกตอบมาแค่ $1$ คำตอบ

\Examples

\begin{example}
\exmp{3 3
xx.
xxx
...}{1 1 2
2 3 1}%
\exmp{4 6
xx....
xx.xxx
...xxx
...xxx}{1 1 2
2 4 3}%
\exmp{5 5
.....
xxx..
xxxx.
xxxx.
.xxx.}{2 1 3
3 2 3}%
\end{example}


\Source

COCI 2008/2009, Contest \#1 – October 18, 2008


\end{problem}

\end{document}