\documentclass[11pt,a4paper]{article}

\usepackage{../../templates/style}

\begin{document}

\begin{problem}{DNA}{standard input}{standard output}{1 second}{64 megabytes}

\textbf{ดีเอ็นเอ \textit{(Deoxyribonucleic acid)}} คือ สารพันธุกรรมที่สามารถนำมาใช้เป็นเอกลักษณ์ของสิ่งมีชีวิต โดยดีเอ็นเอนี้เมื่อผ่านการถอดความหมายมาจากสารพันธุกรรมใดๆแล้วจะถูกนำมาเก็บในลักษณะของสายอักขระ โดยตัวอักขระนั้นจะมีเพียงแค่สี่ตัวเท่านั้น นั่นคือ \textbf{A C G} และ \textbf{T}

ในการวิเคราะห์ดีเอ็นเอของคุณหญิงหมอ มักประสบปัญหาที่ข้อมูลที่ต้องการเปรียบเทียบแม้ว่าสิ่งมีชีวิตชนิดเดียวกันแต่ตำแหน่งเริ่มต้นไม่ตรงกันทำให้ยากที่จะเปรียบเทียบกันได้ ตัวอย่างเช่น สารชนิดแรกได้ผลมาเป็น \\\textbf{AAAACTGCTACCGGT} และชิ้นที่สองคือ \textbf{CTGAATCTACTGCTATTGCAA} หากสังเกตให้ดีจะเห็นได้ว่าส่วนที่เหมือนกันที่มีความยาวต่อเนื่องมากที่สุดคือ \textbf{ACTGCTA}

\bigskip
\underline{\textbf{โจทย์}} จงเขียนโปรแกรมเพื่อช่วยคุณหญิงหมอหาส่วนที่เหมือนกันอย่างต่อเนื่องของดีเอ็นเอจำนวน $2$ สาย โดยที่ข้อมูลที่ซ้ำและยาวที่สุดอาจมีได้หลายชุด

\InputFile

\textbf{บรรทัดแรก} รับสตริงแสดงข้อมูลดีเอ็นเอสายที่ 1

\textbf{บรรทัดที่สอง} รับสตริงแสดงข้อมูลดีเอ็นเอสายที่ 2

สตริงที่งสองบรรทัดจะประกอบด้วยอักขระ ‘A’ ‘C’ ‘G’ ‘T’ เท่านั้น โดยความยาวของแต่ละสายอยู่ในช่วง\\ $1 \leq L \leq 200$ เมื่อ $L$ คือความยาวของสตริงที่รับเข้ามา

\OutputFile
\textbf{มีบรรทัดเดียว} แสดงส่วนที่เหมือนกันที่ยาวที่สุดของดีเอ็นเอทั้งสองสาย ถ้ามีส่วนที่ยาวที่สุดมากกว่าหนึ่งคำตอบให้เลือกตอบส่วนที่ยาวที่สุดสายแรกที่พบ

("สายแรก" หมายถึง สายย่อยที่อยู่ซ้ายที่สุดของดีเอ็นเอสายแรก)

\Examples

\begin{example}
\exmp{AAAACTGCTACCGGT
CTGAATCTACTGCTATTGCAA}{ACTGCTA}%
\end{example}


\Source

การแข่งขันคณิตศาสตร์ วิทยาศาสตร์ โอลิมปิกแห่งประเทศไทย สาขาวิชาคอมพิวเตอร์ ประจำปี 2547

\end{problem}

\end{document}