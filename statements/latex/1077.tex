\documentclass[11pt,a4paper]{article}

\usepackage{../../templates/style}

\begin{document}

\begin{problem}{กุญแจ (key)}{standard input}{standard output}{1 second}{128 megabytes}

  และแล้วคำแนะนำที่ดีเยี่ยมก็โผล่มาดั่งอัศวินขี่ม้าขาว นักเลงคอมพิวเตอร์นิรนามผู้หนึ่งได้ช่วยให้คุณเจาะเข้าไปถึงโครงสร้างข้อมูลซึ่งมีลักษณะเป็นตาราง คุณทราบจากนักเลงคอมพิวเตอร์นิรนามว่ากุญแจสุดท้ายที่จะไขเข้าไปสู่ระบบฐานข้อมูลของ \textit{TOI.C} อยู่ในกระจายอยู่ในตารางนี้ นั่นคือรหัสซึ่งมีทั้งหมด $N$ ตัว กระจายอยู่ตามแต่ละช่องในตารางนี้

            ถึงเวลาที่จะต้องไขพาสเวิร์ดให้ได้ ตารางข้อมูลนี้มีรูปเป็นสี่เหลี่ยมจัตุรัสขนาด $1\,001 \times 1\,001$ หน่วย มุมล่างซ้ายของตารางอยู่ที่ช่อง $(0,0)$ และมุมขวาบนของตารางอยู่ที่ช่อง $(1\,000$, $1\,000)$ ในระนาบ $2$ มิติ คุณไม่สามารถท่องเข้าไปในตารางข้อมูลนี้ได้ เนื่องจากการระบบการป้องกันภัยขั้นสูง

            สิ่งที่คุณทำได้คือการเจาะไปยังช่องใดช่องหนึ่งในตารางตำแหน่ง $(X, Y)$ แล้วกระจายตัวเองออกไปรอบทิศด้วยพลังงาน $K$ คุณจะได้รหัสพาสเวิร์ดทุกตัว ที่อยู่ภายในรูปสี่เหลี่ยมจัตุรัสที่มีจุด $(X – K, Y – K)$ เป็นมุมล่างซ้าย และจุด $(X + K, Y + K)$ เป็นมุมบนขวา ทั้งนี้เป็นไปได้ที่จะมีการแกะรหัสพาสเวิร์ดตัวเดิมเกิดขึ้นหลายครั้ง

เคราะห์ร้ายที่คุณต้องเหนื่อยอีกครั้ง เมื่อพบว่าคุณสามารถเจาะตารางนี้ได้เพียง $M$ ครั้งเท่านั้น

ครั้งนี้ สิ่งที่คุณต้องทำคือทราบให้ได้ว่าการเจาะเข้าไปยังตำแหน่งใดในตารางด้วยพลังงานเท่าไหร่ จะทำให้สามารถแกะรหัสมาได้กี่ตัว

\bigskip
\underline{\textbf{โจทย์}}  เขียนโปรแกรมรับตำแหน่งของรหัสแต่ละตัว และตำแหน่งในการเจาะตาราง แล้วคำนวณว่า การทดลองเจาะตารางแต่ละครั้งแกะรหัสได้ทั้งสิ้นกี่ตัว

\InputFile

\textbf{บรรทัดแรก} ระบุจำนวนเต็ม $N$ $(1 \leq N \leq 1\,000\,000)$ แทนจำนวนตัวของรหัส และจำนวนเต็ม $M$ $(1 \leq M \leq 1\,000\,000)$ แทนจำนวนครั้งของการเจาะ

\textbf{บรรทัดที่ $2$ ถึง $N+1$}  มีข้อมูลของรหัสทั้ง $N$ ตัว โดยในบรรทัดที่ $i + 1$ ระบุจำนวนเต็ม $X_i$ และ $Y_i$ $(0 \leq X_i, Y_i \leq 1\,000)$ ซึ่งเป็นตำแหน่งช่องที่รหัสนั้นอยู่ในตาราง ทั้งนี้อาจมีรหัสสองตัวใดๆ อยู่ในตำแหน่งเดียวกันได้

\textbf{บรรทัดที่ $N+2$ ถึง $N+M+1$} มีข้อมูลการเจาะตาราง โดยในบรรทัดที่ $N + j + 1$ มีจำนวนเต็ม $X_j$ และ $Y_j$ และ $K_j$ $(0 \leq X_j, Y_j \leq 1\,000; 0 \leq K_j \leq 1\,000)$ หมายความว่าในการเจาะตารางครั้งที่ $j$ มีการเจาะที่ตำแหน่ง $(X_j, Y_j)$ ด้วยพลังงาน $K_j$ เนื่องจากคุณง่วงและเบลอ เป็นไปได้ที่คุณจะเจาะตารางซ้ำที่เดิมด้วยพลังงานเดิม

\end{itemize}

\OutputFile

\textbf{มี $M$ บรรทัด} ในบรรทัดที่ $j$ แสดงจำนวนเต็ม $B_j$ แทนจำนวนรหัสที่ทราบมาจากการเจาะตารางครั้งที่ $j$


\Examples

\begin{example}
\exmp{5 2
0 0
0 10
10 0
10 10
5 5
5 5 5
10 10 5}{5
2}%
\exmp{5 2
0 0
2 0
1 1
3 0
6 6
2 1 2
6 6 5}{4
2}%
\end{example}


\Scoring

\textbf{50\% ของชุดข้อมูลทดสอบ:} $N, M \leq 10\,000$ 

\textbf{100\% ของชุดข้อมูลทดสอบ:} $ N, M \leq 1\,000\,000$

\Source

พศิน มนูรังษี

\underline{\href{http://thailandoi.org/toi.c/01-2009}{TOI.C:01-2009}}


\end{problem}

\end{document}