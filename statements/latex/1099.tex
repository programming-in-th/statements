\documentclass[11pt,a4paper]{article}

\usepackage{../../templates/style}

\begin{document}

\begin{problem}{JABUKA}{standard input}{standard output}{1 second}{32 megabytes}

Mirko มีแอปเปิ้ลแดง ($R$) และแอปเปิ้ลเขียว ($G$) เพื่อแบ่งปันให้กับเพื่อน ๆของเขา โดยเขาจะแบ่งแอปเปิ้ลแดงในจำนวนที่เท่ากันและแอปเปิ้ลเขียวในจำนวนที่เท่ากันให้กับเพื่อนทุก ๆคน Mirko ไม่ชอบแอปเปิ้ล ดังนั้นเขาจึงไม่ต้องการที่จะเหลือแอปเปิ้ลไว้ให้กับตัวเขาเองหลังจากแบ่งเสร็จ
ยกตัวอย่างเช่น ถ้า Mirko มีแอปเปิ้ลแดง $4$ ลูกและแอปเปิ้ลเขียว $8$ ลูก เขาจะแบ่งแอปเปิ้ลเหล่านี้ให้กับเพื่อน ๆของเขาได้ทั้งหมด $3$ วิธี ดังนี้
\begin{enumerate}

\item แบ่งให้กับเพื่อนเพียงคนเดียว ซึ่งเพื่อนคนนั้นจะได้รับแอปเปิ้ลแดงทั้ง $4$ ลูกและแอปเปิ้ลเขียวทั้ง $8$ ลูก
\item แบ่งให้กับเพื่อน $2$ คน ซึ่งแต่ละคนจะได้รับแอปเปิ้ลแดง $2$ ลูกและแอปเปิ้ลเขียว $4$ ลูก
\item แบ่งให้กับเพื่อน $4$ คน ซึ่งแต่ละคนจะได้รับแอปเปิ้ลแดง $1$ ลูกและแอปเปิ้ลเขียว $2$ ลูก
 \end{enumerate}   

\bigskip
\underline{\textbf{โจทย์}}  จงเขียนโปรแกรมเพื่อแสดงผลจำนวนวิธีทั้งหมดที่ Mirko สามารถแบ่งแอปเปิ้ลให้กับเพื่อน ๆของเขาได้ โดยสมมติว่า Mirko มีเพื่อนจำนวนมากมายมหาศาลในการแบ่งปันแอปเปิ้ลให้


\InputFile

\textbf{มีบรรทัดเดียว} ประกอบด้วยเลขจำนวนเต็ม $2$ ค่าคือ จำนวนของแอปเปิ้ลแดง ($R$) และจำนวนของแอปเปิ้ลเขียว ($G$) ซึ่งคั่นกันด้วยช่องว่าง โดยมีค่าดังนี้ $1 \leq R, G \leq 1\,000\,000\,000$


\OutputFile


\textbf{มีหลายบรรทัด} ในแต่ละครั้งของการแบ่งแอปเปิ้ลที่เป็นไปได้ ให้แสดงผลเลขจำนวนเต็ม $3$ ค่าคือ $N, X$ และ $Y$ บนบรรทัดเดียวกัน โดย $N$ คือจำนวนของเพื่อนที่ได้รับการแบ่งแอปเปิ้ล ส่วน $X$ และ $Y$ คือจำนวนแอปเปิ้ลแดงและแอปเปิ้ลเขียวที่เพื่อนแต่ละคนได้รับ

ให้แสดงผลของการแบ่งแอปเปิ้ลแบบเดียวกันเพียงแค่ครั้งเดียว โดยจะต้องเรียงลำดับข้อมูลส่งออกดังนี้
\begin{itemize}

\item เรียงตามค่า $N$ จากน้อยไปมาก
\item หากค่า $N$ เท่ากัน ให้เรียงตามค่า $X$ จากน้อยไปมาก
\item หากค่า $N$ และ $X$ เท่ากัน ให้เรียงตามค่า $Y$ จากน้อยไปมาก

\end{itemize}
\Examples

\begin{example}
\exmp{4 8}{1 4 8
2 2 4
4 1 2}%
\exmp{15 12}{1 15 12
3 5 4}%
\exmp{42 105}{1 42 105
3 14 35
7 6 15
21 2 5}%
\end{example}

\Note 

*** ถ้าสังเกตดีๆจะพบว่าไม่ทางที่ $X$ หรือ $Y$ จะเท่ากันได้ lol ***

\Source

COCI 2008/2009, Contest \#5 – February 7, 2009 :: ดัดแปลงเล็กน้อย (:

\end{problem}

\end{document}