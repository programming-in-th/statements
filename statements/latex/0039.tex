\documentclass[11pt,a4paper]{article}

\usepackage{../../templates/style}

\begin{document}

\begin{problem}{อาหารโอชา (Food)}{standard input}{standard output}{1 second}{16 megabytes}

ในพระราชวังแห่งหนึ่ง พ่อครัวสามารถทำอาหารได้ $n$ ชนิด\textbf{ที่แตกต่างกัน} อาหารเหล่านี้อยู่ในเซต $F = \{f_1,f_2,f_3,…,f_n\}$ ซึ่งในการถวายอาหารแก่เจ้าชายซึ่งกินจุมาก พ่อครัวจะถวายอาหารหนึ่งชนิดต่อหนึ่งชั่วโมง และจะถวายจนครบ $n$ ชนิด นอกจากนั้นในการถวายอาหารนี้ พ่อครัวมีกลุ่มของอาหารที่\textbf{ต้องห้ามไม่ให้ถวายเป็นลำดับแรก $m$ ชนิด} โดยกำหนดให้อาหารต้องห้ามทั้งหมดอยู่ในเซต $P \subset F$

\underline{\textbf{โจทย์}} จงเขียนโปรแกรมแสดงลำดับทั้งหมดของการถวายอาหาร $n$ ชนิดที่เป็นไปได้

\InputFile
 \textbf{บรรทัดแรก} เป็นจำนวนชนิดอาหาร $n$ โดยที่ $2 \leq n \leq 8$

\textbf{บรรทัดที่สอง} เป็นจำนวนชนิดอาหารต้องห้าม $m$ โดยที่ $2 \leq m \leq n$

\textbf{บรรทัดที่สาม} แสดงชนิดของอาหารต้องห้ามที่ไม่ให้ถวายเป็นลำดับแรก โดยแสดงเป็นตัวเลขจำนวนเต็มบวก $m$ ตัว \textbf{โดยมีช่องว่างคั่นอยู่ระหว่างตัวเลข}

\OutputFile

\textbf{มีหลายบรรทัด} แสดงลำดับที่เป็นไปได้ทั้งหมด โดยใช้หนึ่งบรรทัดในการแสดงลำดับของอาหารหนึ่งลำดับ ลำดับของอาหารจะเป็นตัวเลขจำนวนเต็มบวกระหว่าง $1$ ถึง $n$ ที่มีช่องว่างคั่นอยู่ระหว่างตัวเลข \textbf{สำหรับการเรียงก่อนหลังของลำดับของอาหาร ให้เรียงตามลำดับ Dictionary จากน้อยไปมาก} ซึ่งจะเปรียบเทียบกันโดยใช้เลขหลักซ้ายสุดที่ไม่เหมือนกัน

\Examples

\begin{example}
\exmp{4
3
1 2 3}{4 1 2 3
4 1 3 2
4 2 1 3
4 2 3 1
4 3 1 2
4 3 2 1}%
\exmp{4
2
3 2}{1 2 3 4
1 2 4 3
1 3 2 4
1 3 4 2
1 4 2 3
1 4 3 2
4 1 2 3
4 1 3 2
4 2 1 3
4 2 3 1
4 3 1 2
4 3 2 1}%
\end{example}


\Note 
\begin{enumerate}

\item แนะนำให้ใช้ printf ในการแสดงผล
\item ข้อมูลส่งออกที่ได้จากชุดทดสอบมีขนาดไม่เกิน 1 MB \end{enumerate}

\Source

การแข่งขันคอมพิวเตอร์โอลิมปิกระดับชาติครั้งที่ 7 (NUTOI7) :: ดัดแปลงเล็กน้อย

\end{problem}

\end{document}