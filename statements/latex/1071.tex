\documentclass[11pt,a4paper]{article}

\usepackage{../../templates/style}

\begin{document}

\begin{problem}{ระเบิด (bomb)}{standard input}{standard output}{1 second}{32 megabytes}

หลังจากที่คุณทราบว่ากองกิ่งไม้ที่จะเผาบูชาไปนั้นทำให้เทพเจ้าทรงพระกริ้วจนถึงขีดสุดแล้ว คุณและนักโบราณคดีจึงพยายามหลบหนีอย่างไม่คิดชีวิต แต่ทว่าเทพเจ้าแห่งตัวเลขตื่นขึ้นแล้ว และต้องการลงโทษคุณและนักโบราณคดีโดยการทิ้งระเบิดลงมายังโบราณสถานมาซู่

ด้วยความตกใจ คุณและผู้ร่วมคณะสำรวจจึงวิ่งออกมายังห้องโถงกว้างแห่งหนึ่ง ห้องโถงกว้างนี้มีพื้นที่เป็นสี่เหลี่ยมจัตุรัส กว้าง $1\,000$ หน่วยและยาว $1\,000$ หน่วย มุมล่างซ้ายของห้องอยู่ที่จุด $(0, 0)$ และมุมบนขวาอยู่ที่จุด $(1\,000 , 1\,000)$ ในระนาบสองมิติ

    เทพเจ้าแห่งตัวเลขจะทิ้งระเบิดลงมาทีละลูก ไปยังตำแหน่งบนพื้นห้องที่มีพิกัดตามแกน $x$ และแกน $y$ เป็นจำนวนเต็ม เมื่อมันตกมาถึงพื้นห้องมันจะทำลายล้างทุกสิ่งที่อยู่รอบมันเป็นบริเวณสี่เหลี่ยมจัตุรัส ซึ่งมีความยาวด้านตามพลังทำลายล้างของระเบิด กล่าวคือหากลูกระเบิดซึ่งมีพลังทำลายล้าง $R$ ตกถึงพื้นที่ตำแหน่ง $(X, Y)$ มันจะทำลายล้างทุกสิ่งที่อยู่ในสี่เหลี่ยมจัตุรัสที่มีจุด $(X - R, Y - R)$ เป็นมุมล่างซ้ายและจุด $(X + R, Y + R)$ เป็นมุมบนขวา
    
    เคราะห์ดีที่คุณพอรู้คุณไสยมาบ้าง ก่อนนักโบราณคดีจะทำพิธีเผากิ่งไม้ คุณได้แปะยันต์ไว้ ณ ตำแหน่งต่างๆ บนพื้นห้องจำนวน $N$ แผ่น โดยตำแหน่งที่คุณแปะจะมีพิกัดแกน $x$ และแกน $y$ เป็น จำนวนเต็มเสมอ ถ้ายันต์อยู่ในเขตการทำลายล้างของระเบิด มันจะถูกทำลาย (เพราะระเบิดจะทำลายล้างทุกสิ่ง) แต่ถ้าคุณไปยืนอยู่ ณ ตำแหน่งที่มียันต์แปะอยู่ และตำแหน่งนั้นไม่อยู่ในรัศมีทำลายล้างของระเบิด คุณจะปลอดภัยจากสะเก็ดระเบิดและภยันตรายอื่นๆ อีกมากมาย
    
    เพื่อทำให้โอกาสอยู่รอดของคณะสำรวจมีค่ามากที่สุด คุณต้องการทราบว่าระเบิดที่ตกลงมาแต่ละลูกจะทำลายยันต์ไปกี่แผ่น คุณมีเวลาไม่มาก เพราะตอนนี้ระเบิดกำลังร่วงลงมาแล้ว ช้าๆ

\bigskip
\underline{\textbf{โจทย์}}  เขียนโปรแกรมรับตำแหน่งของยันต์ และตำแหน่งที่ระเบิดถูกทิ้งลงมาตามลำดับก่อนหลัง แล้วคำนวณว่า ระเบิดแต่ละลูกจะทำลายยันต์กี่แผ่น

\InputFile

\textbf{บรรทัดแรก} ระบุจำนวนเต็ม $N$ $(1 \leq N \leq 100\,000)$ แทนจำนวนของยันต์ และจำนวนเต็ม $M$ $(1 \leq M \leq 100\,000)$ แทนจำนวนของระเบิด

\textbf{บรรทัดที่ $2$ ถึง $N+1$} มีข้อมูลของยันต์ $N$ แผ่น แต่ละบรรทัดระบุจำนวนเต็ม $x$ และ $y$ $(0 \leq x, y \leq 1\,000)$ หมายความว่ายันต์แผ่นหนึ่งถูกแปะอยู่ที่จุด $(x, y)$ เรารับประกันว่า ณ ตำแหน่งเดียวกันจะไม่มียันต์แปะอยู่มากกว่าหนึ่งแผ่น

\newpage
\textbf{บรรทัดที่ $N+2$ ถึง $N+M+1$} มีข้อมูลของระเบิด $M$ ลูกตามลำดับที่เทพเจ้าทิ้งลงมา แต่ละบรรทัดมีจำนวนเต็ม $X$ และ $Y$ และ $R$ $(0 \leq X, Y \leq 1\,000; 5 \leq R \leq 15)$ หมายความว่ามีระเบิดลูกหนึ่งถูกทิ้งลงมาที่จุด $(X, Y)$ และระเบิดนั้นมีพลังทำลายล้าง $R$ (เทพเจ้าสามารถทิ้งระเบิดมากกว่าหนึ่งลูกลงที่จุดเดียวกันได้)


\OutputFile

\textbf{มี $M$ บรรทัด }ในบรรทัดที่ $i$ แสดงจำนวนเต็ม $B_i$ แทนจำนวนยันต์ที่ระเบิดลูกที่ $i$ ทำลาย

\Examples

\begin{example}
\exmp{5 2
0 0
0 10
10 0
10 10
5 5
5 5 5
10 10 5}{5
0}%
\exmp{5 2
0 0
2 0
1 1
3 0
6 6
0 0 5
1 1 10}{4
1}%
\end{example}

\Scoring 

\textbf{50\% ของชุดข้อมูลทดสอบ:}  $N \leq 1\,000; M ≤ 100\,000$ 

\textbf{100\%ของชุดข้อมูลทดสอบ:} $N \leq 100\,000 ;M \leq 100\,000$

\Source

การแข่งขัน YTOPC Challenge เมษายน 2552


\end{problem}

\end{document}