\documentclass[11pt,a4paper]{article}

\usepackage{../../templates/style}

\begin{document}

\begin{problem}{จัดลำดับการทดลอง (schedule)}{standard input}{standard output}{1 second}{32 megabytes}

นายเมธาต้องการทำการทดลองทางวิทยาศาสตร์อยู่สองงาน โดยที่แต่ละงานประกอบด้วยขั้นตอนทั้งหมด $N$ ขั้นตอน คือขั้นตอน  $J_1,J_2,J_3,..,J_N$ สำหรับงานแรก และขั้นตอน $K_1,K_2,K_3,...,K_N$ สำหรับงานที่สอง ซึ่งแต่ละขั้นตอนอาจใช้เวลาเท่ากันหรือต่างกันก็ได้ อย่างไรก็ตามขั้นตอนในงานเดียวกันไม่สามารถสลับลำดับกันได้ กล่าวคือ สำหรับงานแรก ขั้นตอน $J_1$ จะต้องถูกทำเป็นอันดับแรก และขั้นตอน $J_2,J_3,...,J_N$ จะถูกทำต่อมาตามลำดับดังกล่าว สำหรับงานที่สองก็เช่นกัน ขั้นตอน $K_1$ จะต้องถูกทำเป็นอันดับแรก และขั้นตอน $K_2,K_3,...,K_N$ จะถูกทำตามลำดับ

แม้จะไม่สามารถสลับลำดับขั้นตอนในงานเดียวกันได้ แต่เมธาก็สามารถสลับลำดับขั้นตอนระหว่างงานแรกกับงานที่สองได้ เป็นต้นว่าถ้า $N=3$ เมธาสามารถที่จะทำการทดลองในลำดับ $K_1,K_2,J_1,K_3,J_2,J_3$ เพราะลักษณะนี้เป็นการทำการทดลองแต่ละงานตามลำดับจากขั้นตอนแรกไปขั้นตอนสุดท้าย

โชคไม่ดีนัก เมธาพบว่าห้องปฏิบัติการมีเครื่องมือสำหรับทำการทดลองอยู่เพียงชุดเดียว และงานทั้งสองก็ต้องใช้เครื่องมือชุดเดียวกันนี้  ยิ่งไปกว่านั้นเครื่องมือสามารถทำงานได้เพียง $M$ นาทีในแต่ละวัน และการทดลองแต่ละขั้นตอนก็ต้องดำเนินการอย่างต่อเนื่องให้สำเร็จภายในวันเดียวเท่านั้น

\textbf{ยกตัวอย่างเช่น} หากงานแต่ละงานมีสองขั้นตอน $N=2$ และใช้เครื่องได้ $300$ นาทีต่อวัน $(M=300)$ เมื่อ $J_1=200, J_2=150, K_1=50$ และ $K_2=150$ ถ้าหากเมธาจัดลำดับการทดลองเป็น $J_1,J_2,K_1,K_2$ ตามลำดับ ขั้นตอน $J_2$ จะไม่สามารถทำได้ในวันแรกเพราะเวลารวมในวันแรกจะเกิน $300$ นาที ทำให้ต้องเลื่อนไปทำในวันที่สอง และการทดลองตามลำดับนี้ จะใช้เวลาทั้งหมด $3$ วัน โดยวันสุดท้าย (วันที่สาม) จะใช้เวลาทั้งหมด $150$ นาที แต่หากเมธาจัดลำดับการทดลองใหม่เป็น $J_1,K_1,K_2,J_2$ การทดลองทั้งหมดจะแล้วเสร็จในเวลาเพียง $2$ วัน โดยวันสุดท้าย (วันที่สอง) จะใช้เวลาทั้งหมด $300$ นาที


\bigskip
\underline{\textbf{โจทย์}}  จงเขียนโปรแกรมที่มีประสิทธิภาพในการจัดลำดับขั้นตอนการทดลองที่ทำให้การทดลองทั้งสองงานเสร็จด้วยเวลาที่น้อยที่สุด


\InputFile

\textbf{บรรทัดแรก} รับเลขจำนวนเต็ม $M$ ระบุเวลาที่สามารถใช้เครื่องมือได้ในแต่ละวัน
โดยที่ $1 \leq M \leq 600$ และ $M$ มีหน่วยเป็นนาที

\textbf{บรรทัดที่สอง} รับจำนวนเต็ม $N$ ระบุจำนวนขั้นตอนในแต่ละงานโดยที่ $2 \leq N \leq 1\,000$

\textbf{บรรทัดที่สาม} รับจำนวนเต็มบวก $N$ จำนวน คือ $a_1,a_2,a_3,...,a_n$ แต่ละตัวคั่นด้วยช่องว่าง จำนวนแต่ละจำนวนนี้แทนเวลาที่ต้องใช้ทำการทดลองขั้นตอน $J_1,J_2,J_3,...,J_N$ ของงานแรกตามลำดับ มีหน่วยเป็นนาที จำนวนแต่ละจำนวนถูกคั่นด้วยช่องว่าง โดยที่ $1 \leq a_i \leq M; i = 1,...,N$

\textbf{บรรทัดที่สี่} รับจำนวนเต็มบวก $N$ จำนวนในลักษณะเดียวกับบรรทัดที่สาม  คือ $b_1,b_2,b_3,...,b_n$ แต่จำนวนเหล่านี้แทนเวลาที่ต้องใช้ในการทดลองขั้นตอน $K_1,K_2,K_3,...,K_N$ สำหรับงานที่สอง ซึ่ง $1 \leq b_i \leq M; i = 1,...,N$


\OutputFile

\textbf{มีสองบรรทัด} ระบุจำนวนวันที่ต้องใช้ในการทดลองของเมธา และจำนวนนาทีที่ใช้ในการทดลองวันสุดท้าย โดยข้อมูลส่งออกต้องอยู่ในรูปแบบดังต่อไปนี้

\begin{enumerate}

\item \textbf{บรรทัดแรก} ระบุจำนวนวันที่ต้องใช้ในการทดลองเป็นจำนวนเต็ม
\item \textbf{บรรทัดที่สอง} ระบุจำนวนนาทีที่ใช้สำหรับการทดลองในวันสุดท้าย โดยที่จำนวนนาทีนี้มีค่าตั้งแต่หนึ่งและไม่เกิน $M$
\end{enumerate}

\textbf{หมายเหตุ:}  เวลาในการทดลองที่ดีที่สุดถือตามจำนวนวันเป็นลำดับแรก ในกรณีที่การจัดลำดับขั้นตอนสองแบบใช้จำนวนวันเท่ากัน จะนับเวลาที่ดีที่สุดจากจำนวนนาทีที่ใช้ในวันสุดท้าย

\Examples

\begin{example}
\exmp{8
4
4 5 6 4
3 3 2 4}{4
8}%
\exmp{8
6
2 3 4 5 3 2
6 2 3 2 4 5}{6
5}%
\exmp{10
12
1 7 5 4 3 6 2 3 4 5 1 8
3 4 4 8 3 9 1 7 3 2 4 5}{11
8}%
\end{example}


\Source

การแข่งขันคอมพิวเตอร์โอลิมปิกระดับชาติครั้งที่ 8 (SUTOI8)

\end{problem}

\end{document}