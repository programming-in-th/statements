\documentclass[11pt,a4paper]{article}

\usepackage{../../templates/style}

\begin{document}

\begin{problem}{คู่ตัวเลขเด่น (Pair)}{standard input}{standard output}{2 second}{32 megabytes}

ให้ชุดของคู่อันดับจำนวนเต็มบวกมา $n$ ชุด คือ $(a_1,b_1), (a_2,b_2), ..., (a_n,b_n)$ โดยที่ $a_i \neq a_j$ ถ้า $i \neq j$ และ $b_k \neq b_l$ ถ้า $k \neq l$ โดยกำหนดว่า $1 \leq a_i \leq 100\,000$  และ $1 \leq b_j \leq n$ เราเรียกคู่อันดับ $2$ คู่ $(a_i,b_i)$ และ $(a_j,b_j)$ ว่าคู่ตัวเลขเด่นก็ต่อเมื่อ $a_i > a_j$ และ $b_i < b_j$



\bigskip
\underline{\textbf{โจทย์}}  จงเขียนโปรแกรมที่มีประสิทธิภาพในการหาค่าผลรวมของ $a_i + a_j$ ทั้งหมด เมื่อ คู่ $(a_i,b_i)$ และ $(a_j,b_j)$ เป็นคู่ตัวเลขเด่น


\InputFile

\textbf{บรรทัดแรก} รับค่าของ $n$ โดยที่ $2 \leq n \leq 100\,000$

\textbf{บรรทัดที่สอง} รับค่าของคู่ตัวเลข $a_i$ และ $b_i$ จำนวน $n$ คู่ โดยจะเรียงจากคู่ที่หนึ่งไปจนกระทั่งถึงคู่ที่ $n$ โดยมีตัวเลขทั้งหมด $2n$  ตัว และมีช่องว่างคั่นอยู่ระหว่างตัวเลขเหล่านี้

\OutputFile

\textbf{มีบรรทัดเดียว} เป็นตัวเลขจำนวนเต็มบวกหนึ่งค่า ซึ่งแสดงถึงผลรวมของ $a_i + a_j$ ทั้งหมด เมื่อ คู่ $(a_i,b_i)$ และ $(a_j,b_j)$ เป็นคู่ตัวเลขเด่น

\bigskip
\textbf{หมายเหตุ}
\begin{enumerate}

\item แนะนำให้ใช้ scanf ในการรับค่าและ printf ในการแสดงผล
\item แนะนำให้ใช้ตัวแปรชนิด double ในการเก็บค่าผลรวม และแสดงผลโดยใช้รูปแบบ “\%.0lf”
\end{enumerate}

\Examples

\begin{example}
\exmp{6
2 1 7 6 9 3 18 4 3 2 5 5}{78}%
\exmp{4
1 4 3 2 2 3 7 1}{39}%
\end{example}

  
\Source

การแข่งขันคอมพิวเตอร์โอลิมปิกระดับชาติครั้งที่ 7 (NUTOI7)

\end{problem}

\end{document}