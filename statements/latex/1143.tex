\documentclass[11pt,a4paper]{article}

\usepackage{../../templates/style}

\begin{document}

\begin{problem}{โอลิมปิก (olympic)}{standard input}{standard output}{1 second}{32 megabytes}

หลังจบการแข่งขันโอลิมปิกฤดูร้อน ค.ศ. 2032 ทางประเทศเจ้าภาพต้องการจัดอันดับประเทศที่เข้าร่วมแข่งขันทั้งหมดตามผลการแข่งขัน โดยมีหลักเกณฑ์คือ หากประเทศหนึ่งได้รับเหรียญทอง $G$ เหรียญ เหรียญเงิน $S$ เหรียญ และเหรียญทองแดง $B$ เหรียญ ประเทศนั้นจะได้คะแนนเท่ากับ $GW_G + SW_S + BW_B$ โดยที่ $W_G, W_S, W_B$ เป็นจำนวนจริงบวกที่ $W_G ≥ W_S ≥ W_B$ การจัดอันดับประเทศจะเรียงตามคะแนนจากมากไปหาน้อย และหากมีประเทศมากกว่า $1$ ประเทศได้คะแนนเท่ากัน จะถือว่าประเทศเหล่านั้นได้อันดับที่ดีที่สุดร่วมกัน

ทางประเทศเจ้าภาพต้องการเลือกค่าถ่วงน้ำหนัก $W_G, W_S, W_B$ ที่เหมาะสม ที่จะทำให้ประเทศของตนอยู่ในอันดับที่ดีที่สุดที่เป็นไปได้

\bigskip
\underline{\textbf{โจทย์}}  จงเขียนโปรแกรมเพื่อรับจำนวนเหรียญทอง เหรียญเงิน และเหรียญทองแดงที่แต่ละประเทศได้รับ แล้วคำนวณหาอันดับที่ดีที่สุดที่เป็นไปได้ของประเทศเจ้าภาพ เมื่อเลือกค่าถ่วงน้ำหนักที่เหมาะสม


\InputFile

\textbf{บรรทัดแรก} ระบุจำนวนเต็ม $N$ $(2 \leq N \leq 1\,000)$ แทนจำนวนประเทศที่เข้าร่วมแข่งขัน

\textbf{บรรทัดที่ $2$ ถึง $N+1$} ในบรรทัดที่ $i+1$ $(1 \leq i \leq N)$ ระบุจำนวนเต็ม $G_i, S_i$ และ $B_i$ $(0 \leq G_i, S_i, B_i \leq 400)$ แทนจำนวนเหรียญทอง เหรียญเงิน และเหรียญทองแดงที่ประเทศที่ $i$ ได้รับ โดยประเทศที่ $1$ หมายถึงประเทศเจ้าภาพ และประเทศที่ $2, 3, 4$ ไปเรื่อยๆ จนถึง $N$ คือประเทศอื่นๆ ที่เข้าร่วมแข่งขัน


\OutputFile

\textbf{มีบรรทัดเดียว} ระบุอันดับที่ดีที่สุดที่เป็นไปได้ของประเทศเจ้าภาพ เมื่อเลือกค่าถ่วงน้ำหนักที่เหมาะสม

\Examples

\begin{example}
\exmp{4
1 2 3
2 3 4
3 0 0
10 10 10}{3}%
\exmp{5
3 3 0
2 5 0
4 1 0
50 0 0
0 0 50}{2}%
\end{example}

\Note 

\textbf{ในตัวอย่างที่ $1$} ค่าถ่วงน้ำหนักที่เป็นไปได้ เช่น $W_G = W_S = W_B = 1$ ซึ่งจะทำให้ประเทศเจ้าภาพได้อันดับ $3$ 

\textbf{ในต้วอย่างที่ $2$} ค่าถ่วงน้ำหนักที่เป็นไปได้ เช่น $W_G = 2, W_S = 1, W_B = 0.1$ ซึ่งจะทำให้ประเทศเจ้าภาพได้อันดับ $2$ ร่วมกับประเทศที่ $2$ และประเทศที่ $3$

\Source

สุธี เรืองวิเศษ

\end{problem}

\end{document}