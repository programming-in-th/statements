\documentclass[11pt,a4paper]{article}

\usepackage{../../templates/style}

\begin{document}

\begin{problem}{เติมวงเล็บ}{standard input}{standard output}{1 second}{32 megabytes}

ให้สตริงที่ประกอบไปด้วยวงเล็บเปิด ‘(‘ และวงเล็บปิด ‘)’  จงคำนวณหาจำนวนวงเล็บเปิดหรือปิดที่น้อยที่สุดที่ต้องเติมในสตริงดังกล่าว เพื่อทำให้วงเล็บเปิดและปิดจับคู่กันได้อย่างถูกต้อง  (สำหรับนิยามอย่างเป็นทางการของการจับคู่ได้อย่างถูกต้องดูได้จากส่วนอธิบายเพิ่มเติมท้ายโจทย์)

พิจารณาตัวอย่างตามตารางด้านล่าง

\begin{center}
\begin{tabular}{|c|c|c|}
\hline
สตริงตั้งต้น & การเติมที่น้อยที่สุดแบบหนึ่ง & จำนวนที่ต้องเติม\\
\hline\hline
(()( & (()\textbf{)}(\textbf{)} & 2\\
\hline
())( & \textbf{(}())(\textbf{)} & 2\\
\hline
((())()) & ((())()) & 0\\
\hline
(()()))()) & (()()\textbf{(}))()\textbf{(}) & 2\\
\hline
\end{tabular}
\end{center}

\bigskip
\underline{\textbf{โจทย์}}  เขียนโปรแกรมรับสตริงที่ประกอบด้วยวงเล็บเปิดและวง เล็บปิด จากนั้นคำนวณหาจำนวนวงเล็บที่ต้องเติมเข้าไปในสตริงดังกล่าว เพื่อให้เป็นสตริงที่วงเล็บเปิดและปิดจับคู่กันได้อย่างถูกต้อง


\InputFile

\textbf{มีบรรทัดเดียว} รับสตริงที่ประกอบด้วยวงเล็บเปิดและวงเล็บปิด ความยาวไม่เกิน $200$ ตัวอักษร


\OutputFile

\textbf{มีบรรทัดเดียว} แสดงจำนวนเต็มแทนจำนวนวงเล็บที่น้อยที่สุดที่ต้องเติมเข้าไปในสตริงเพื่อให้เป็นสตริงที่ถูกต้อง

\Examples

\begin{example}
\exmp{())(}{2}%
\exmp{(()()))())}{2}%
\end{example}

\Note \textit{(ไม่จำเป็นนักต่อการทำโจทย์)}

สำหรับสตริงที่ประกอบไปด้วยวงเล็บเปิดและวงเล็บปิด เราจะกล่าวว่าสตริงดังกล่าวมีการจับกันของวงเล็บเปิดและปิดอย่างถูกต้อง ก็ต่อเมื่อเราสามารถจับคู่วงเล็บเปิดกับวงเล็บปิดได้แบบ $1$ ต่อ $1$ โดยที่สอดคล้องกับเงื่อนไขต่อไปนี้:  ถ้าวงเล็บเปิดที่เป็นอักขระที่ $i$ จับคู่กับวงเล็บปิดที่เป็นอักขระที่ $j$ ในสตริง เราจะได้ว่า
\begin{itemize}

\item $i < j$   (วงเล็บเปิดอยู่หน้าวงเล็บปิด)
\item สำหรับวงเล็บเปิดใด ๆที่อยู่ที่ตำแหน่ง $a$ ที่ $i < a < j$, วงเล็บเปิดนั้นจะต้องจับคู่กับวงเล็บปิดที่อยู่ที่ตำแหน่ง $b$ ที่ $a < b < j$ เท่านั้น
\item ในทางกลับกัน วงเล็บปิดใด ๆ ที่อยู่ที่ตำแหน่ง $a$ ที่ $i < a < j$, วงเล็บปิดนั้นจะต้องจับคู่กับวงเล็บเปิดที่อยู่ที่ตำแหน่ง $b$ ที่ $i < b < a$ เท่านั้น เช่นกัน
\end{itemize}

\Source

IOI Thailand League 2010 เดือนพฤษภาคม

\end{problem}

\end{document}