\documentclass[11pt,a4paper]{article}

\usepackage{../../templates/style}

\begin{document}

\begin{problem}{Find the Distance}{standard input}{standard output}{1 second}{64 megabytes}

ถ้ามีเลขฐานสอง $2$ จำนวนซึ่งมีจำนวนหลักเท่ากันแล้ว \textit{Hamming distance} ของเลขสองจำนวนนี้ หาได้จาก จำนวนหลักที่มีเลขต่างกัน ตัวอย่างเช่น

$010010$\\
และ\\
$100010$

มี\textit{ Hamming distance} เป็น $2$ เนื่องจากหลักทางซ้ายสุด $2$ หลักแรกแตกต่างกัน 
\bigskip

อีกตัวอย่างหนึ่งคือ

$0111110$\\
และ\\
$0011100$

ก็จะมี \textit{Hamming distance} เป็น $2$ เช่นกัน

\bigskip
ถ้าพิจารณาเลขฐานสองที่มี $K$ หลัก และกำหนดให้ $N$ เป็นเลขจำนวนเต็มโดยที่ $N \leq 2^k-1$ แล้ว หน้าที่ของคุณคือให้หาผลรวมของ \textit{Hamming distance} ของค่า $0$ และ $1$, $1$ และ $2$, เรื่อยไปจนถึงระหว่าง $N-1$ และ $N$

ตัวอย่างเช่น ถ้า $K=3$ และ $N=4$ แล้วคำตอบคือ $7$ ซึ่งได้มาจากการหา \textit{Hamming distance} ดังนี้

$000$ และ $001$ (หรือจาก $0$ และ $1$ เมื่อเขียนเป็นเลขฐานสอง) มี \textit{Hamming distance} เป็น $1$\\
$001$ และ $010$ มี \textit{Hamming distance} เป็น $2$\\
$010$ และ $011$ มี \textit{Hamming distance} เป็น $1$\\
$011$ และ $100$ มี \textit{Hamming distance} เป็น $3$

ดังนั้นผลรวมของ \textit{Hamming distance} คือ $1+2+1+3 = 7$ นั่นเอง

\bigskip
\underline{\textbf{โจทย์}}  คุณจะได้รับค่า $K$ และ $N$ หน้าที่ของคุณคือต้องหาผลรวมของ\textit{ Hamming distance} ของเลขฐานสอง $K$ หลัก ระหว่างค่า $0$ และ $1$, $1$ และ $2$, $...$, $N-1$ และ $N$

\InputFile

\textbf{มีบรรทัดเดียว} รับจำนวนเต็มสองจำนวน $K$ $N$ $( K \leq 32; N \leq 2^{32}-1)$

\OutputFile

\textbf{มีบรรทัดเดียว} แสดงค่าผลรวมของ \textit{Hamming distance} ดังได้อธิบายไปแล้ว

\Examples

\begin{example}
\exmp{3 4}{7}%
\end{example}


\Source

Indian National Olympiad in Informatics
Online Programming Contest, 24-25 December 2005

\end{problem}

\end{document}