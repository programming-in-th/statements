\documentclass[11pt,a4paper]{article}

\usepackage{../../templates/style}

\begin{document}

\begin{problem}{สถานีอวกาศ (Space)}{standard input}{standard output}{1 second}{32 megabytes}

วิศวกรต้องการสร้างอาณานิคมแห่งหนึ่งในจักรวาล อาณานิคมนี้ประกอบด้วยสถานีอวกาศ (Space Station) ทั้งหมด $2^d$ สถานี โดยมีเลขประจำสถานี $d$ หลัก ที่ประกอบด้วยเลขศูนย์และเลขหนึ่งเท่านั้น เช่นในกรณีที่ $d = 2$ จำนวนของสถานีทั้งหมดจะเท่ากับ $4$ และมีเลขประจำสถานีคือ $00, 01, 10$ และ $11$ ในการสร้างอาณานิคมแห่งนี้ วิศวกรจะต้องสร้างเส้นทางเชื่อมระหว่างสถานี โดยมีกฎในการสร้างอยู่ว่าสถานีสองแห่งใดๆ จะมีเส้นทางเชื่อมต่อกันก็ต่อเมื่อ เลขประจำสถานีของทั้งสองสถานีแตกต่างกันอยู่หนึ่งหลักพอดี



\bigskip
\underline{\textbf{โจทย์}}  จงเขียนโปรแกรมในการสร้างเส้นทางเชื่อมเหล่านี้ โดยพิมพ์เส้นทางเชื่อมแต่ละเส้นทางเพียงครั้งเดียว สำหรับทางเชื่อมแต่ละทางให้แสดงสถานทีที่มีเลขน้อยกว่ามาก่อน สำหรับลำดับของสถานีในการแสดงเส้นทางเชื่อม ให้เรียงจากน้อยไปหามาก โดย $(x_1,y_1)$ จะน้อยกว่า $(x_2,y_2)$ ก็ต่อเมื่อ  $x_1 < x_2$   หรือ   $x_1 = x_2$ และ $y_1 < y_2$ 


\InputFile

\textbf{มีบรรทัดเดียว} รับตัวเลขจำนวนเต็มบวกที่แสดงค่าของ $d$ โดยที่ $2 \leq d \leq 14$

\OutputFile

\textbf{มี $d \times 2^{d-1}$ บรรทัด} แสดงเส้นทางเชื่อมทั้งหมด โดยให้แต่ละบรรทัดแสดงเส้นทางเชื่อมหนึ่งเส้น โดยแสดงเป็นเลขประจำสถานี $d$ หลักสองสถานีที่มีเส้นทางเชื่อมกันอยู่ โดยมีช่องว่างคั่นระหว่างตัวเลขสองตัวนั้น

\textbf{หมายเหตุ:} แนะนำให้ใช้  printf ในการแสดงผล

\Examples

\begin{example}
\exmp{2}{00 01
00 10
01 11
10 11}%
\end{example}

\begin{example}
\exmp{3}{	000 001
000 010
000 100
001 011
001 101
010 011
010 110
011 111
100 101
100 110
101 111
110 111}%
\end{example}


\Source

การแข่งขันคอมพิวเตอร์โอลิมปิกระดับชาติครั้งที่ 7 (NUTOI7) :: ดัดแปลงเล็กน้อย

\end{problem}

\end{document}