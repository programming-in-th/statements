\documentclass[11pt,a4paper]{article}

\usepackage{../../templates/style}

\begin{document}

\begin{problem}{Logic}{standard input}{standard output}{1 second}{16 megabytes}

ในทางตรรกศาสตร์ การเขียนข้อสรุป $C$ จากข้อสมมติ $A$ นั้นจะถูกต้องถ้าประโยคนั้นเป็นสัจนิรันดร์ (หรือในอีกทางหนึ่งก็คือ ในทุก ๆกรณีที่ $A$ เป็นจริง, $C$ จะต้องเป็นจริงด้วย)

เราจะใช้ตัวอักษรตั้งแต่ \textbf{a} ถึง \textbf{p} (รวม $16$ ตัว) แทนประพจน์ใด ๆ สำหรับการระบุข้อสมมติและข้อสรุปเราจะระบุด้วยวิธีการเขียนแบบ \textit{Conjunctive Normal Form} นั่นคือเราจะระบุนิพจน์ทางตรรกศาสตร์ในรูปของตัวดำเนินการ \textit{AND} กันของนิพจน์ที่ประกอบด้วยตัวแปรที่ \textit{OR} กัน  ตัวอย่างเช่น  \textbf{(a $\lor$ b $\lor$ $\neg$c) $\land$ (c $\lor$ $\neg$d) $\land$ (d $\lor$ $\neg$a)} เป็นการ \textit{AND} กันของสามนิพจน์คือ \textbf{(a $\lor$ b $\lor$ $\neg$c) , (c $\lor$ $\neg$d)} และ\textbf{ (d $\lor$ $\neg$a)} เราสังเกตได้ว่าแต่ละนิพจน์จะมีตัวดำเนินการ \textit{OR} กันของประพจน์ภายใน เราขอเรียกนิพจน์นี้ว่า \textit{clause} ในการเขียนนิพจน์ดังกล่าวข้างต้นในข้อมูลนำเข้าเราจะเขียนแยกเป็นสามบรรทัด บรรทัดละ \textit{clause} ดังนี้

3 +a +b –c  	คือ\textbf{ (a $\lor$ b $\lor$ $\neg$c)}\\
2 +c –d 		  คือ\textbf{ (c $\lor$ $\neg$d)}\\
2 +d –a    		  คือ\textbf{ (d $\lor$ $\neg$a)}\\

โดยในแต่ละบรรทัดจะขึ้นด้วยจำนวนเต็ม $c$ แทนจำนวนตัวแปรใน \textit{clause} นั้น  $(1 \leq c \leq 32)$ ตามด้วยสายอักขระความยาวสองตัวอักขระอีก $c$ สาย โดยแต่ละสายอักขระคั่นด้วยช่องว่างหนึ่งช่อง ทั้งนี้สายอักขระความยาวสองแต่ละสายอักขระจะขึ้นด้วยอักขระ + หรือ – โดย จะเป็นการระบุว่าตัวแปรที่ตามมานั้นไม่ใส่ หรือใส่ ตัวดำเนินการ \textit{NOT} ($\neg$)  อักขระถัดมาจะแทนตัวแปร

\bigskip
\underline{\textbf{โจทย์}}  คุณได้รับข้อสรุปและข้อสมมติมาหลายชุด จงเขียนโปรแกรมเพื่อตรวจสอบว่าข้อสรุปดังกล่าวนั้น ถูกต้องตามข้อสมมติหรือไม่?

\InputFile

\textbf{บรรทัดแรก} รับจำนวนเต็ม $k$ แทนจำนวนข้อมูลชุดทดสอบย่อย $(1 \leq k \leq 3)$  จากนั้นข้อมูลจะประกอบด้วยข้อมูลชุดทดสอบย่อย จำนวน $k$ ชุด เรียงกันไป   ข้อมูลชุดทดสอบย่อยแต่ละชุดจะไม่เกี่ยวข้องกัน

\textbf{บรรทัดแรกของแต่ละข้อมูลชุดทดสอบย่อย} รับจำนวนเต็ม $n$ และ $m$ แทนจำนวน \textit{clause} ของข้อสมมติ และจำนวน \textit{clause} ของข้อสรุปตามลำดับ $(1\leq n \leq 100; 1 \leq m \leq 100)$

\textbf{บรรทัดที่ $2$ ถึงบรรทัดที่ $n+1$ ในแต่ละข้อมูลชุดทดสอบย่อย} แต่ละบรรทัดแสดงแต่ละ clause ของข้อสมมติ  โดยรับจำนวนเต็ม $c$ และตัวอักขระอีก $2c$ ตัว โดยทุก ๆสองตัวให้คั่นด้วยช่องว่าง ตามเงื่อนไขที่ได้กล่าวไป

\textbf{บรรทัดที่ $n+2$ ถึงบรรทัดที่ $n+m+1$ ในแต่ละข้อมูลชุดทดสอบย่อย} แต่ละบรรทัดแสดงแต่ละ clause ของข้อข้อสรุปในรูปแบบเดียวกันกับข้อสมมติ โดยรับจำนวนเต็ม $c$ และตัวอักขระอีก $2c$ ตัว โดยทุก ๆสองตัวให้คั่นด้วยช่องว่าง ตามเงื่อนไขที่ได้กล่าวไป

\OutputFile

\textbf{มี $k$ บรรทัด} โดยในบรรทัดที่ $i$ จะมีข้อความว่า YES ถ้าข้อสรุปของข้อมูลชุดทดสอบย่อยที่ $i$ ถูกต้องตามข้อสมมติ  และจะมีข้อความว่า NO ถ้าไม่ถูกต้อง

\Examples

\begin{example}
\exmp{2
3 1
2 -a +b
2 -b +c
1 -c
1 –a
1 1
1 +b
1 -a}{YES
NO}%
\end{example}

\Note 

\textbf{ข้อมูลชุดทดสอบย่อยแรก}\\
ข้อสมมติ:  \textbf{($\neg$a $\lor$ b) $\land$ ($\neg$b $\lor$ c) $\land$ ($\neg$c)}\\
ข้อสรุป:  \textbf{$\neg$a} \textbf{(ถูกต้อง)}

\textbf{ข้อมูลชุดทดสอบย่อยที่สอง}\\
ข้อสมมติ:  \textbf{b}\\
ข้อสรุป:  \textbf{$\neg$a }   \textbf{(ไม่ถูกต้อง)}

\Source

สอบปฏิบัติครั้งที่ 1 ค่ายคัดเลือกผู้แทนประเทศไทยไปแข่งขันคอมพิวเตอร์โอลิมปิกระหว่างประเทศปี 2550 ค่ายที่ 1

\end{problem}

\end{document}